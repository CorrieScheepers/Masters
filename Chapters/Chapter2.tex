\chapter{Methodology}
\label{chap:2}
\ChapterPageStuff{2}

\section{Preamble} The literature in \Cref{chap:1} is used for the method to create a logging mechanism that can capture user-based activity logs to improve software maintenance by analysing the obtained logs. The web-based application system on this logging mechanism will be implemented on is an energy management system for the mining industry.\par In \Cref{Ch2:LoggingMechanism} the methodology to create a logging mechanism to capture user-generated events is discussed for web-based applications. The different functional requirements and interfaces are discussed in this section \cite{Anish2015}.\par In \Cref{ch2:system_utilisation_analysis} the methodology is discussed to analyse these obtained logs to improve software maintenance.

\section{Logging mechanism}\label{Ch2:LoggingMechanism} The logging mechanism will need to meet the requirements discussed in \Cref{sec:EventLogging} to capture the required logs to apply system utilisation analysis on it. \Cref{fig:CH2_SystemA_Arch_Design} is the design for the logging mechanism to capture the user's activities. In this figure, the logging mechanism is split up into two functional requirements parts (F/R) which consist of the client and server functional requirements.\par Each functional requirement has an interface requirement that transfers the data from one interface to another interface. These interfaces are labeled as I/F in \Cref{fig:CH2_SystemA_Arch_Design}. The \Cref{fig:CH2_SystemA_Arch_Design} is the client interface (F/R 1) and the server interface (F/R 2) that forms the entire logging mechanism to capture the user-based activity logs.

\begin{figure}[!htb] % An h :here, t: top, b: bottom.
	\centering % cent the figure
	\includegraphics[width=0.9\textwidth]{Chapter2/SystemA_Architecture_Diagram/SystemA_Architecture_Diagram.pdf}
	\caption[System A logging mechanism architecture design]
	{\textit{System A logging mechanism architecture design}}\label{fig:CH2_SystemA_Arch_Design}
\end{figure}

\clearpage

\subsection{Clients functional requirements}

The client's functional requirements (F/R 1) are where the user-based activity is triggered. In \Cref{fig:CH2_SystemA_Arch_Design} the client interface consists of three main functional requirements. These interfaces in \Cref{tbl:Ch2_Client_Functional_Requirements} parses the input from the user to create a basic user event log. 

\begin{table}[!htb]
	\centering
	\small
	\caption[Client functional requirements]
	{\textit{Client functional requirements (F/R 1)}}
	\label{tbl:Ch2_Client_Functional_Requirements}
	\begin{tabularx}{\textwidth}{|l|l|X|}
		\hline \textbf{Requirement ID} & \textbf{Name} & \textbf{Description} \\
		\hline F/R 1.1 & User & The user serves as the primary initiator of the activity events.\\
		\hline F/R 1.2 & User's device & The device that the user uses to access the website from where the activity events are generated.\\
		\hline F/R 1.3 & User-generated events & These are any activity events that the user started that need to communicate back to the server.\\
		\hline
	\end{tabularx}
\end{table}

\subsubsection{Requirements for an event log to be classified as a user-based activity}
The user is the initiator of the logging mechanism. Each action or event they trigger by interacting with the user interface on their device (F/R 1.2) can be a potential user-generated event. In \Cref{tbl:ch2_requirementsForUserActivtyEvent} is the sub-requirements for the user (F/R 1.1) which the event log should fulfill to be classified as an user-based activity log.

\begin{table}[!htb]
	\centering
	\small
	\caption[Requirements for an event to be a user-based activity]
	{\textit{Requirements for an event to be a user-based activity}}
	\label{tbl:ch2_requirementsForUserActivtyEvent}
	\begin{tabularx}{\textwidth}{|l|X|}
		\hline \textbf{Requirement ID} & \textbf{Description}\\
		\hline F/R 1.1.1 & The event has to be triggered by the user interacting with the user interface using their device. \\
		\hline F/R 1.1.2 & The event must consists of different cases ($ca~ \epsilon~CA$ the cases consists of events) which are noteworthy to make the event log identifiable \cite{Slaninova2014}. \\
		\hline F/R 1.1.3 & For certain types of event logs for F/R 1.1.2, the user-generated event should have an origin from which the event took place. \\
		\hline F/R 1.1.4 & The event log should consist of attributes that expands the identity of the user-based activity. \\
		\hline
	\end{tabularx}
\end{table}

Every interaction the user has with the user interface of the device to the software system can be seen as an event triggered by the user. Most of these events won't have meaningful impact as they won't fulfill F/R 1.1.2 and F/R 1.1.4 in \Cref{tbl:ch2_requirementsForUserActivtyEvent}.\par For the user activity event to meet the requirement of 1.1.2 it has to have defined cases that describe the activity type of each event. These activity types form the basic criteria for which event can be parsed which significantly reduces the number of logs that will be obtained. This will ensure that the event logging process will produce quality user-based logs as discussed in \Cref{sec:CH1_LoggingQuality}:

\begin{itemize}
	\item A basic structural complexity to simplify log parsing,
	\item Increase accuracy by only obtaining the logs that are part of the different defined cases,
	\item Keep the logging consistent by not deviating from the defined cases, and
	\item Ensure that the event log's other attributes are complete and available
\end{itemize}

\subsubsection{User activity types}
As previously stated this logging mechanism is for web-based applications, therefore the user activity types need to be established for these types of applications. Web applications consist of different page architectures. The Model-View-Controller (MVC) architecture is mostly used for web-based applications.\par The MVC architecture in \Cref{fig:ch2_flowMVC_Architecture} consists of 3 basic parts which are the \cite{Jailia2016}:

\begin{itemize}
	\item \textit{Model:} Is the representation of the records in the database which also interacts with the database through a database access layer or service.
	\item \textit{Controller:} Is operates both the \textit{View} and \textit{Model} and serves as the connection between the user and the system by controlling the data flow of the \textit{Model} and \textit{View}.
	\item \textit{View:} This shows the results of the data contained in the \textit{Model} and enables the user to manipulate the data.
\end{itemize}

\begin{figure}[!htb] % An h :here, t: top, b: bottom.
	\centering % cent the figure
	\includegraphics[width=0.95\textwidth]{Chapter2/Flow_MVC_Architecture/Flow_MVC_Architecture.pdf}
	\caption[Request flow in MVC architecture]
	{\textit{Request flow in MVC architecture \cite{Gu2010}}}\label{fig:ch2_flowMVC_Architecture}
\end{figure}

This flow of data in \Cref{fig:ch2_flowMVC_Architecture} depicts that any \textit{update}, \textit{create} and \textit{delete} event will have a request and response. Only the request can be used as that will possibly be a user-based activity that has been triggered by the user.\par The request meets the first requirement to be classified as a user-based activity (F/R 1.1.1) as the user will initiate an action to get a response back. Only the request will have the necessary data to complete the rest of the log attributes later in the logging mechanism.

The user-activity logs will be split into three main event types as in \Cref{tbl:Ch2_User_ActivityTypes}. The general user activity event type (F/R 1.2.3) will the be most common user activity event and be split up into different user activity events. This is determined by the need of what utilisation stage requires to analyse specific user activity events. 

\clearpage

\begin{table}[!htb]
	\centering
	\small
	\caption[User activity types]
	{\textit{User activity types}}
	\label{tbl:Ch2_User_ActivityTypes}
	\begin{tabularx}{\textwidth}{|l|l|X|}
		\hline \textbf{Requirement ID} & \textbf{Activity Type} & \textbf{Description} \\
		\hline F/R 1.2.1 & Web page accessed & The user may navigate through different web pages in a session.\\
		\hline F/R 1.2.2 & Session changes & This is any user activities that directly involve an extension session before their session times out after a period of inactivity. This also tracks when the session starts as any login attempt is a user-based activity.\\
		\hline F/R 1.2.3 & General activity & Any events excluding the first two user activity types that the user initiates when they interact with the web page. Most of the user activity logs will have this event type.\\ 
		\hline
	\end{tabularx}
\end{table}

\subsubsection{Logging points and log attributes}
In \Cref{sec:Ch1_LoggignPoints} the logging points should be strategically placed in the software system to capture the attributes for the user-based activity log. To meet the requirements of \Cref{tbl:ch2_requirementsForUserActivtyEvent} for a user-based activity the logging points can be placed in the software system:

\begin{table}[!htb]
	\centering
	\small
	\caption[Logging points requirements]
	{\textit{Logging points requirements}}
	\label{tbl:ch2_loggingPointRequirement}
	\begin{tabularx}{\textwidth}{|l|X|}
		\hline \textbf{Requirement ID} & \textbf{Description} \\
		\hline F/R 1.3.1 & The logging point should be placed where the user's interaction with the software system will send a \textit{request} back to the server.\\
		\hline F/R 1.3.2 & Each logging point should consistently capture the user-based activity. \\
		\hline F/R 1.3.3 & Logging points should be globally complete to capture the user-based activities in the giving software system without too much modification between each point in the same software system. \\
		\hline
	\end{tabularx}
\end{table}

The defined logging attributes in \Cref{tbl:Ch2_KeyLogging_Attributes} are the base attributes that form part of the main structure of the event log. 

\begin{table}[!htb]
	\centering
	\small
	\caption[Key logging attributes]
	{\textit{Key logging attributes}}
	\label{tbl:Ch2_KeyLogging_Attributes}
	\begin{tabularx}{\textwidth}{|l|l|X|}
		\hline \textbf{Requirement ID} & \textbf{Logging point} & \textbf{Description} \\
		\hline F/R 1.4.1 & Identification number & The activity identification is an incremental number of the event that is logged.\\
		\hline F/R 1.4.2 & Timestamp & This is the time the user initiated the user-based activity event.\\
		\hline F/R 1.4.3 & Activity type & Each event can be classified into types. This is the user activity types in \Cref{tbl:Ch2_User_ActivityTypes} \\
		\hline F/R 1.4.4 & User identification & Each user has a unique identification number that links the event to them. \\
		\hline F/R 1.4.5 & Request origin & In web applications, there are always requests sent back to the server and will call the primary function to handle the request. The file in which the function is the request origin. \\
		\hline F/R 1.4.6 & Metadata & The metadata of the event contains request parameters or other relevant request data of the event. This metadata adds more information about the user's activity. In \Cref{fig:Ch2_Metadata_Json_Example} is an example representation of the metadata. \\
		\hline F/R 1.4.7 & Miscellaneous attributes & These are any non-metadata attributes that can be consistently captured to be used in the utilasation analysis.\\ \hline
	\end{tabularx}
\end{table}

\clearpage

\subsubsection{Client functional requirements interaction}
\par In \Cref{fig:ch2_user_based_actvity_classification} is the complete process of the user interacting with the UI to trigger a user-based activity event to be logged later for the client's functional requirements. It starts with the user interacting with the user interface. The potential user activity needs to meet the first requirement (F/R 1.1.1) of \Cref{tbl:ch2_requirementsForUserActivtyEvent} or else it should be not be logged. The default activity type is set to general activity (F/R 1.2.3) until it is further processed later in the logging mechanism.\par If the activity has any additional metadata such as the request parameters, it will also be logged. The other metadata can also be captured in this stage from the client side like the element that the user clicked on to start the event.

\begin{figure}[!htb] % An h :here, t: top, b: bottom.
	\centering % cent the figure
	\includegraphics[width=0.6\textwidth]{Chapter2/client_functional_requirement_flow_diagram/client_functional_requirement_flow_diagram.pdf}
	\caption[User-based activity log classification flow diagram]
	{\textit{User-based activity log classification flow diagram}}\label{fig:ch2_user_based_actvity_classification}
\end{figure}

\clearpage

\subsection{Server's functional requirements}
The server functional requirements in \Cref{tbl:Ch2_Server_Functional_Requirements} for \Cref{fig:CH2_SystemA_Arch_Design} is the rest of the logging mechanism. At this stage, the obtained user-generated event of \Cref{fig:ch2_user_based_actvity_classification} will be attempted to be completed into a user-based activity log and stored in a database.. 

\begin{table}[!htb]
	\centering
	\small
	\caption[Server functional requirements]
	{\textit{Server functional requirements (F/R 2)}}
	\label{tbl:Ch2_Server_Functional_Requirements}
	\begin{tabularx}{\textwidth}{|l|l|X|}
		\hline \textbf{Requirement ID} & \textbf{Name} & \textbf{Description} \\
		\hline F/R 2.1 & User activity logger & The key logging points are used to capture and create the user-based event log that will be stored in a database.\\
		\hline F/R 2.2 & Database & The event log is stored in a database until it is needed for further analysis.\\
		\hline
	\end{tabularx}
\end{table}

Each of these log attributes combined creates the base log from which key logging points can be created in the software system to capture the user-based activity logs. The 

\begin{figure}[!htb]
	\centering
	\begin{lstlisting}[style=json] 
		{ "RequestOrigin" : "/Area4/Controller4",
		  "RequestElementID" : "Button4",
		  "RequestParameters": {
		  "Parameter1": 4,
		  "Parameter2": "Hello World!",
			"Parameter3": true
			"Parameter4": 40.404
		  }		
		}
	\end{lstlisting}
	\caption[Metadata JSON]
	{\textit{Metadata JSON}}\label{fig:Ch2_Metadata_Json_Example}
\end{figure}

The interfaces in \Cref{tbl:Ch2_Server_Interface_Requirements} for the client functional requirement (F/R 2) is to obtain the base log from the client functional requirement (F/R 1) and finally store the completed log into a database.

\begin{table}[!htb]
	\centering
	\small
	\caption[Server interface requirements]
	{\textit{Server interface requirements for F/R 2}}
	\label{tbl:Ch2_Server_Interface_Requirements}
	\begin{tabularx}{\textwidth}{|l|l|X|}
		\hline \textbf{Requirement ID} & \textbf{Name} & \textbf{Description} \\
		\hline I/F 2.1 & Log parsing to server & The captured log is parsed onto the server for further processing of the captured user-generated event.\\
		\hline I/F 2.2 & Store in database & The event log is sent to a database for storing.\\
		\hline
	\end{tabularx}
\end{table}

\clearpage

\section{System utilisation analysis}\label{ch2:system_utilisation_analysis}

\section{Integration}

\section{Conclusion}