\chapter{Methodology}
\label{chap:2}
\ChapterPageStuff{2}

\section{Preamble} The literature in \Cref{chap:1} is used for the method to create a logging mechanism that can capture user-based activity logs to improve software maintenance by analysing the obtained logs.\par In \Cref{Ch2:LoggingMechanism} the methodology to create a logging mechanism to capture user-generated events is discussed. The different functional requirements and interfaces are discussed in this section \cite{Anish2015}.\par In \Cref{ch2:system_utilisation_analysis} the methodology is discussed to analyse these obtained logs to improve software maintenance.

\section{Logging mechanism}\label{Ch2:LoggingMechanism} The logging mechanism will need to meet the requirements discussed in \Cref{sec:EventLogging} to capture the required logs to apply system utilisation analysis on it. \Cref{fig:CH2_SystemA_Arch_Design} is the design for the logging mechanism to capture the user's activities. In this figure, the logging mechanism is split up into two functional requirements parts (F/R) which consist of the client and server functional requirements.\par Each functional requirement has an interface requirement that transfers the data from one interface to another interface. These interfaces are labeled as I/F in \Cref{fig:CH2_SystemA_Arch_Design}. The \Cref{fig:CH2_SystemA_Arch_Design} is the client interface (F/R 1) and the server interface (F/R 2)

\begin{figure}[!htb] % An h :here, t: top, b: bottom.
	\centering % cent the figure
	\includegraphics[width=0.95\textwidth]{Chapter2/SystemA_Architecture_Diagram/SystemA_Architecture_Diagram.pdf}
	\caption[System A logging mechanism architecture design]
	{\textit{System A logging mechanism architecture design}}\label{fig:CH2_SystemA_Arch_Design}
\end{figure}

\clearpage

\subsection{Clients functional requirements}

The client's functional requirements (F/R 1) consist of the user's device interacting (I/F 1.1) with the system interface (F/R 2). Some of the log attributes are captured on the client-side. The log attributes are obtained by (I/F 1.2) to create the user-generated activity event (F/R 1.3). In \Cref{tbl:Ch2_Client_Functional_Requirements} is the functional requirements that are on the client-side of the user activity logging mechanism.

\begin{table}[!htb]
	\centering
	\small
	\caption[Client functional requirements]
	{\textit{Client functional requirements (F/R 1)}}
	\label{tbl:Ch2_Client_Functional_Requirements}
	\begin{tabularx}{\textwidth}{|l|l|X|}
		\hline \textbf{Requirement ID} & \textbf{Name} & \textbf{Description} \\
		\hline F/R 1.1 & User & The user serves as primary initiator of the activity events.\\
		\hline F/R 1.2 & User's device & The device that user uses to access the website from where the activity events are generated.\\
		\hline F/R 1.3 & User generated events & These are any activity events that user started that needs to communicate back to server.\\
		\hline
	\end{tabularx}
\end{table}

To generate the logs some of the 


\clearpage

The functional requirements in \Cref{tbl:Ch2_Client_Functional_Requirements} needs to interface with each other. These interfaces in \Cref{tbl:Ch2_Client_Interface_Requirements} parses the input from the user to create basic user event log.

\begin{table}[!htb]
	\centering
	\small
	\caption[Client interface requirements]
	{\textit{Client interface requirements for F/R 1}}
	\label{tbl:Ch2_Client_Interface_Requirements}
	\begin{tabularx}{\textwidth}{|l|l|X|}
		\hline \textbf{Requirement ID} & \textbf{Name} & \textbf{Description} \\
		\hline I/F 1.1 & User input & The user starts the activity events by using the user interface to give the website any input.\\
		\hline I/F 1.2 & Log parsing of logging points & The user generated event is captured to get some of the key logging points of \Cref{tbl:CH1_Log_Basic_Attributes}.\\
		\hline
	\end{tabularx}
\end{table}

The server functional requirements in \Cref{tbl:Ch2_Server_Functional_Requirements} captures the rest of the key logging points and complete the event log to be stored in a database. 

\begin{table}[!htb]
	\centering
	\small
	\caption[Server functional requirements]
	{\textit{Server functional requirements (F/R 2)}}
	\label{tbl:Ch2_Server_Functional_Requirements}
	\begin{tabularx}{\textwidth}{|l|l|X|}
		\hline \textbf{Requirement ID} & \textbf{Name} & \textbf{Description} \\
		\hline F/R 2.1 & User activity logger & The rest of the key logging points are captured and the event log is completed to stored in a database.\\
		\hline F/R 2.2 & Database & The event log is stored in a database until it is needed for further analysis.\\
		\hline
	\end{tabularx}
\end{table}

The interfaces in \Cref{tbl:Ch2_Server_Interface_Requirements} for the client functional requirement (F/R 2) is to obtain the base log from the client functional requirement (F/R 1) and finally store the completed log into a database.

\begin{table}[!htb]
	\centering
	\small
	\caption[Server interface requirements]
	{\textit{Server interface requirements for F/R 2}}
	\label{tbl:Ch2_Server_Interface_Requirements}
	\begin{tabularx}{\textwidth}{|l|l|X|}
		\hline \textbf{Requirement ID} & \textbf{Name} & \textbf{Description} \\
		\hline I/F 2.1 & Log parsing to server & The captured log is parsed onto the server for further processing of the captured user generated event.\\
		\hline I/F 2.2 & Store in database & The event log is send to a database for storing.\\
		\hline
	\end{tabularx}
\end{table}

\clearpage

\subsection{Logging points}
In \Cref{sec:Ch1_LoggignPoints} the basic log event attributes need to be identified to capture the needed logging points that will be used to create the event log. Tracking the events can be divided in different events types as in \Cref{tbl:Ch2_User_ActivityTypes}. \par Accessing a certain web page is used in the utilisation analysis to track which part of the website is being visited by the user. Certain session changes have direct input from the user like login in or clicking on a log out button. \par The rest of the user events are any other activities that the user does on the page excluding the previous two types of events in \Cref{tbl:Ch2_User_ActivityTypes}. These events can be further divided into other types for the system utilisation analysis.

\begin{table}[!htb]
	\centering
	\small
	\caption[User activity types]
	{User activity types}
	\label{tbl:Ch2_User_ActivityTypes}
	\begin{tabularx}{\textwidth}{|l|X|}
		\hline \textbf{Activity Type} & \textbf{Description} \\
		\hline Page accessed & The user may enter different web pages in a session. Tracking which pages the user navigates through.\\
		\hline Session changes & This is any user activities that directly involves extension session before their session times out after a period of inactivity. This also tracks when the session starts as any login attempt is a user based activity.\\
		\hline General user events & Any events that is other than accessing a certain web page that user initiates that communicates with the server. Most of the user activities will have this event type.\\ 
		\hline
	\end{tabularx}
\end{table}

\begin{table}[!htb]
	\centering
	\small
	\caption[Key logging points]
	{\textit{Key logging points}}
	\label{tbl:Ch2_KeyLogging_Points}
	\begin{tabularx}{\textwidth}{|l|X|}
		\hline \textbf{Logging point} & \textbf{Description} \\
		\hline Identification number & The activity identification is an incremental number of the event that is logged.\\
		\hline Timestamp & This is the time which the event took place.\\
		\hline Activity type & Each event can be classified into types. This is the user activity types in \Cref{tbl:Ch2_User_ActivityTypes} \\
		\hline User identification & Each user has a unique identification number that links the event to them. \\
		\hline Request origin & In web applications there are always requests send back to the server and will call the primary function to handle the request. The file which the function in is the request origin. \\
		\hline Meta data & The meta data of the event contains request parameters, the HTML element from which the request is initiated and other relevant request data of the event. This can also be other meta data is important to get that adds more information about the user's activity. In \Cref{fig:Ch2_Metadata_Json_Example}. is the representation of the meta data. \\
		\hline
	\end{tabularx}
\end{table}

\begin{figure}[!htb]
	\centering
	\begin{lstlisting}[style=json] 
		{ "RequestOrigin" : "/Area4/Controller4",
		  "RequestElementID" : "Button4",
		  "RequestParameters": {
		  "Parameter1": 4,
		  "Parameter2": "Hello World!",
			"Parameter3": true
			"Parameter4": 40.404
		  }		
		}
	\end{lstlisting}
	\caption[System B meta data JSON]
	{\textit{System B meta data JSON}}\label{fig:Ch2_Metadata_Json_Example}
\end{figure}

\section{System utilisation analysis}\label{ch2:system_utilisation_analysis}

\section{integration}

\section{Conclusion}