\cleardoublepage
\appendix

\chapter{Logging practise in software engineering}\label{apx:loggingPractice}
Providing a guide for software engineers and developers to implement a suitable logging mechanism in their software systems has proven to be a vital tool for both industrial use and academic progress \cite{Rong2018a}. \cite{Rong2018a} conducted a study to review published articles on logging practises and improve the performance and efficiency of logging implementation. From their study, they established selection criteria to include (as in \Cref{tbl:CH1_RongIncSelectionCriteria}) and exclude (as in \Cref{tbl:CH1_RongExlSelectionCriteria}) academic papers on logging practises \cite{Rong2018a,Rong2018}.\par Rong's selection criteria yielded numerous research articles on logging practises applied in the industry, either through creating a new logging mechanism or optimising existing logging mechanisms. Reviewing 41 identified articles, they found that many practitioners and researchers recognise the importance of logging practises in software engineering. However, there is a lack of guidance available to provide software engineers or developers with the tools necessary to create or improve efficient logging mechanisms \cite{Rong2018a,Zhu2015}.

\begin{table}[!htb]
	\centering
	\caption[G. Rong's inclusion selection criteria]
	{\textit{G. Rong's inclusion selection criteria \cite{Rong2018a}}}
	\label{tbl:CH1_RongIncSelectionCriteria}
	\begin{tabularx}{\textwidth}{cX}
            \toprule
            \thead{Identification} & \thead{Criteria} \\
            \midrule
            \rowcolor{lightgray}
            I1. & Publications that investigate the methodology for logging practice. \\
            I2. & Publications that investigate the tools, frameworks, and systems which support logging practice. \\
            \rowcolor{lightgray}
            I3. & Publications that propose a standard for logging practice.\\
            I4. & Publications that are peer-reviewed (conference paper, journal article). \\
            \rowcolor{lightgray}
            I5. & Publications that are primary studies on logging practice. \\		
            \bottomrule
	\end{tabularx}
\end{table}

\clearpage

\begin{table}[!htb]
	\centering
	\caption[G. Rong's exclusion selection criteria]
	{\textit{G. Rong's exclusion selection criteria \cite{Rong2018a}}}
	\label{tbl:CH1_RongExlSelectionCriteria}
	\begin{tabularx}{\textwidth}{cX}
            \toprule
            \thead{Identification} & \thead{Criteria} \\
            \midrule
            \rowcolor{lightgray}
            E1. & Publications that investigate log analysis. \\
            E2. & Publications that investigate the usage of logs. \\
            \rowcolor{lightgray}
            E3. & Publications that investigate the technologies on logging user behaviours. \\
            E4. & Publications that are not written in English. \\
            \rowcolor{lightgray}
            E5. & Additionally, short papers, demo or industry publications are excluded. \\	
            \bottomrule
	\end{tabularx}
\end{table}

\Cref{fig:PushblisedPapers} shows the distribution of the 41 published papers obtained for \cite{Rong2018a} research on logging practises. Event logging plays an increasingly important role in modern software systems; therefore, research on logging practises in software engineering has been on the rise between 1990 and 2017.

\begin{figure}[!htb] % An h :here, t: top, b: bottom.
	\centering % cent the figure
	\includegraphics[width=0.95\textwidth]{Chapter1/Ronga2018.pdf}
	\caption[The distribution of the papers’ published years]
	{\textit{The distribution of the papers’ published years \cite{Rong2018a}}} \label{fig:PushblisedPapers}
\end{figure} 

\chapter{Case study results}\label{apx:caseStudies}
\section{Case Study A}


    \begin{xltabular}{\textwidth}{|X|X|X|X|X|}
        \caption[Case study A results]
        {\textit{Case study A results}}
        \label{tbl:apx_projectA_Normilised} \\
        
        \hline
        \textbf{Subsystem} & \textbf{$P_N$}  & \textbf{$A_X$} & \textbf{$A_N$} & \textbf{$M_{PF}$} \\
        \hline
        \endfirsthead

        \multicolumn{5}{c}
        {\tablename\ \thetable{} -- continued from previous page} \\
        \hline
        \textbf{Subsystem} & \textbf{$P_N$}  & \textbf{$A_X$} & \textbf{$A_N$} & \textbf{$M_{PF}$} \\ 
        \endhead

        \multicolumn{5}{|r|}{{Continued on next page}} \\ \hline
        \endfoot

        \hline
        \endlastfoot
    $S_{538}$ & 1.0000 & 3189 & 0.5281 & 0.5281 \\ \hline
 $S_{413}$ & 0.5469 & 5086 & 0.8423 & 0.4606 \\ \hline
 $S_{570}$ & 0.4531 & 5459 & 0.9041 & 0.4097 \\ \hline
 $S_{582}$ & 0.2531 & 6038 & 1.0000 & 0.2531 \\ \hline
 $S_{412}$ & 0.8344 & 1663 & 0.2753 & 0.2297 \\ \hline
 $S_{546}$ & 0.6281 & 1353 & 0.2240 & 0.1407 \\ \hline
 $S_{580}$ & 0.2250 & 3577 & 0.5923 & 0.1333 \\ \hline
 $S_{454}$ & 0.4781 & 1445 & 0.2392 & 0.1144 \\ \hline
 $S_{577}$ & 0.2656 & 2268 & 0.3755 & 0.0997 \\ \hline
 $S_{593}$ & 0.2031 & 2715 & 0.4496 & 0.0913 \\ \hline
 $S_{568}$ & 0.4719 & 795 & 0.1315 & 0.0621 \\ \hline
 $S_{445}$ & 0.4813 & 705 & 0.1166 & 0.0561 \\ \hline
 $S_{735}$ & 0.2625 & 899 & 0.1487 & 0.0390 \\ \hline
 $S_{476}$ & 0.2844 & 578 & 0.0956 & 0.0272 \\ \hline
 $S_{729}$ & 0.2594 & 590 & 0.0976 & 0.0253 \\ \hline
 $S_{404}$ & 0.7000 & 205 & 0.0338 & 0.0237 \\ \hline
 $S_{441}$ & 0.5563 & 249 & 0.0411 & 0.0229 \\ \hline
 $S_{600}$ & 0.2375 & 489 & 0.0808 & 0.0192 \\ \hline
 $S_{458}$ & 0.4500 & 255 & 0.0421 & 0.0189 \\ \hline
 $S_{469}$ & 0.4656 & 217 & 0.0358 & 0.0167 \\ \hline
 $S_{435}$ & 0.5156 & 117 & 0.0192 & 0.0099 \\ \hline
 $S_{581}$ & 0.4750 & 122 & 0.0200 & 0.0095 \\ \hline
 $S_{755}$ & 0.2812 & 141 & 0.0232 & 0.0065 \\ \hline
 $S_{753}$ & 0.2375 & 157 & 0.0258 & 0.0061 \\ \hline
 $S_{616}$ & 0.3219 & 100 & 0.0164 & 0.0053 \\ \hline
 $S_{503}$ & 0.4500 & 61 & 0.0099 & 0.0045 \\ \hline
 $S_{754}$ & 0.4500 & 37 & 0.0060 & 0.0027 \\ \hline
 $S_{756}$ & 0.2375 & 61 & 0.0099 & 0.0024 \\ \hline
 $S_{736}$ & 0.4500 & 29 & 0.0046 & 0.0021 \\ \hline
 $S_{716}$ & 0.4500 & 26 & 0.0041 & 0.0019 \\ \hline
 $S_{428}$ & 0.2594 & 44 & 0.0071 & 0.0018 \\ \hline
 $S_{737}$ & 0.4500 & 23 & 0.0036 & 0.0016 \\ \hline
 $S_{668}$ & 0.2625 & 34 & 0.0055 & 0.0014 \\ \hline
 $S_{675}$ & 0.4531 & 20 & 0.0031 & 0.0014 \\ \hline
 $S_{663}$ & 0.2625 & 30 & 0.0048 & 0.0013 \\ \hline
 $S_{537}$ & 0.2219 & 35 & 0.0056 & 0.0012 \\ \hline
 $S_{667}$ & 0.4500 & 15 & 0.0023 & 0.0010 \\ \hline
 $S_{432}$ & 0.5156 & 13 & 0.0020 & 0.0010 \\ \hline
 $S_{618}$ & 0.2281 & 28 & 0.0045 & 0.0010 \\ \hline
 $S_{639}$ & 0.2625 & 24 & 0.0038 & 0.0010 \\ \hline
 $S_{544}$ & 0.4594 & 14 & 0.0022 & 0.0010 \\ \hline
 $S_{748}$ & 0.2812 & 22 & 0.0035 & 0.0010 \\ \hline
 $S_{536}$ & 0.2875 & 21 & 0.0033 & 0.0010 \\ \hline
 $S_{563}$ & 0.4688 & 13 & 0.0020 & 0.0009 \\ \hline
 $S_{679}$ & 0.4531 & 13 & 0.0020 & 0.0009 \\ \hline
 $S_{623}$ & 0.2625 & 21 & 0.0033 & 0.0009 \\ \hline
 $S_{578}$ & 0.4594 & 12 & 0.0018 & 0.0008 \\ \hline
 $S_{701}$ & 0.4531 & 12 & 0.0018 & 0.0008 \\ \hline
 $S_{680}$ & 0.4531 & 12 & 0.0018 & 0.0008 \\ \hline
 $S_{496}$ & 0.4813 & 11 & 0.0017 & 0.0008 \\ \hline
 $S_{757}$ & 0.2625 & 18 & 0.0028 & 0.0007 \\ \hline
 $S_{510}$ & 0.4844 & 10 & 0.0015 & 0.0007 \\ \hline
 $S_{738}$ & 0.2250 & 19 & 0.0030 & 0.0007 \\ \hline
 $S_{614}$ & 0.2594 & 16 & 0.0025 & 0.0006 \\ \hline
 $S_{468}$ & 0.4844 & 9 & 0.0013 & 0.0006 \\ \hline
 $S_{311}$ & 0.4625 & 9 & 0.0013 & 0.0006 \\ \hline
 $S_{459}$ & 0.4906 & 8 & 0.0012 & 0.0006 \\ \hline
 $S_{571}$ & 0.2188 & 14 & 0.0022 & 0.0005 \\ \hline
 $S_{629}$ & 0.4594 & 7 & 0.0010 & 0.0005 \\ \hline
 $S_{547}$ & 0.4562 & 7 & 0.0010 & 0.0005 \\ \hline
 $S_{464}$ & 0.5094 & 6 & 0.0008 & 0.0004 \\ \hline
 $S_{559}$ & 0.4750 & 6 & 0.0008 & 0.0004 \\ \hline
 $S_{500}$ & 0.2625 & 10 & 0.0015 & 0.0004 \\ \hline
 $S_{742}$ & 0.0906 & 27 & 0.0043 & 0.0004 \\ \hline
 $S_{504}$ & 0.5312 & 5 & 0.0007 & 0.0004 \\ \hline
 $S_{508}$ & 0.5000 & 5 & 0.0007 & 0.0003 \\ \hline
 $S_{567}$ & 0.4688 & 5 & 0.0007 & 0.0003 \\ \hline
 $S_{566}$ & 0.4531 & 5 & 0.0007 & 0.0003 \\ \hline
 $S_{681}$ & 0.4531 & 5 & 0.0007 & 0.0003 \\ \hline
 $S_{727}$ & 0.4500 & 5 & 0.0007 & 0.0003 \\ \hline
 $S_{658}$ & 0.2250 & 8 & 0.0012 & 0.0003 \\ \hline
 $S_{659}$ & 0.2625 & 7 & 0.0010 & 0.0003 \\ \hline
 $S_{507}$ & 0.4875 & 4 & 0.0005 & 0.0002 \\ \hline
 $S_{620}$ & 0.4531 & 4 & 0.0005 & 0.0002 \\ \hline
 $S_{677}$ & 0.4500 & 4 & 0.0005 & 0.0002 \\ \hline
 $S_{740}$ & 0.0969 & 13 & 0.0020 & 0.0002 \\ \hline
 $S_{517}$ & 0.4500 & 3 & 0.0003 & 0.0001 \\ \hline
 $S_{709}$ & 0.4500 & 3 & 0.0003 & 0.0001 \\ \hline
 $S_{493}$ & 0.5031 & 2 & 0.0002 & 0.0001 \\ \hline
 $S_{506}$ & 0.4906 & 2 & 0.0002 & 0.0001 \\ \hline
 $S_{467}$ & 0.4781 & 2 & 0.0002 & 0.0001 \\ \hline
 $S_{541}$ & 0.4531 & 2 & 0.0002 & 0.0001 \\ \hline
 $S_{657}$ & 0.4500 & 2 & 0.0002 & 0.0001 \\ \hline
 $S_{688}$ & 0.4500 & 2 & 0.0002 & 0.0001 \\ \hline
 $S_{694}$ & 0.4500 & 2 & 0.0002 & 0.0001 \\ \hline
 $S_{739}$ & 0.4500 & 2 & 0.0002 & 0.0001 \\ \hline
 $S_{470}$ & 0.0469 & 7 & 0.0010 & 0.0000 \\ \hline
 $S_{477}$ & 0.2500 & 2 & 0.0002 & 0.0000 \\ \hline
 $S_{583}$ & 0.2500 & 2 & 0.0002 & 0.0000 \\ \hline
 $S_{481}$ & 0.2469 & 2 & 0.0002 & 0.0000 \\ \hline
 $S_{466}$ & 0.4813 & 1 & 0.0000 & 0.0000 \\ \hline
 $S_{485}$ & 0.0000 & 4 & 0.0005 & 0.0000 \\ \hline
 $S_{732}$ & 0.4500 & 1 & 0.0000 & 0.0000 \\ \hline
 $S_{465}$ & 0.4781 & 1 & 0.0000 & 0.0000 \\ \hline
 $S_{576}$ & 0.4500 & 1 & 0.0000 & 0.0000 \\ \hline
 $S_{725}$ & 0.4500 & 1 & 0.0000 & 0.0000 \\ \hline
 $S_{711}$ & 0.2437 & 1 & 0.0000 & 0.0000 \\ \hline
 $S_{699}$ & 0.4500 & 1 & 0.0000 & 0.0000 \\ \hline
 $S_{693}$ & 0.4500 & 1 & 0.0000 & 0.0000 \\ \hline
 $S_{505}$ & 0.4906 & 1 & 0.0000 & 0.0000 \\ \hline
 $S_{676}$ & 0.4594 & 1 & 0.0000 & 0.0000 \\ \hline
 $S_{513}$ & 0.4844 & 1 & 0.0000 & 0.0000 \\ \hline
 $S_{514}$ & 0.4844 & 1 & 0.0000 & 0.0000 \\ \hline
 $S_{661}$ & 0.2437 & 1 & 0.0000 & 0.0000 \\ \hline
 $S_{515}$ & 0.5125 & 1 & 0.0000 & 0.0000 \\ \hline
 $S_{520}$ & 0.4750 & 1 & 0.0000 & 0.0000 \\ \hline
 $S_{528}$ & 0.5156 & 1 & 0.0000 & 0.0000 \\ \hline
 $S_{530}$ & 0.2469 & 1 & 0.0000 & 0.0000 \\ \hline
 $S_{607}$ & 0.4562 & 1 & 0.0000 & 0.0000 \\ \hline
 $S_{543}$ & 0.4500 & 1 & 0.0000 & 0.0000 \\ \hline
 $S_{573}$ & 0.2469 & 1 & 0.0000 & 0.0000 \\ \hline
    \end{xltabular}
    

\clearpage

\section{Case Study B}


    \begin{small}
    \begin{xltabular}{\textwidth}{|X|X|X|X|X|}
        \caption[Case study B results]
        {\textit{Case study B results}}
        \label{tbl:apx_projectB_Normilised} \\
        
        \hline
        \textbf{Subsystem ID} & \textbf{$P_N$}  & \textbf{$A_X$} & \textbf{$A_N$} & \textbf{$M_{PF}$} \\
        \hline
        \endfirsthead

        \multicolumn{5}{c}
        {\tablename\ \thetable{} -- continued from previous page} \\
        \hline
        \textbf{Subsystem ID} & \textbf{$P_N$}  & \textbf{$A_X$} & \textbf{$A_N$} & \textbf{$M_{PF}$} \\ 
        \endhead

        \multicolumn{5}{|r|}{{Continued on next page}} \\ \hline
        \endfoot

        \hline
        \endlastfoot
    311 & 0.4642 & 9 & 0.0013 & 0.0006 \\ \hline
 404 & 0.7009 & 205 & 0.0338 & 0.0237 \\ \hline
 412 & 0.8349 & 1663 & 0.2753 & 0.2298 \\ \hline
 413 & 0.5483 & 5086 & 0.8423 & 0.4618 \\ \hline
 428 & 0.2617 & 44 & 0.0071 & 0.0019 \\ \hline
 432 & 0.5171 & 13 & 0.0020 & 0.0010 \\ \hline
 435 & 0.5171 & 117 & 0.0192 & 0.0099 \\ \hline
 441 & 0.5576 & 249 & 0.0411 & 0.0229 \\ \hline
 445 & 0.4829 & 705 & 0.1166 & 0.0563 \\ \hline
 454 & 0.4798 & 1445 & 0.2392 & 0.1148 \\ \hline
 458 & 0.4517 & 255 & 0.0421 & 0.0190 \\ \hline
 459 & 0.4922 & 8 & 0.0012 & 0.0006 \\ \hline
 464 & 0.5109 & 6 & 0.0008 & 0.0004 \\ \hline
 465 & 0.4798 & 1 & 0.0000 & 0.0000 \\ \hline
 466 & 0.4829 & 1 & 0.0000 & 0.0000 \\ \hline
 467 & 0.4798 & 2 & 0.0002 & 0.0001 \\ \hline
 468 & 0.4860 & 9 & 0.0013 & 0.0006 \\ \hline
 469 & 0.4673 & 217 & 0.0358 & 0.0167 \\ \hline
 470 & 0.0498 & 7 & 0.0010 & 0.0000 \\ \hline
 476 & 0.2866 & 578 & 0.0956 & 0.0274 \\ \hline
 477 & 0.2523 & 2 & 0.0002 & 0.0000 \\ \hline
 481 & 0.2492 & 2 & 0.0002 & 0.0000 \\ \hline
 485 & 0.0031 & 4 & 0.0005 & 0.0000 \\ \hline
 493 & 0.5047 & 2 & 0.0002 & 0.0001 \\ \hline
 496 & 0.4829 & 11 & 0.0017 & 0.0008 \\ \hline
 500 & 0.2648 & 10 & 0.0015 & 0.0004 \\ \hline
 503 & 0.4517 & 61 & 0.0099 & 0.0045 \\ \hline
 504 & 0.5327 & 5 & 0.0007 & 0.0004 \\ \hline
 505 & 0.4922 & 1 & 0.0000 & 0.0000 \\ \hline
 506 & 0.4922 & 2 & 0.0002 & 0.0001 \\ \hline
 507 & 0.4891 & 4 & 0.0005 & 0.0002 \\ \hline
 508 & 0.5016 & 5 & 0.0007 & 0.0003 \\ \hline
 510 & 0.4860 & 10 & 0.0015 & 0.0007 \\ \hline
 513 & 0.4860 & 1 & 0.0000 & 0.0000 \\ \hline
 514 & 0.4860 & 1 & 0.0000 & 0.0000 \\ \hline
 515 & 0.5140 & 1 & 0.0000 & 0.0000 \\ \hline
 517 & 0.4517 & 3 & 0.0003 & 0.0001 \\ \hline
 520 & 0.4766 & 1 & 0.0000 & 0.0000 \\ \hline
 528 & 0.5171 & 1 & 0.0000 & 0.0000 \\ \hline
 530 & 0.2492 & 1 & 0.0000 & 0.0000 \\ \hline
 536 & 0.2897 & 21 & 0.0033 & 0.0010 \\ \hline
 537 & 0.2243 & 35 & 0.0056 & 0.0013 \\ \hline
 538 & 1.0000 & 3189 & 0.5281 & 0.5281 \\ \hline
 541 & 0.4548 & 2 & 0.0002 & 0.0001 \\ \hline
 543 & 0.4517 & 1 & 0.0000 & 0.0000 \\ \hline
 544 & 0.4611 & 14 & 0.0022 & 0.0010 \\ \hline
 546 & 0.6293 & 1353 & 0.2240 & 0.1409 \\ \hline
 547 & 0.4579 & 7 & 0.0010 & 0.0005 \\ \hline
 559 & 0.4766 & 6 & 0.0008 & 0.0004 \\ \hline
 563 & 0.4704 & 13 & 0.0020 & 0.0009 \\ \hline
 566 & 0.4548 & 5 & 0.0007 & 0.0003 \\ \hline
 567 & 0.4704 & 5 & 0.0007 & 0.0003 \\ \hline
 568 & 0.4735 & 795 & 0.1315 & 0.0623 \\ \hline
 570 & 0.4548 & 5459 & 0.9041 & 0.4112 \\ \hline
 571 & 0.2212 & 14 & 0.0022 & 0.0005 \\ \hline
 573 & 0.2492 & 1 & 0.0000 & 0.0000 \\ \hline
 576 & 0.4517 & 1 & 0.0000 & 0.0000 \\ \hline
 577 & 0.2679 & 2268 & 0.3755 & 0.1006 \\ \hline
 578 & 0.4611 & 12 & 0.0018 & 0.0008 \\ \hline
 580 & 0.2274 & 3577 & 0.5923 & 0.1347 \\ \hline
 581 & 0.4766 & 122 & 0.0200 & 0.0096 \\ \hline
 582 & 0.2555 & 6038 & 1.0000 & 0.2555 \\ \hline
 583 & 0.2523 & 2 & 0.0002 & 0.0000 \\ \hline
 593 & 0.2056 & 2715 & 0.4496 & 0.0924 \\ \hline
 600 & 0.2399 & 489 & 0.0808 & 0.0194 \\ \hline
 607 & 0.4579 & 1 & 0.0000 & 0.0000 \\ \hline
 614 & 0.2617 & 16 & 0.0025 & 0.0007 \\ \hline
 616 & 0.3240 & 100 & 0.0164 & 0.0053 \\ \hline
 618 & 0.2305 & 28 & 0.0045 & 0.0010 \\ \hline
 620 & 0.4548 & 4 & 0.0005 & 0.0002 \\ \hline
 623 & 0.2648 & 21 & 0.0033 & 0.0009 \\ \hline
 629 & 0.4611 & 7 & 0.0010 & 0.0005 \\ \hline
 639 & 0.2648 & 24 & 0.0038 & 0.0010 \\ \hline
 657 & 0.4517 & 2 & 0.0002 & 0.0001 \\ \hline
 658 & 0.2274 & 8 & 0.0012 & 0.0003 \\ \hline
 659 & 0.2648 & 7 & 0.0010 & 0.0003 \\ \hline
 661 & 0.2461 & 1 & 0.0000 & 0.0000 \\ \hline
 663 & 0.2648 & 30 & 0.0048 & 0.0013 \\ \hline
 667 & 0.4517 & 15 & 0.0023 & 0.0010 \\ \hline
 668 & 0.2648 & 34 & 0.0055 & 0.0014 \\ \hline
 675 & 0.4548 & 20 & 0.0031 & 0.0014 \\ \hline
 676 & 0.4611 & 1 & 0.0000 & 0.0000 \\ \hline
 677 & 0.4517 & 4 & 0.0005 & 0.0002 \\ \hline
 679 & 0.4548 & 13 & 0.0020 & 0.0009 \\ \hline
 680 & 0.4548 & 12 & 0.0018 & 0.0008 \\ \hline
 681 & 0.4548 & 5 & 0.0007 & 0.0003 \\ \hline
 688 & 0.4517 & 2 & 0.0002 & 0.0001 \\ \hline
 693 & 0.4517 & 1 & 0.0000 & 0.0000 \\ \hline
 694 & 0.4517 & 2 & 0.0002 & 0.0001 \\ \hline
 699 & 0.4517 & 1 & 0.0000 & 0.0000 \\ \hline
 701 & 0.4548 & 12 & 0.0018 & 0.0008 \\ \hline
 709 & 0.4517 & 3 & 0.0003 & 0.0001 \\ \hline
 711 & 0.2461 & 1 & 0.0000 & 0.0000 \\ \hline
 716 & 0.4517 & 26 & 0.0041 & 0.0019 \\ \hline
 725 & 0.4517 & 1 & 0.0000 & 0.0000 \\ \hline
 727 & 0.4517 & 5 & 0.0007 & 0.0003 \\ \hline
 729 & 0.2617 & 590 & 0.0976 & 0.0255 \\ \hline
 732 & 0.4517 & 1 & 0.0000 & 0.0000 \\ \hline
 735 & 0.2648 & 899 & 0.1487 & 0.0394 \\ \hline
 736 & 0.4517 & 29 & 0.0046 & 0.0021 \\ \hline
 737 & 0.4517 & 23 & 0.0036 & 0.0016 \\ \hline
 738 & 0.2274 & 19 & 0.0030 & 0.0007 \\ \hline
 739 & 0.4517 & 2 & 0.0002 & 0.0001 \\ \hline
 740 & 0.0997 & 13 & 0.0020 & 0.0002 \\ \hline
 742 & 0.0935 & 27 & 0.0043 & 0.0004 \\ \hline
 748 & 0.2835 & 22 & 0.0035 & 0.0010 \\ \hline
 753 & 0.2399 & 157 & 0.0258 & 0.0062 \\ \hline
 754 & 0.4517 & 37 & 0.0060 & 0.0027 \\ \hline
 755 & 0.2835 & 141 & 0.0232 & 0.0066 \\ \hline
 756 & 0.2399 & 61 & 0.0099 & 0.0024 \\ \hline
 757 & 0.2648 & 18 & 0.0028 & 0.0007 \\ \hline
    \end{xltabular}
    \end{small}
    

\clearpage


\section{Case Study C}


    \begin{xltabular}{\textwidth}{XXXXXXX}
        \caption[Case study C results]
        {\textit{Case study C results}}
        \label{tbl:apx_projectC_Normilised} \\
        \toprule
         \thead{$S_{X}$} & \thead{$P_X$} & \thead{$P_N$}  & \thead{$A_X$} & \thead{$A_N$} & \thead{$M_{PF}$} & \thead{$P_{R}$} \\
        \midrule
        \endfirsthead

        \multicolumn{7}{c}
        {\tablename\ \thetable{} -- continued from previous page} \\
        \midrule
        \thead{$S_{X}$} & \thead{$P_X$} & \thead{$P_N$}  & \thead{$A_X$} & \thead{$A_N$} & \thead{$M_{PF}$} & \thead{$P_{R}$} \\
        \midrule
        \endhead

        \midrule
        \multicolumn{7}{r}{{Continued on next page}} \\ \midrule
        \endfoot
        \endlastfoot
    \rowcolor{lightgray} $S_{97}$ & 145 & 1.0000 & 84494 & 1.0000 & 1.0000 & 1 \\ 
  $S_{93}$ & 144 & 0.9931 & 71516 & 0.8464 & 0.8405 & 2 \\ 
 \rowcolor{lightgray} $S_{12}$ & 86 & 0.5903 & 14312 & 0.1694 & 0.1000 & 3 \\ 
  $S_{8}$ & 82 & 0.5625 & 14899 & 0.1763 & 0.0992 & 4 \\ 
 \rowcolor{lightgray} $S_{5}$ & 43 & 0.2917 & 27985 & 0.3312 & 0.0966 & 5 \\ 
  $S_{6}$ & 103 & 0.7083 & 7257 & 0.0859 & 0.0608 & 6 \\ 
 \rowcolor{lightgray} $S_{1}$ & 126 & 0.8681 & 5580 & 0.0660 & 0.0573 & 7 \\ 
  $S_{82}$ & 126 & 0.8681 & 2461 & 0.0291 & 0.0253 & 8 \\ 
 \rowcolor{lightgray} $S_{14}$ & 76 & 0.5208 & 3307 & 0.0391 & 0.0204 & 9 \\ 
  $S_{90}$ & 113 & 0.7778 & 1951 & 0.0231 & 0.0180 & 10 \\ 
 \rowcolor{lightgray} $S_{13}$ & 48 & 0.3264 & 4473 & 0.0529 & 0.0173 & 11 \\ 
  $S_{92}$ & 26 & 0.1736 & 3034 & 0.0359 & 0.0062 & 12 \\ 
 \rowcolor{lightgray} $S_{11}$ & 37 & 0.2500 & 1886 & 0.0223 & 0.0056 & 13 \\ 
  $S_{10}$ & 41 & 0.2778 & 1593 & 0.0188 & 0.0052 & 14 \\ 
 \rowcolor{lightgray} $S_{46}$ & 61 & 0.4167 & 800 & 0.0095 & 0.0039 & 15 \\ 
  $S_{67}$ & 18 & 0.1181 & 1644 & 0.0194 & 0.0023 & 16 \\ 
 \rowcolor{lightgray} $S_{7}$ & 41 & 0.2778 & 594 & 0.0070 & 0.0019 & 17 \\ 
  $S_{91}$ & 95 & 0.6528 & 246 & 0.0029 & 0.0019 & 18 \\ 
 \rowcolor{lightgray} $S_{70}$ & 13 & 0.0833 & 1662 & 0.0197 & 0.0016 & 19 \\ 
  $S_{95}$ & 24 & 0.1597 & 647 & 0.0076 & 0.0012 & 20 \\ 
 \rowcolor{lightgray} $S_{39}$ & 7 & 0.0417 & 2015 & 0.0238 & 0.0010 & 21 \\ 
  $S_{96}$ & 33 & 0.2222 & 258 & 0.0030 & 0.0007 & 22 \\ 
 \rowcolor{lightgray} $S_{79}$ & 6 & 0.0347 & 1535 & 0.0182 & 0.0006 & 23 \\ 
  $S_{64}$ & 10 & 0.0625 & 850 & 0.0100 & 0.0006 & 24 \\ 
 \rowcolor{lightgray} $S_{51}$ & 4 & 0.0208 & 2070 & 0.0245 & 0.0005 & 25 \\ 
  $S_{3}$ & 40 & 0.2708 & 155 & 0.0018 & 0.0005 & 26 \\ 
 \rowcolor{lightgray} $S_{2}$ & 14 & 0.0903 & 400 & 0.0047 & 0.0004 & 27 \\ 
  $S_{15}$ & 18 & 0.1181 & 303 & 0.0036 & 0.0004 & 28 \\ 
 \rowcolor{lightgray} $S_{65}$ & 14 & 0.0903 & 391 & 0.0046 & 0.0004 & 29 \\ 
  $S_{54}$ & 8 & 0.0486 & 477 & 0.0056 & 0.0003 & 30 \\ 
 \rowcolor{lightgray} $S_{30}$ & 5 & 0.0278 & 655 & 0.0077 & 0.0002 & 31 \\ 
  $S_{58}$ & 9 & 0.0556 & 304 & 0.0036 & 0.0002 & 32 \\ 
 \rowcolor{lightgray} $S_{45}$ & 36 & 0.2431 & 57 & 0.0007 & 0.0002 & 33 \\ 
  $S_{76}$ & 9 & 0.0556 & 214 & 0.0025 & 0.0001 & 34 \\ 
 \rowcolor{lightgray} $S_{77}$ & 10 & 0.0625 & 122 & 0.0014 & 0.0001 & 35 \\ 
  $S_{80}$ & 7 & 0.0417 & 179 & 0.0021 & 0.0001 & 36 \\ 
 \rowcolor{lightgray} $S_{24}$ & 5 & 0.0278 & 208 & 0.0024 & 0.0001 & 37 \\ 
  $S_{60}$ & 6 & 0.0347 & 142 & 0.0017 & 0.0001 & 38 \\ 
 \rowcolor{lightgray} $S_{87}$ & 4 & 0.0208 & 193 & 0.0023 & 0.0000 & 39 \\ 
  $S_{63}$ & 4 & 0.0208 & 174 & 0.0020 & 0.0000 & 40 \\ 
 \rowcolor{lightgray} $S_{59}$ & 6 & 0.0347 & 103 & 0.0012 & 0.0000 & 41 \\ 
  $S_{9}$ & 12 & 0.0764 & 45 & 0.0005 & 0.0000 & 42 \\ 
 \rowcolor{lightgray} $S_{53}$ & 5 & 0.0278 & 115 & 0.0013 & 0.0000 & 43 \\ 
  $S_{33}$ & 4 & 0.0208 & 125 & 0.0015 & 0.0000 & 44 \\ 
 \rowcolor{lightgray} $S_{72}$ & 4 & 0.0208 & 117 & 0.0014 & 0.0000 & 45 \\ 
  $S_{68}$ & 4 & 0.0208 & 88 & 0.0010 & 0.0000 & 46 \\ 
 \rowcolor{lightgray} $S_{17}$ & 3 & 0.0139 & 130 & 0.0015 & 0.0000 & 47 \\ 
  $S_{83}$ & 13 & 0.0833 & 22 & 0.0002 & 0.0000 & 48 \\ 
 \rowcolor{lightgray} $S_{44}$ & 4 & 0.0208 & 82 & 0.0010 & 0.0000 & 49 \\ 
  $S_{71}$ & 4 & 0.0208 & 73 & 0.0009 & 0.0000 & 50 \\ 
 \rowcolor{lightgray} $S_{85}$ & 5 & 0.0278 & 49 & 0.0006 & 0.0000 & 51 \\ 
  $S_{69}$ & 4 & 0.0208 & 65 & 0.0008 & 0.0000 & 51 \\ 
 \rowcolor{lightgray} $S_{31}$ & 4 & 0.0208 & 53 & 0.0006 & 0.0000 & 53 \\ 
  $S_{62}$ & 3 & 0.0139 & 79 & 0.0009 & 0.0000 & 53 \\ 
 \rowcolor{lightgray} $S_{78}$ & 3 & 0.0139 & 70 & 0.0008 & 0.0000 & 55 \\ 
  $S_{66}$ & 7 & 0.0417 & 23 & 0.0003 & 0.0000 & 56 \\ 
 \rowcolor{lightgray} $S_{16}$ & 3 & 0.0139 & 57 & 0.0007 & 0.0000 & 57 \\ 
  $S_{75}$ & 6 & 0.0347 & 23 & 0.0003 & 0.0000 & 58 \\ 
 \rowcolor{lightgray} $S_{29}$ & 3 & 0.0139 & 43 & 0.0005 & 0.0000 & 59 \\ 
  $S_{89}$ & 9 & 0.0556 & 11 & 0.0001 & 0.0000 & 60 \\ 
 \rowcolor{lightgray} $S_{42}$ & 2 & 0.0069 & 51 & 0.0006 & 0.0000 & 61 \\ 
  $S_{74}$ & 2 & 0.0069 & 48 & 0.0006 & 0.0000 & 62 \\ 
 \rowcolor{lightgray} $S_{84}$ & 2 & 0.0069 & 37 & 0.0004 & 0.0000 & 63 \\ 
  $S_{38}$ & 2 & 0.0069 & 33 & 0.0004 & 0.0000 & 64 \\ 
 \rowcolor{lightgray} $S_{19}$ & 3 & 0.0139 & 15 & 0.0002 & 0.0000 & 65 \\ 
  $S_{22}$ & 2 & 0.0069 & 24 & 0.0003 & 0.0000 & 66 \\ 
 \rowcolor{lightgray} $S_{23}$ & 3 & 0.0139 & 10 & 0.0001 & 0.0000 & 67 \\ 
  $S_{47}$ & 3 & 0.0139 & 9 & 0.0001 & 0.0000 & 68 \\ 
 \rowcolor{lightgray} $S_{28}$ & 2 & 0.0069 & 14 & 0.0002 & 0.0000 & 69 \\ 
  $S_{50}$ & 3 & 0.0139 & 6 & 0.0001 & 0.0000 & 70 \\ 
 \rowcolor{lightgray} $S_{32}$ & 2 & 0.0069 & 9 & 0.0001 & 0.0000 & 71 \\ 
  $S_{36}$ & 2 & 0.0069 & 8 & 0.0001 & 0.0000 & 72 \\ 
 \rowcolor{lightgray} $S_{26}$ & 2 & 0.0069 & 7 & 0.0001 & 0.0000 & 73 \\ 
  $S_{25}$ & 2 & 0.0069 & 3 & 0.0000 & 0.0000 & 74 \\ 
 \rowcolor{lightgray} $S_{43}$ & 2 & 0.0069 & 3 & 0.0000 & 0.0000 & 74 \\ 
  $S_{61}$ & 1 & 0.0000 & 1 & 0.0000 & 0.0000 & 76 \\ 
 \rowcolor{lightgray} $S_{86}$ & 1 & 0.0000 & 1 & 0.0000 & 0.0000 & 76 \\ 
  $S_{4}$ & 1 & 0.0000 & 14 & 0.0002 & 0.0000 & 76 \\ 
 \rowcolor{lightgray} $S_{18}$ & 1 & 0.0000 & 8 & 0.0001 & 0.0000 & 76 \\ 
  $S_{20}$ & 1 & 0.0000 & 10 & 0.0001 & 0.0000 & 76 \\ 
 \rowcolor{lightgray} $S_{94}$ & 1 & 0.0000 & 2 & 0.0000 & 0.0000 & 76 \\ 
  $S_{21}$ & 1 & 0.0000 & 4 & 0.0000 & 0.0000 & 76 \\ 
 \rowcolor{lightgray} $S_{27}$ & 1 & 0.0000 & 1 & 0.0000 & 0.0000 & 76 \\ 
  $S_{88}$ & 1 & 0.0000 & 8 & 0.0001 & 0.0000 & 76 \\ 
 \rowcolor{lightgray} $S_{34}$ & 1 & 0.0000 & 1 & 0.0000 & 0.0000 & 76 \\ 
  $S_{35}$ & 1 & 0.0000 & 61 & 0.0007 & 0.0000 & 76 \\ 
 \rowcolor{lightgray} $S_{57}$ & 1 & 0.0000 & 2 & 0.0000 & 0.0000 & 76 \\ 
  $S_{37}$ & 1 & 0.0000 & 3 & 0.0000 & 0.0000 & 76 \\ 
 \rowcolor{lightgray} $S_{40}$ & 1 & 0.0000 & 1 & 0.0000 & 0.0000 & 76 \\ 
  $S_{81}$ & 1 & 0.0000 & 48 & 0.0006 & 0.0000 & 76 \\ 
 \rowcolor{lightgray} $S_{41}$ & 1 & 0.0000 & 2 & 0.0000 & 0.0000 & 76 \\ 
  $S_{48}$ & 1 & 0.0000 & 36 & 0.0004 & 0.0000 & 76 \\ 
 \rowcolor{lightgray} $S_{49}$ & 1 & 0.0000 & 26 & 0.0003 & 0.0000 & 76 \\ 
  $S_{52}$ & 1 & 0.0000 & 61 & 0.0007 & 0.0000 & 76 \\ 
 \rowcolor{lightgray} $S_{55}$ & 1 & 0.0000 & 1 & 0.0000 & 0.0000 & 76 \\ 
  $S_{56}$ & 1 & 0.0000 & 5 & 0.0000 & 0.0000 & 76 \\ 
 \rowcolor{lightgray} $S_{73}$ & 1 & 0.0000 & 3 & 0.0000 & 0.0000 & 76 \\ 
  $S_{99}$ & 1 & 0.0000 & 1 & 0.0000 & 0.0000 & 76 \\
        \bottomrule
    \end{xltabular}
    


