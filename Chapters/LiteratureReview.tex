


\begin{table}[htbp]
	\centering
	\rowcolors{2}{gray!5}{gray!5}
	\begin{tabularx}{\textwidth}{|l|X|}
		\hline
		\rowcolor{pastelgreen!40!black}
		\multicolumn{2}{|c|}{On the Relationship between Software Complexity and Maintenance Costs\cite{Ogheneovo2014}} \\ \hline
		\textbf{Study Summary:} & Defines the different problems with the implementation of software maintenance and its complexities as described in Section 1 (Maintenance Problems). \\ \hline
		\textbf{Method:} & They used three different software operating systems to compare their estimated development costs, maintenance costs, code lines, and code complexity. This is used in the analysis to determine the relationship between the complexity of the software and the maintenance costs. \\ \hline
		\textbf{Results:} & The results showed that the estimated maintenance costs are high enough that the complexity of the software needs management to reduce these costs. \\ \hline
		\textbf{Commentary:} & This study is used in this review of the literature when discussing the importance of software maintenance difficulties in the implementation of a software maintenance model. The study does not provide solutions on how to implement software maintenance effectively. It also explores the strategies and best practices in its review of the literature to help with software maintenance resource costs. \\ \hline
	\end{tabularx}
\end{table}



\begin{tcolorbox}[colback=gray!5!white, colframe=pastelgreen!40!black, title=On the Relationship between Software Complexity and Maintenance Costs\cite{Ogheneovo2014}]
	\begin{minipage}[t]{0.25\textwidth}
		\textbf{Study summary:}
	\end{minipage}
	\hfill
	\begin{minipage}[t]{0.65\textwidth}
		Defines the different problems with the implementation of software maintenance and its complexities
as described in \Cref{sec:ch1_maintenanceProblems}. 
	\end{minipage}

	% Add space here
	\vspace{0.75em} 

	\begin{minipage}[t]{0.25\textwidth}
		\textbf{Method:}
	\end{minipage}
	\hfill
	\begin{minipage}[t]{0.65\textwidth}
		They used three different software operating systems to compare their estimated development costs, maintenance costs, code lines, and code complexity. This is used in the analysis to determine the relationship between the complexity of the software and the maintenance costs.
	\end{minipage}

	% Add space here
	\vspace{0.75em} 

	\begin{minipage}[t]{0.25\textwidth}
		\textbf{Results:}
	\end{minipage}
	\hfill
	\begin{minipage}[t]{0.65\textwidth}
		The results showed that the estimated maintenance costs are high enough that the complexity of the software needs management to reduce these costs.
	\end{minipage}

	% Add space here
	\vspace{0.75em} 

	\begin{minipage}[t]{0.25\textwidth}
		\textbf{Commentary:}
	\end{minipage}
	\hfill
	\begin{minipage}[t]{0.65\textwidth}
		This study is used in this review of the literature when discussing the importance of software
maintenance difficulties in the implementation of a software maintenance model. The study does not
provide solutions on how to implement software maintenance effectively. It also explores the strategies and best
practises in its review of the literature to help with software maintenance resource costs.
	\end{minipage}
	\tcblower
	\textbf{Study keywords:} Software; Software Maintenance; Software Evolution; maintenance costs;
	software; software evolution; software maintenance
\end{tcolorbox}

\begin{tcolorbox}[colback=gray!5!white, colframe=pastelgreen!40!black, title=Metric-based tracking management in software maintenance\cite{Tang2010}]
	\begin{minipage}[t]{0.25\textwidth}
		\textbf{Study summary:}
	\end{minipage}
	\hfill
	\begin{minipage}[t]{0.65\textwidth}
		This study explores what happens when software maintenance needs to be implemented with certain roleplayers involved.
	\end{minipage}

	% Add space here
	\vspace{0.75em} 

	\begin{minipage}[t]{0.25\textwidth}
		\textbf{Method:}
	\end{minipage}
	\hfill
	\begin{minipage}[t]{0.65\textwidth}
		Assigned roles of each individual in the software maintenance process (user, coordinator, decision maker and maintenance operator) and evaluated their impact on the software maintenance process. 
	\end{minipage}

	% Add space here
	\vspace{0.75em} 

	\begin{minipage}[t]{0.25\textwidth}
		\textbf{Results:}
	\end{minipage}
	\hfill
	\begin{minipage}[t]{0.65\textwidth}
		The results showed that there is explicitly a management issue when software maintenance is implemented. 
	\end{minipage}

	% Add space here
	\vspace{0.75em} 

	\begin{minipage}[t]{0.25\textwidth}
		\textbf{Commentary:}
	\end{minipage}
	\hfill
	\begin{minipage}[t]{0.65\textwidth}
		The study recommended implementing a practical software maintenance model to solve management problems and make maintenance processes more efficient.
	\end{minipage}
	\tcblower
	\textbf{Study keywords:} Metric; Software maintenance; Tracking management
\end{tcolorbox}

\begin{tcolorbox}[colback=gray!5!white, colframe=pastelgreen!40!black, title=A cost model for software maintenance \& evolution\cite{Sneed2004}]
	\begin{minipage}[t]{0.25\textwidth}
		\textbf{Study summary:}
	\end{minipage}
	\hfill
	\begin{minipage}[t]{0.65\textwidth}
		This study examines the cost of software maintenance implementation. It also investigates difficulties with the implementation of software maintenance and the evolution of software maintenance models. 
	\end{minipage}

	% Add space here
	\vspace{0.75em} 

	\begin{minipage}[t]{0.25\textwidth}
		\textbf{Method:}
	\end{minipage}
	\hfill
	\begin{minipage}[t]{0.65\textwidth}
		The study attempted to assess the cost involved for a software system by implementing a prediction cost analysis of multiple types of maintenance activities.
	\end{minipage}

	% Add space here
	\vspace{0.75em} 

	\begin{minipage}[t]{0.25\textwidth}
		\textbf{Results:}
	\end{minipage}
	\hfill
	\begin{minipage}[t]{0.65\textwidth}
		The results showed that the estimated maintenance costs depended on multiple factors, such as the size of the development team, the complexity of the code and the quality of the code.
	\end{minipage}

	% Add space here
	\vspace{0.75em} 

	\begin{minipage}[t]{0.25\textwidth}
		\textbf{Commentary:}
	\end{minipage}
	\hfill
	\begin{minipage}[t]{0.65\textwidth}
		This study highlights the need to implement software maintenance efficiently, but does not provide any solutions to improve it. It provides information on all the possible resource costs involved when implementing a software maintenance model.
	\end{minipage}
	\tcblower
	\textbf{Study keywords:} Maintenance Cost Estimation; Software Life Cycle Costing Models;
Software Maintenance and Evolution; Software Product Management
\end{tcolorbox}


\begin{tcolorbox}[colback=gray!5!white, colframe=pastelgreen!40!black, title=Trends in software maintenance tasks distribution among programmers: A study in a micro software company \cite{Stojanov2017}]
	\begin{minipage}[t]{0.25\textwidth}
		\textbf{Study summary:}
	\end{minipage}
	\hfill
	\begin{minipage}[t]{0.65\textwidth}
		This study examines the software maintenance efforts when implementing a software
maintenance model. It also looks at how software developers prioritise and divide these
maintenance tasks.
	\end{minipage}

	% Add space here
	\vspace{0.75em} 

	\begin{minipage}[t]{0.25\textwidth}
		\textbf{Method:}
	\end{minipage}
	\hfill
	\begin{minipage}[t]{0.65\textwidth}
		Maintenance tasks were classified into different types and distributed among developers with multiple test distribution techniques.
	\end{minipage}

	% Add space here
	\vspace{0.75em} 

	\begin{minipage}[t]{0.25\textwidth}
		\textbf{Results:}
	\end{minipage}
	\hfill
	\begin{minipage}[t]{0.65\textwidth}
		The distribution techniques effectively divide the maintenance task between the developers and improve the study scheduling.
	\end{minipage}

	% Add space here
	\vspace{0.75em} 

	\begin{minipage}[t]{0.25\textwidth}
		\textbf{Commentary:}
	\end{minipage}
	\hfill
	\begin{minipage}[t]{0.65\textwidth}
		This study's decision-making results are created from the maintenance tasks obtained by the developers. It specifically only looks at the contribution of these maintenance tasks from
each developer and how efficiently they distribute the work among them.
	\end{minipage}
	\tcblower
	\textbf{Study keywords:} Maintenance engineering, Market research, Companies, Software
maintenance
\end{tcolorbox}

\begin{tcolorbox}[colback=gray!5!white, colframe=pastelgreen!40!black, title=Supporting Software Architecture Maintenance by Providing Task-specific Recommendations \cite{Galster2019}]
	\begin{minipage}[t]{0.25\textwidth}
		\textbf{Study summary:}
	\end{minipage}
	\hfill
	\begin{minipage}[t]{0.65\textwidth}
		This study discusses the different types of maintenance implemented in the industry and the
		need for software maintenance. The study defines the types of maintenance and typical activities associated when implementing it for the Operations and Maintenance of \Cref{tbl:ch1_SDLC}.
	\end{minipage}

	% Add space here
	\vspace{0.75em} 

	\begin{minipage}[t]{0.25\textwidth}
		\textbf{Method:}
	\end{minipage}
	\hfill
	\begin{minipage}[t]{0.65\textwidth}
		Attempt to address developers' lack of information when trying to implement maintenance.
	\end{minipage}

	% Add space here
	\vspace{0.75em} 

	\begin{minipage}[t]{0.25\textwidth}
		\textbf{Results:}
	\end{minipage}
	\hfill
	\begin{minipage}[t]{0.65\textwidth}
		List multiple solutions for software maintenance recommendations.
	\end{minipage}

	% Add space here
	\vspace{0.75em} 

	\begin{minipage}[t]{0.25\textwidth}
		\textbf{Commentary:}
	\end{minipage}
	\hfill
	\begin{minipage}[t]{0.65\textwidth}
		These studies examine the software maintenance process in the industry, not its improvement. It mainly focusses on maintenance tasks and the correct approach to efficiently
implementing maintenance.
	\end{minipage}
	\tcblower
	\textbf{Study keywords:} Software maintenance; natural language processing; software
	architecture; text classification
\end{tcolorbox}

\begin{tcolorbox}[colback=gray!5!white, colframe=pastelgreen!40!black, title=Analyzing Forty Years of Software Maintenance Models\cite{Lenarduzzi2017}]
	\begin{minipage}[t]{0.25\textwidth}
		\textbf{Study summary:}
	\end{minipage}
	\hfill
	\begin{minipage}[t]{0.65\textwidth}
		This study will analyse software maintenance models and their characteristics for the last
four decades.
	\end{minipage}

	% Add space here
	\vspace{0.75em} 

	\begin{minipage}[t]{0.25\textwidth}
		\textbf{Method:}
	\end{minipage}
	\hfill
	\begin{minipage}[t]{0.65\textwidth}
		Studies were obtained and a systematic analysis of 1,044 articles between 1970 and 2015 on software maintenance models was implemented on the data.
	\end{minipage}

	% Add space here
	\vspace{0.75em} 

	\begin{minipage}[t]{0.25\textwidth}
		\textbf{Results:}
	\end{minipage}
	\hfill
	\begin{minipage}[t]{0.65\textwidth}
		The results showed the common aspects of software maintenance in the industry and common problems. These issues were that the research had limited third-party validation, did not always attempt to improve or build on existing software maintenance models, had limited comparison with the literature, and was difficult to replicate due to private data sets and custom tools
	\end{minipage}

	% Add space here
	\vspace{0.75em} 

	\begin{minipage}[t]{0.25\textwidth}
		\textbf{Commentary:}
	\end{minipage}
	\hfill
	\begin{minipage}[t]{0.65\textwidth}
		This study didn't provide any ways to improve software maintenance, but the limitations of
software maintenance in the industry. This study highlighted these limitations by the slow adaptability of software maintenance models, the lack of comparison in the literature due to close
source models and the replicability of the researched studies.
	\end{minipage}
	\tcblower
	\textbf{Study keywords:} Software Maintenance; Systematic review of the literature.
\end{tcolorbox}

\begin{tcolorbox}[colback=gray!5!white, colframe=pastelgreen!40!black, title=A Software Maintenance Methodology: An Approach Applied to Software Aging\cite{Araujo2021}]
	\begin{minipage}[t]{0.25\textwidth}
		\textbf{Study summary:}
	\end{minipage}
	\hfill
	\begin{minipage}[t]{0.65\textwidth}
		The study primarily discusses implementing suitable software maintenance models for software
maintenance activities through the characterisation of the software system.
	\end{minipage}

	% Add space here
	\vspace{0.75em} 

	\begin{minipage}[t]{0.25\textwidth}
		\textbf{Method:}
	\end{minipage}
	\hfill
	\begin{minipage}[t]{0.65\textwidth}
		The study implemented a software rejuvenation method by implementing maintenance on specific systems and evaluating the performance of the system afterward of these older systems.
	\end{minipage}

	% Add space here
	\vspace{0.75em} 

	\begin{minipage}[t]{0.25\textwidth}
		\textbf{Results:}
	\end{minipage}
	\hfill
	\begin{minipage}[t]{0.65\textwidth}
		The results showed that applying the software rejuvenation method on specific systems improved system performance.
	\end{minipage}

	% Add space here
	\vspace{0.75em} 

	\begin{minipage}[t]{0.25\textwidth}
		\textbf{Commentary:}
	\end{minipage}
	\hfill
	\begin{minipage}[t]{0.65\textwidth}
		The study doesn't provide a guideline to efficiently implement maintenance, but on which
software maintenance strategies can be used for older systems without doing full software maintenance.
	\end{minipage}
	\tcblower
	\textbf{Study keywords:} Software ageing and rejuvenation; methodology; software maintenance
\end{tcolorbox}

\begin{tcolorbox}[colback=gray!5!white, colframe=pastelgreen!40!black, title=Tools and Benchmarks for Automated Log Parsing\cite{Zhu2019}]
	\begin{minipage}[t]{0.25\textwidth}
		\textbf{Study summary:}
	\end{minipage}
	\hfill
	\begin{minipage}[t]{0.65\textwidth}
		This study focusses on log parsing and benchmarking different log parsing methods. 
	\end{minipage}

	% Add space here
	\vspace{0.75em} 

	\begin{minipage}[t]{0.25\textwidth}
		\textbf{Method:}
	\end{minipage}
	\hfill
	\begin{minipage}[t]{0.65\textwidth}
		Compare multiple third-party log parsers with each other with different benchmarks.
	\end{minipage}

	% Add space here
	\vspace{0.75em} 

	\begin{minipage}[t]{0.25\textwidth}
		\textbf{Results:}
	\end{minipage}
	\hfill
	\begin{minipage}[t]{0.65\textwidth}
		The results showed these log parses' robustness, efficiency and accuracy with the benchmarking data.
	\end{minipage}

	% Add space here
	\vspace{0.75em} 

	\begin{minipage}[t]{0.25\textwidth}
		\textbf{Commentary:}
	\end{minipage}
	\hfill
	\begin{minipage}[t]{0.65\textwidth}
		Highlight the efficiency that event logging needs to adhere to. The study uses mainly log
parsers for system diagnostics and not specifically for user-based event logging.
	\end{minipage}
	\tcblower
	\textbf{Study keywords:} AIOps; anomaly detection; log analysis; log management; log parsing
\end{tcolorbox}

\begin{tcolorbox}[colback=gray!5!white, colframe=pastelgreen!40!black, title=A Systematic Review of Logging Practice in Software Engineering\cite{Rong2018}]
	\begin{minipage}[t]{0.25\textwidth}
		\textbf{Study summary:}
	\end{minipage}
	\hfill
	\begin{minipage}[t]{0.65\textwidth}
		This study examines how logging practises are implemented in the industry. The study also
examines the number of publications on logging practises in software engineering, further
discussed in \Cref{apx:loggingPractice}. This emphasises the lack of research on logging practises.
	\end{minipage}

	% Add space here
	\vspace{0.75em} 

	\begin{minipage}[t]{0.25\textwidth}
		\textbf{Method:}
	\end{minipage}
	\hfill
	\begin{minipage}[t]{0.65\textwidth}
		The study created multiple research equations to obtain and analyse the research done for logging practise in software engineering.
	\end{minipage}

	% Add space here
	\vspace{0.75em} 

	\begin{minipage}[t]{0.25\textwidth}
		\textbf{Results:}
	\end{minipage}
	\hfill
	\begin{minipage}[t]{0.65\textwidth}
		The results showed that the relevant estimated research was carried out each year and the topic of development of each article was categorised.
	\end{minipage}

	% Add space here
	\vspace{0.75em} 

	\begin{minipage}[t]{0.25\textwidth}
		\textbf{Commentary:}
	\end{minipage}
	\hfill
	\begin{minipage}[t]{0.65\textwidth}
		This study further highlights the need to create comprehensive methods for developers to
implement a logging practise. There is not enough research done on logging practises and
specific use cases are beneficial to the software industry. 
	\end{minipage}
	\tcblower
	\textbf{Study keywords:} Logging Practise; Software Engineering; Systematic Literature Review
\end{tcolorbox}

\begin{tcolorbox}[colback=gray!5!white, colframe=pastelgreen!40!black, title=Learning to Log: Helping Developers Make Informed Logging Decisions \cite{Zhu2015}]
	\begin{minipage}[t]{0.25\textwidth}
		\textbf{Study summary:}
	\end{minipage}
	\hfill
	\begin{minipage}[t]{0.65\textwidth}
		This study summarises the importance of how and what to log in to help software.
		developers in creating efficient logging mechanisms. 
	\end{minipage}

	% Add space here
	\vspace{0.75em} 

	\begin{minipage}[t]{0.25\textwidth}
		\textbf{Method:}
	\end{minipage}
	\hfill
	\begin{minipage}[t]{0.65\textwidth}
		An automated \textit{LogAdvisor} tool for developers trained in different learning models.
	\end{minipage}

	% Add space here
	\vspace{0.75em} 

	\begin{minipage}[t]{0.25\textwidth}
		\textbf{Results:}
	\end{minipage}
	\hfill
	\begin{minipage}[t]{0.65\textwidth}
		The results showed the benchmark results of the tool performance with different learning models.
	\end{minipage}

	% Add space here
	\vspace{0.75em} 

	\begin{minipage}[t]{0.25\textwidth}
		\textbf{Commentary:}
	\end{minipage}
	\hfill
	\begin{minipage}[t]{0.65\textwidth}
		This study provides a generic method for the log analysis of events based on the intended log analysis.
		The study provides important guidelines discussed in the literature review for event logging
and log analysis, which should be part of the method of a new logging mechanism.
	\end{minipage}
	\tcblower
	\textbf{Study keywords:} keywords
\end{tcolorbox}

\begin{tcolorbox}[colback=gray!5!white, colframe=pastelgreen!40!black, title=Towards a better assessment of event logs quality\cite{Kherbouche2017}]
	\begin{minipage}[t]{0.25\textwidth}
		\textbf{Study summary:}
	\end{minipage}
	\hfill
	\begin{minipage}[t]{0.65\textwidth}
		This study examines what development practises can be implemented to improve or maintain
event log quality.
	\end{minipage}

	% Add space here
	\vspace{0.75em} 

	\begin{minipage}[t]{0.25\textwidth}
		\textbf{Method:}
	\end{minipage}
	\hfill
	\begin{minipage}[t]{0.65\textwidth}
		The study created metrics for the quality assessment of the event log in real-life and artificial case studies.
	\end{minipage}

	% Add space here
	\vspace{0.75em} 

	\begin{minipage}[t]{0.25\textwidth}
		\textbf{Results:}
	\end{minipage}
	\hfill
	\begin{minipage}[t]{0.65\textwidth}
		The results have been validated with natural and synthetic data that validated the quality metrics of the event log.	
	\end{minipage}

	% Add space here
	\vspace{0.75em} 

	\begin{minipage}[t]{0.25\textwidth}
		\textbf{Commentary:}
	\end{minipage}
	\hfill
	\begin{minipage}[t]{0.65\textwidth}
		The event logging quality assessment highlights the importance of maintaining an acceptable
log quality to make the logs consistent and complete. It also ensures that the logging mechanism can accurately capture the logs with the least structural and behavioural complexity. 
	\end{minipage}
	\tcblower
	\textbf{Study keywords:} event logs; process mining; process mining algorithms; qualitative
model
\end{tcolorbox}

\begin{tcolorbox}[colback=gray!5!white, colframe=pastelgreen!40!black, title=Central Audit Logging Mechanism in Personal Data Web Services\cite{Hasiloglu2018}]
	\begin{minipage}[t]{0.25\textwidth}
		\textbf{Study summary:}
	\end{minipage}
	\hfill
	\begin{minipage}[t]{0.65\textwidth}
		This study aims to create a logging mechanism to get audit logs from a Web application.
	\end{minipage}

	% Add space here
	\vspace{0.75em} 

	\begin{minipage}[t]{0.25\textwidth}
		\textbf{Method:}
	\end{minipage}
	\hfill
	\begin{minipage}[t]{0.65\textwidth}
		This study created a central logging mechanism for audit logs in web services.
	\end{minipage}

	% Add space here
	\vspace{0.75em} 

	\begin{minipage}[t]{0.25\textwidth}
		\textbf{Results:}
	\end{minipage}
	\hfill
	\begin{minipage}[t]{0.65\textwidth}
		The results of this study showed the use of an applied logging model to create the central audit logging mechanism for a Web service.	
	\end{minipage}

	% Add space here
	\vspace{0.75em} 

	\begin{minipage}[t]{0.25\textwidth}
		\textbf{Commentary:}
	\end{minipage}
	\hfill
	\begin{minipage}[t]{0.65\textwidth}
		This study did create a method for audit logs for a web service, but no log analysis was done afterward on the logs. The advantages and disadvantages of the created logging mechanism also explored the differences between client- and server-side logging mechanisms. 
	\end{minipage}
	\tcblower
	\textbf{Study keywords:} API; API Policy; Audit Logging; Personal Data; Web Service
\end{tcolorbox}

\begin{tcolorbox}[colback=gray!5!white, colframe=pastelgreen!40!black, title=User behavioural patterns and reduced user profiles extracted from log files\cite{Slaninova2014}]
	\begin{minipage}[t]{0.25\textwidth}
		\textbf{Study summary:}
	\end{minipage}
	\hfill
	\begin{minipage}[t]{0.65\textwidth}
		This study focusses on the behavioural patterns extracted from log files through log
analysis. Create user profiles to compare them with other individual users or groups of users.
	\end{minipage}

	% Add space here
	\vspace{0.75em} 

	\begin{minipage}[t]{0.25\textwidth}
		\textbf{Method:}
	\end{minipage}
	\hfill
	\begin{minipage}[t]{0.65\textwidth}
		The study analyses the behavioural patterns in a given set of log files generated by users interacting with the system. It attempts to get the similarities between the logs through specific log attributes.
	\end{minipage}

	% Add space here
	\vspace{0.75em} 

	\begin{minipage}[t]{0.25\textwidth}
		\textbf{Results:}
	\end{minipage}
	\hfill
	\begin{minipage}[t]{0.65\textwidth}
		The results show the similarities between users or groups of users when they interact with the system.	
	\end{minipage}

	\vspace{0.75em} 

	\begin{minipage}[t]{0.25\textwidth}
		\textbf{Commentary:}
	\end{minipage}
	\hfill
	\begin{minipage}[t]{0.65\textwidth}
		In this study, no logging mechanism method is provided that can be implemented on any
		system. It only implements a log analysis to create user profiles. It also explores what
events in a system can be seen as user-based events for the log analysis.
	\end{minipage}
	\tcblower
	\textbf{Study keywords:} Analysis of users' behaviour; Behavioural patterns; Complex networks;
User profiles
\end{tcolorbox}

\begin{tcolorbox}[colback=gray!5!white, colframe=pastelgreen!40!black, title=Tracking User Activities and Marketplace Dynamics in Classified Ads\cite{Waqar2017}]
	\begin{minipage}[t]{0.25\textwidth}
		\textbf{Study summary:}
	\end{minipage}
	\hfill
	\begin{minipage}[t]{0.65\textwidth}
		This study tracks users' activities on a Web application when users interact with ads. A log
		analysis was implemented to create a probabilistic model of user behaviour data based on the
ads the user has interacted with.
	\end{minipage}

	% Add space here
	\vspace{0.75em} 

	\begin{minipage}[t]{0.25\textwidth}
		\textbf{Method:}
	\end{minipage}
	\hfill
	\begin{minipage}[t]{0.65\textwidth}
		They created models to track user activities when interacting with Web page ads.
	\end{minipage}

	% Add space here
	\vspace{0.75em} 

	\begin{minipage}[t]{0.25\textwidth}
		\textbf{Results:}
	\end{minipage}
	\hfill
	\begin{minipage}[t]{0.65\textwidth}
		The results showed the performance and evaluation of each model used.
	\end{minipage}

	% Add space here
	\vspace{0.75em} 

	\begin{minipage}[t]{0.25\textwidth}
		\textbf{Commentary:}
	\end{minipage}
	\hfill
	\begin{minipage}[t]{0.65\textwidth}
		In this study, only user interactions with the ads are recorded for the log analysis. This
study shows a clear goal for the logging mechanism to capture certain user-based data for
user marketing profiling. In industry, this would be the main goal of organisations when
they attempt to track user-based data for commercial reasons.
	\end{minipage}
	\tcblower
	\textbf{Study keywords:} Classified ads; Temporal analysis; User modelling; User tracking
\end{tcolorbox}

\begin{tcolorbox}[colback=gray!5!white, colframe=pastelgreen!40!black, title=Analysis of visitor's behavior from Web Log using WebLog Expert Tool\cite{Kumar2017}]
	\begin{minipage}[t]{0.25\textwidth}
		\textbf{Study summary:}
	\end{minipage}
	\hfill
	\begin{minipage}[t]{0.65\textwidth}
		This study explains the log analysis for websites using a Web log tool that performs data processing, pattern discovery, and research. This log analysis aims to obtain the user's
activity data of the total time spent per webpage, which webpage they visited, and other visitor browser analysis data.
	\end{minipage}

	% Add space here
	\vspace{0.75em} 

	\begin{minipage}[t]{0.25\textwidth}
		\textbf{Method:}
	\end{minipage}
	\hfill
	\begin{minipage}[t]{0.65\textwidth}
		Use a Web Log Tool to capture certain event logs for a Web Log analysis of the user's behaviour.
	\end{minipage}

	% Add space here
	\vspace{0.75em} 

	\begin{minipage}[t]{0.25\textwidth}
		\textbf{Results:}
	\end{minipage}
	\hfill
	\begin{minipage}[t]{0.65\textwidth}
		The log data used in the log analysis could be used to improve the usability of the website.
	\end{minipage}

	% Add space here
	\vspace{0.75em} 

	\begin{minipage}[t]{0.25\textwidth}
		\textbf{Commentary:}
	\end{minipage}
	\hfill
	\begin{minipage}[t]{0.65\textwidth}
		This study doesn't provide a method for the logging mechanism, but rather the log analysis of
the user's data.  
	\end{minipage}
	\tcblower
	\textbf{Study keywords:} web server log; Web usage mining
\end{tcolorbox}