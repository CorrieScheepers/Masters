% Define a custom command for the table for word conversion
\begin{comment}
\newcommand{\customtable}[6]{%
    \begin{table}[!htb]
        \centering
		\rowcolors{2}{gray!5}{gray!5}
		\begin{tabularx}{\textwidth}{|l|X|}
			\hline
			\rowcolor{pastelgreen!40!black}
			\multicolumn{2}{|p{\dimexpr\linewidth-2\tabcolsep}|}{#1} \\ \hline
			\textbf{Study Summary:} & #2 \\ \hline
			\textbf{Method:} & #3 \\ \hline
			\textbf{Results:} & #4 \\ \hline
			\textbf{Commentary:} & #5 \\ \hline
			\multicolumn{2}{|p{\dimexpr\linewidth-2\tabcolsep}|}{\textbf{Study keywords:} #6} \\ \hline
		\end{tabularx}
    \end{table}

	\pagebreak
}
\end{comment}

%\begin{comment}
\newcommand{\customtable}[6]{%
	\begin{tcolorbox}[colback=gray!5!white, colframe=pastelgreen!40!black, title=#1]
		\begin{minipage}[t]{0.25\textwidth}
			\textbf{Study summary:}
		\end{minipage}
		\hfill
		\begin{minipage}[t]{0.65\textwidth}
			#2
		\end{minipage}

		% Add space here
		\vspace{0.75em} 

		\begin{minipage}[t]{0.25\textwidth}
			\textbf{Method:}
		\end{minipage}
		\hfill
		\begin{minipage}[t]{0.65\textwidth}
			#3
		\end{minipage}

		% Add space here
		\vspace{0.75em} 

		\begin{minipage}[t]{0.25\textwidth}
			\textbf{Results:}
		\end{minipage}
		\hfill
		\begin{minipage}[t]{0.65\textwidth}
			#4
		\end{minipage}

		% Add space here
		\vspace{0.75em} 

		\begin{minipage}[t]{0.25\textwidth}
			\textbf{Commentary:}
		\end{minipage}
		\hfill
		\begin{minipage}[t]{0.65\textwidth}
			#5
		\end{minipage}
		\tcblower
		\textbf{Study keywords:} #6
	\end{tcolorbox}
}
%\end{comment}

% Use the customtable command with different content
\customtable{On the Relationship between Software Complexity and Maintenance Costs\cite{Ogheneovo2014}}{Defines the different problems with the implementation of software maintenance and its complexities as described in Section 1 (Maintenance Problems).}{Three different software operating systems were utilised to compare estimated development costs, maintenance costs, code lines, and code complexity. This is used in the analysis to determine the relationship between the complexity of the software and the maintenance costs.}{The results showed that the estimated maintenance costs are such that the complexity of the software needs management to reduce these costs.}{The study is used in this review of the literature when discussing the importance of software maintenance difficulties in the implementation of a software maintenance model. The study does not provide solutions on how to implement software maintenance effectively. It explores the strategies and best practices in the literature review to help with software maintenance resource costs.}{Software; Software maintenance; Software evolution; Maintenance costs; Software evolution; Software maintenance}

\customtable{
    % Title of the study
    Metric-based tracking management in software maintenance\cite{Tang2010}
}
{
    % Study summary
    Explores the implementation of software maintenance with role-players involved.
}
{
    % Methodology
    Assigned roles of each individual in the software maintenance process (user, coordinator, decision maker, and maintenance operator) and evaluated their impact on the software maintenance process.
}
{
    % Results
    Results indicate explicit management issues when implementing software maintenance.
}
{
    % Commentary
    The study recommends implementing a practical software maintenance model to solve management problems and make maintenance processes more efficient.
}
{
    % Study keywords
    Metric; Software maintenance; Tracking management
}


\customtable{
    % Title of the study
    A cost model for software maintenance \& evolution\cite{Sneed2004}
}
{
    % Study summary
    Examines the cost of software maintenance implementation and investigates the difficulties with software maintenance and the evolution of maintenance models.
}
{
    % Methodology
    Attempts to assess the cost involved for a software system by implementing a prediction cost analysis of multiple types of maintenance activities.
}
{
    % Results
    Shows that estimated maintenance costs depend on factors like the size of the development team, code complexity, and code quality.
}
{
    % Commentary
    Highlights the need for efficient software maintenance but doesn't provide solutions. Offers insights into possible resource costs when implementing a software maintenance model.
}
{
    % Study keywords
    Maintenance cost estimation; Software life cycle costing models; Software Maintenance and evolution; Software product management
}

\customtable{
    % Title of the study
    Trends in software maintenance tasks distribution among programmers: A study in a micro software company \cite{Stojanov2017}
}
{
    % Study summary
    Examines software maintenance efforts when implementing a software maintenance model. Investigates how software developers prioritise and distribute maintenance tasks.
}
{
    % Methodology
    Classifies maintenance tasks into different types and distributes them among developers using multiple test distribution techniques.
}
{
    % Results
    Effective distribution techniques divide maintenance tasks between developers and improve study scheduling.
}
{
    % Commentary
    Focuses on the contribution of maintenance tasks by developers and how efficiently they distribute the work among themselves. Primarily looks at decision making based on maintenance tasks obtained by developers.
}
{
    % Study keywords
    Maintenance engineering, Market research, Companies, Software maintenance
}

\customtable{
    % Title of the study
    Supporting Software Architecture Maintenance by Providing Task-specific Recommendations \cite{Galster2019}
}
{
    % Study summary
    Discusses various types of maintenance in the industry and the need for software maintenance. Defines maintenance types and associated activities for operations and maintenance.
}
{
    % Methodology
    Aims to address developers' lack of information when implementing maintenance.
}
{
    % Results
    Lists multiple solutions for software maintenance recommendations.
}
{
    % Commentary
    Examines the software maintenance process in the industry, with a focus on maintenance tasks and the correct approach to efficient implementation rather than improvement.
}
{
    % Study keywords
    Software maintenance; Natural language processing; Software architecture; Text classification
}

\customtable{
    % Title of the study
    Analysing Forty Years of Software Maintenance Models \cite{Lenarduzzi2017}
}
{
    % Study summary
    Analyses software maintenance models and their characteristics over the last four decades.
}
{
    % Methodology
    Obtained studies and conducted a systematic analysis of 1,044 articles from 1970 to 2015 on software maintenance models.
}
{
    % Results
    Identifies common aspects and problems in the software maintenance industry, including limited third-party validation, lack of improvement on existing models, limited literature comparison, and challenges in replicability due to private data sets and custom tools.
}
{
    % Commentary
    Focuses on highlighting the limitations of software maintenance in the industry rather than providing strategies to improve it. Discusses issues such as slow adaptability of maintenance models, lack of literature comparison due to closed-source models, and challenges in replicating research studies.
}
{
    % Study keywords
    Software maintenance; Systematic review of the literature.
}

\customtable{
    % Title of the study
    A Software Maintenance Methodology: An Approach Applied to Software Aging \cite{Araujo2021}
}
{
    % Study summary
    Discusses the implementation of suitable software maintenance models for software maintenance activities by characterising the software system.
}
{
    % Methodology
    Implements a software rejuvenation method by performing maintenance on specific systems and evaluates the performance improvement of these older systems.
}
{
    % Results
    Shows that applying the software rejuvenation method to specific systems improves system performance.
}
{
    % Commentary
    Focuses on software maintenance strategies for older systems without providing a guideline for efficient maintenance implementation. Addresses the use of maintenance methods for such systems.
}
{
    % Study keywords
    Software ageing and rejuvenation; Methodology; Software maintenance
}

\customtable{
    % Title of the study
    Tools and Benchmarks for Automated Log Parsing \cite{Zhu2019}
}
{
    % Study summary
    Focuses on log parsing and benchmarking different log parsing methods.
}
{
    % Methodology
    Compares multiple third-party log parsers using various benchmarks.
}
{
    % Results
    Demonstrates the robustness, efficiency, and accuracy of these log parsers through benchmarking data.
}
{
    % Commentary
    Emphasizes the importance of efficiency in event logging. Primarily addresses log parsers for system diagnostics rather than user-based event logging.
}
{
    % Study keywords
    AIOps; anomaly detection; Log analysis; Log management; Log parsing
}

\customtable{
    % Title of the study
    A Systematic Review of Logging Practice in Software Engineering \cite{Rong2018}
}
{
    % Study summary
    Examines how logging practices are implemented in the industry and highlights the lack of research in this area in software engineering.
}
{
    % Methodology
    Created multiple research equations to obtain and analyse research on logging practices in software engineering.
}
{
    % Results
    Shows that relevant research has been conducted each year, categorising the development topics of each article.
}
{
    % Commentary
    Emphasizes the need to establish comprehensive methods for developers to implement logging practices. Highlights the lack of sufficient research on logging practices and the importance of specific use cases in the software industry.
}
{
    % Study keywords
    Logging practice; Software engineering; Systematic literature review
}

\customtable{
    % Title of the study
    Learning to Log: Helping Developers Make Informed Logging Decisions \cite{Zhu2015}
}
{
    % Study summary
    Summarises the importance of how and what to log to assist software developers in creating efficient logging mechanisms.
}
{
    % Methodology
    Developed an automated tool called “LogAdvisor" for developers, trained using different learning models.
}
{
    % Results
    Presents benchmark results of the tool's performance with various learning models.
}
{
    % Commentary
    Provides a generic method for log analysis of events based on intended log analysis. Offers important guidelines for event logging and log analysis, which should be incorporated into the method of a new logging mechanism.
}
{
    % Study keywords
    Keywords (not specified in the provided text)
}

\customtable{
    % Title of the study
    Towards a Better Assessment of Event Logs Quality \cite{Kherbouche2017}
}
{
    % Study summary
    Examines development practices that can be implemented to improve or maintain the quality of event logs.
}
{
    % Methodology
    Created metrics for the quality assessment of event logs in both real-life and artificial case studies.
}
{
    % Results
    Validated the quality metrics of event logs using natural and synthetic data.
}
{
    % Commentary
    Highlights the importance of maintaining acceptable log quality to ensure consistency and completeness of logs. Emphasizes the need for logging mechanisms to accurately capture logs with minimal structural and behavioral complexity.
}
{
    % Study keywords
    Event logs; Process mining; Process mining algorithms; Qualitative model
}

\customtable{
    % Title of the study
    Central Audit Logging Mechanism in Personal Data Web Services \cite{Hasiloglu2018}
}
{
    % Study summary
    Aims to create a logging mechanism to obtain audit logs from a web application.
}
{
    % Methodology
    Created a central logging mechanism for audit logs in web services.
}
{
    % Results
    Demonstrated the use of an applied logging model to establish a central audit logging mechanism for a web service.
}
{
    % Commentary
    Created a method for audit logs in a web service but did not perform log analysis on the logs. Explored the advantages and disadvantages of the created logging mechanism and discussed the differences between client- and server-side logging mechanisms.
}
{
    % Study keywords
    API; API policy; Audit logging; Personal data; Web service
}

\customtable{
    % Title of the study
    User Behavioral Patterns and Reduced User Profiles Extracted from Log Files \cite{Slaninova2014}
}
{
    % Study summary
    Focuses on extracting behavioral patterns from log files through log analysis and creating user profiles for comparison with individual users or user groups.
}
{
    % Methodology
    Analyses behavioral patterns in a set of log files generated by users interacting with a system. Aims to identify similarities between logs based on specific log attributes.
}
{
    % Results
    Demonstrates similarities between users or user groups when they interact with the system.
}
{
    % Commentary
    This study does not provide a specific logging mechanism method for implementation on any system. Instead, it conducts log analysis to create user profiles and explores what events in a system can be considered user-based events for log analysis.
}
{
    % Study keywords
    Analysis of users' behavior; Behavioral patterns; Complex networks; User profiles
}

\customtable{
    % Title of the study
    Tracking User Activities and Marketplace Dynamics in Classified Advertisements \cite{Waqar2017}
}
{
    % Study summary
    Focuses on tracking user activities on a web application when users interact with advertisement. Implements log analysis to create a probabilistic model of user behaviour data based on advertisement interactions.
}
{
    % Methodology
    Creates models to track user activities when interacting with web page advertisements.
}
{
    % Results
    Demonstrates the performance and evaluation of each model used.
}
{
    % Commentary
    This study specifically records user interactions with advertisements for log analysis, highlighting the goal of capturing user-based data for marketing profiling. This reflects a common objective in the industry where organisations track user data for commercial purposes.
}
{
    % Study keywords
    Classified ads; Temporal analysis; User modeling; User tracking
}

\customtable{
    % Title of the study
    Analysis of Visitor's Behavior from Web Log using WebLog Expert Tool \cite{Kumar2017}
}
{
    % Study summary
    Explains the log analysis for websites using a Web log tool that performs data processing, pattern discovery, and research. Aims to obtain user activity data, including total time spent per webpage, visited webpages, and other visitor browser analysis data.
}
{
    % Methodology
    Utilizes a Web Log Tool to capture specific event logs for Web Log analysis of user behavior.
}
{
    % Results
    Indicates that the log data used in the analysis could be leveraged to enhance the website's usability.
}
{
    % Commentary
    Focuses on log analysis of user data rather than providing a method for the logging mechanism itself.
}
{
    % Study keywords
    web server log; Web usage mining
}