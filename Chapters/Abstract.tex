
\cleardoublepage
\fancyhead[R]{Abstract}
\chapter*{Abstract}
\addcontentsline{toc}{chapter}{Abstract}

\begin{tabular}{l p{12cm}}
    \textbf{Title:} & \ThesisTitle\\
    \textbf{Author:} & \AuthorTitle\ \Author\\
    \textbf{Supervisor:} & \Supervisor\\
    \textbf{Degree:} & \DegreeName\\
    \textbf{Keywords:} & Software maintenance, logging mechanism, user activities, system utilisation, Web-based\\
\end{tabular}
%\vspace{24pt}

Maintenance of software is continuous and is a reduced form of software development. Research suggests that allocating $15\%$ of the total development cost is to implement a maintenance model on a specific software system. Unused software components or software not meeting the user's requirements will increase over the project's life cycle. Deprecating, some of these systems may reduce the number of resources needed to maintain the entire project. Deciding how many resources for maintenance needs to be allocated to each system can be difficult without a suitable method.\par Software logging is an essential mechanism to improve software maintenance in the form of troubleshooting. In large software systems, logging enables the development team to monitor specific events. System utilisation of each software system can be identified if a suitable logging mechanism is implemented. In Web-based applications, each software system's utilisation is based on how much the users will interact with it. The interactions between the user the software system can be logged. \par Analysing these logs can be challenging when the logging mechanism does not track the desired user-based events. Developing a method to track these events for a specific purpose is more efficient. Relevant data creates a more effective analysis of a specific topic and increases the optimisation of a model. The need to integrate the method used to create a logging mechanism and analyse the data will improve software systems' maintenance.\par During this study, a method is developed for a Web-based application to track the user's activities. The logging mechanism is designed to get any user activities on any software systems they are interacting with. Ajax-requests contains significant and relevant data than tracking all the actions of the user of a specific event. Capturing each Ajax-requests generated from the user activities is the main communication interface between the user's client device and the server that runs the software systems.\par The analysis is performed on the logs that were obtained from the user activity logging mechanism. Software systems utilisation is compared to each other. The resources allocated to maintain each software system is compared to each other. Using the resource allocated and the system utilisation, new decisions can be made for the software maintenance. This formed part of the data-driven decision making about the utilisation of the different software components.