
\cleardoublepage
\fancyhead[R]{Abstract}
\chapter*{Abstract}
\addcontentsline{toc}{chapter}{Abstract}

\begin{tabular}{l p{12cm}}
    \textbf{Title:} & \ThesisTitle\\
    \textbf{Author:} & \AuthorTitle\ \Author\\
    \textbf{Supervisor:} & \Supervisor\\
    \textbf{Degree:} & \DegreeName\\
    \textbf{Keywords:} & Software maintenance, event logging, logging mechanisms, user activities,data-driven decision making, system utilisation analysis, Web-based\\
\end{tabular}
\vspace{24pt}

Maintenance of software is continuous and is a reduced form of software development. These systems are complex and may be inefficient as the need for new development increases. Utilisation of the software systems may become difficult to track as the development continues.

About $15\%$ of the total development costs is allocated to create a maintenance model for a specific software system. In some cases maintenance may be neglected due various reasons. Unused software components or software not meeting the user's requirements will increase over the life cycle of the project.

Tracking the users activities can be a solution of how the system is used to improve software maintenance. Currently logging is used to track any events that occurs any given software system. Analysing these logs provides the nesseacary data for developers of the software system. This mainly use for system troubleshooting. The need to create a suitable logging method for the purpose of analysing user activities will improve software maintenance.

During this study a method was developed for Web-based application to track the user's activities. The method is split-up into creating a suitable logging method to track the user's activities and analysis of the logs. The functionality of these methods address the need of the study to create complete integrated method.