
\cleardoublepage
\fancyhead[R]{Abstract}
\chapter*{Abstract}
\addcontentsline{toc}{chapter}{Abstract}

\begin{tabular}{l p{12cm}}
    \textbf{Title:} & \ThesisTitle\\
    \textbf{Author:} & \AuthorTitle\ \Author\\
    \textbf{Supervisor:} & \Supervisor\\
    \textbf{Degree:} & \DegreeName\\
    \textbf{Keywords:} & Software maintenance, logging mechanism, user activities, system utilisation, Web-based\\
\end{tabular}
%\vspace{24pt} 

Software maintenance is continuous and a form of software development. Research suggests that $15\%$ of the total development cost should be allocated to implement a maintenance model for a software system. Unused components or non-compliant software will increase over the project's life cycle. Discontinuing some systems may reduce the resources required to maintain the entire project. Deciding how many resources to allocate for maintenance to each system can be challenging without a suitable method.\par Software logging is crucial to improve software maintenance via troubleshooting. In extensive software systems, logging enables the development team to monitor specific events. A suitable logging mechanism can identify system utilisation for each software system. In web-based applications, user interactions with each software system determine utilisation. The interactions between the user and the software system can be logged.\par Analysing logs can be challenging if the logging mechanism doesn't track the desired user-based events. Developing a method to track these events for a specific purpose is more efficient. Relevant data produces effective analysis and enhances the optimisation of a model. Thus, integrating the method to create a logging mechanism and analysing data improves software system maintenance.\par This study developed a method to track user activities on a web-based application. The logging mechanism captures any user activities on software systems they interact with. Ajax-requests have more significant and relevant data than tracking each user action of a specific event. Capturing each Ajax-request generated from the user's activities is the primary communication interface between the user's client device and the server that runs the software systems.\par The analysis of the logs obtained from the user activity logging mechanism compares software system utilisation and allocated maintenance resources. These comparisons enable making new decisions for software maintenance. This forms part of data-driven decision making regarding the utilisation of various software components.