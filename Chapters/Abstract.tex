
\cleardoublepage
\fancyhead[R]{Abstract}
\chapter*{Abstract}
\addcontentsline{toc}{chapter}{Abstract}

\begin{table}[!htb]
    \small
    \begin{tabular}{l p{12cm}}
        \textbf{Title:} & \ThesisTitle\\
        \textbf{Author:} & \AuthorTitle\ \Author~\orcidlink{0000-0001-6089-7110} \\
        \textbf{Supervisor:} & \Supervisor\\
        \textbf{Degree:} & \DegreeName\\
        \textbf{Keywords:} & software maintenance, logging mechanism, user activities, system utilisation, maintenance prioritisation, web-based
    \end{tabular}
\end{table}
%\vspace{24pt} 

Software maintenance is continuous and forms part of the software development process. Research suggests that $15\%$ of the total software development cost should be allocated to implementing a maintenance model for a software system. Unused components or non-compliant software will increase over the project's life cycle. The discontinuation of some systems may reduce the resources required to maintain the system. Deciding how many resources to allocate for maintenance for each subsystem can be challenging without a suitable software maintenance prioritisation method. \par Software logging is crucial to improve software maintenance by troubleshooting. In extensive software systems, logging enables the development team to monitor specific events. A suitable logging mechanism can identify the most used software system. In web-based applications, the user interactions with each software system can be determined by capturing user-based events. \par Analysing logs can be challenging if the logging mechanism does not track the desired user-based events. Developing a method to track these events for a specific purpose is more efficient and reliable. The author suggests integrating the method to create a logging mechanism and a log analysis to prioritise software maintenance. \par The method for this study was divided into two main functional parts, namely the logging mechanism and the log analysis. The characteristics of these user-based events, user types, and user-based attributes are defined. The logging mechanism captures any user activity on the software systems, and the logging points in strategic locations in the software systems capture the log attributes. HTTP requests have more significant and relevant data about a specific event. Additional data from request parameters are obtained as metadata that can be used for system diagnostic purposes. \par For the log analysis for this study, log quality is monitored to ensure that event logs are consistent, reliable, and complete to create prioritisation recommendations for software maintenance. A test system validates the method's work, making modifications where needed. The results of this method applied to case studies proved that the method can prioritise software maintenance. The results are evaluated and discussed, and positive and negative points are highlighted by implementing this method. Therefore, the study objectives have addressed the need to develop a method for log analysis for software maintenance by creating a suitable logging mechanism to capture user-based event logs.
