\chapter{Conclusion}
\label{chap:4}
\ChapterPageStuff{4}

\section{Discussion}

\subsection{Contributions made to literature}
\Cref{chap:1} stated that the two main objectives of this study are to create a logging mechanism and implement a user-based utilisation analysis for software maintenance prioritisation. From the obtained literature, implementing software maintenance is an activity that cannot be overlooked in the entire life cycle of a given software system's lifetime. \par Efficiently implementing software maintenance is an obstacle in the industry. This study aims to provide a comprehensive method to more efficiently prioritise software maintenance activities based on the user's utilisation of the software system. This is done using a proven method to get critical data about the software system during its runtime. This data is the user-based events captured and used for log analysis.\par The study introduces new requirements for the logging mechanism to adhere to. This is needed to ensure the log data is correct, trustworthy and complete for the log analysis. The logging mechanism and log analysis are two different systems but must be designed together for prioritising software maintenance.\par The study also provides a software maintenance prioritising method and validates if the software maintenance of different systems can be prioritised. The method adds value by creating this method to expand on the existing literature on software maintenance prioritisation.

\subsection{Value added to industry}
The method in \Cref{chap:2} is used on different case studies in \Cref{sec:ch3_caseStudies}. Each of these case studies explored the various applications of the generic logging mechanism to obtain user-based events. The results proved that the logging mechanism could get the desired user-based activities. However, additional adaptions were needed for each case study to ensure the log quality was acceptable for consistent, reliable log analysis.\par These adaptations were due to the software environment (software languages and design methodologies used) and the purpose of the software system. Older systems had to use different logging points to yield the same result as newer software systems that can use less or one logging point.\par These systems used the same user-based event type with similar operational software use cases. Systems with the same software architecture but different functional use cases also had various adaptations to their logging mechanism. These adaptions ensured that the desired log attributes were correctly obtained for the log analysis. \par The maintenance prioritisation recommendations for each study case used the defined generic methodology. The log analysis for each case study was similar as the log requirements for the user-based utilisation for all the case studies were only different for each of their user-based event types. These user-based event types represent the operational use cases for the case study to further observe the kind of utilisation of the subsystems. \par The results proved that the generic methodology defined in \Cref{chap:2} could be implemented on different software systems with other operational use cases for each case study. In the web development industry, similar software systems as the case study or other software systems will benefit from implementing this methodology to improve software maintenance decisions using these recommendations for prioritisation.

\subsection{Validation strategy}
The following validation strategy is implemented for this study:

\begin{enumerate}[label=\textbf{\Roman*.}]
	\item \textbf{Methodology:} In \Cref{chap:2} development of solution's functional requirements are made in \Cref{sec:ch2_developementOfSolution}. This provides a generic method to create a logging mechanism and analyse the obtained user-based logs. A test logging mechanism is designed with a utilisation log analysis to validate the development of solution functional requirements of \Cref{tbl:ch2_developmenetRequirements}.\par This logging mechanism aims to implement each sub-functional requirement on a test system by testing certain inputs to verify expected outputs. The following validation method was used for this test implementation in \Cref{sec:ch3_implementation}:
		\begin{itemize}
			\item Identifying and creating the log attributes needed for the log analysis (\ref{fr:logAttributes}). The software system's key log attributes are precisely defined for maintenance prioritisation. The user-activity type forms the base requirement for a user-based log.
			
			\item Logging points are made to capture these logging attributes at specific locations in the software system (\ref{fr:loggingPoints}). The placement of the logging points to capture user-based logs is verified if it can be done consistently, accurately and discretely without impacting the software system's performance. 
			
			\item The log analysis is verified using a third-party log analysis tool or implementation of a log analysis (\ref{fr:logAnalysis}). For the log analysis to succeed, the quality must be acceptable. This is verified by the logging points' performance and the usability of the logs without too many post-logging corrections made.
			
			\item Creating software maintenance prioritising (\ref{fr:maintenancePrioritising}) from the results of the log analysis. In the log analysis, the different subsystems' maintenance priority ($M_{PF}$) are calculated from the normalised total active users ($P_N$) multiplied by the normalised total user activity ($A_N$) for a specified subsystem in \Cref{eq:ch2_priorityNormalised,eq:ch2_maintenanceFactorSimplified,eq:ch2_eventNormalised}.\par These results are verified with the test case study and Case Studies A, B and C in \Cref{sec:ch3_caseStudies} on different software systems with different operational use cases. The results obtained for the maintenance priority validate the implementation of the previous above using the defined user-based logs to do the log analysis for maintenance priority. 
		\end{itemize}

	\item \textbf{Study objectives:} The study objectives in \Cref{sec:ch1_objectives} is validated through the method used in \Cref{chap:2} by:
		\begin{itemize}
			\item \textbf{Logging mechanism:}
			  \begin{enumerate}
				\item The log attributes need to be defined for the log analysis. In this study, the log analysis is for maintenance prioritisation. The method created demonstrates the functional requirements to develop the required criteria for which events in the software system are user-based events.
				\item the defined functional requirements for a user-based event log validate item User-based events classification objective. The method to only use the needed log attributes and user activity types describes what events are potentially user-based.
				\item Logging points are the main structure of the logging mechanism. A guideline on strategically placing logging points in a software system is provided in \Cref{sec:ch2_loggingPoints}. This section focuses on how to log in to both client and server side to get all the needed log attributes.
			  \end{enumerate}

			\item \textbf{Log analysis:}
			 \begin{enumerate}
				\item The log analysis evaluates the obtained logs for user-based utilisation results. The evaluation is proven by using third-party or custom-made software to do a comprehensive log analysis to validate if this objective has been met.
				\item The user-based utilisation log analysis implementation needs to create a software maintenance prioritisation. The proposed method to create a consistent measurement to rank software systems based on the needed maintenance validates this objective in the method.
			 \end{enumerate}
		\end{itemize}

	\item \textbf{Need of the study:} There is a need to create a logging mechanism for a log analysis for software maintenance prioritisation. Fulfilling this need can be a valuable method for industry developers to improve the resources used to implement software maintenance more efficiently. The study objectives broke down the needed objectives for this need to be satisfied by splitting it into two parts: logging mechanism and log analysis. 
	\item \textbf{Problem statement:} Software maintenance is a problem in the industry due to how inefficiently developers prioritise maintenance activities. A proven method to monitor software behaviours is logging. The need for study emphasised that the logging mechanism and log analysis to improve software maintenance needs to be explicitly designed for user-based events.
	\item \textbf{Gap identification:} Software maintenance is an important stage in most software development. Implementing it efficiently with the available resources that organisations or individuals have is beneficial. Assisting developers to make better-informed decisions on which systems to prioritise maintenance is a validated gap by identifying the problems that need to be resolved.
\end{enumerate}

\section{Recommendations}
In \Cref{sec:ch3_caseStudies}, the case studies highlighted the limitations of the methodology used. The functional requirements can be expanded on the process to improve the method.

\subsection{Logging quality}
\Cref{sec:ch1_loggingQuality} described the event log quality as an essential part of the logging mechanism. The event logs must be accurate, manage complex structures, and be consistent and complete. For the methodology used, not all of the dimensions of \Cref{fig:ch1_EventQModel} are used to design the logging mechanism.\par This presents Case Study B's logging points functional requirement (\ref{fr:loggingPoints}). The multiple logging points introduced inconsistencies in the event logs for the groups of subsystems used. There are studies on improving event log quality, but creating an event log quality model specifically for user-based event logging can increase the log quality. \par Some of the defined event log quality model requirements defined in \Cref{fig:ch1_EventQModel} has not been fully integrated into the method. Some of these requirements can add value to the logging quality.

\subsection{Maintenance prioritisation}
The maintenance prioritisation (\ref{fr:maintenancePrioritising}) can have improvements made in multiple ways. The \Cref{eq:ch2_maintenanceFactorSimplified,eq:ch2_eventNormalised,eq:ch2_priorityNormalised} makes use of the full user-based event logs per subsystem and total users linked per system as its base variables.\par From the results observed of the case studies in \Cref{sec:ch3_caseStudies} the base variables had an overwhelming impact on the prioritisation factor. Introducing other variables that represent other factors from the log attributes can improve the accuracy of the maintenance prioritisation. One important variable that could have been used is the user activity type of each case study.

\subsubsection{User activity types}
Throughout the study, the user activity type was essential to describe what can be classified as a user event. This primary log attribute can add another dimension to the results, as some user activities may be more important than others. \par For software systems, as in Case Study C, where most user activity types were grouped, can benefit the final results. There aren't many differences between some activity types. 

\subsubsection{User and activity parameters}
\par Normalisation is used in \Cref{eq:ch2_maintenanceFactorSimplified,eq:ch2_eventNormalised,eq:ch2_priorityNormalised} to make a comparable scale for both of the main variables for the software maintenance prioritisation. While Normalisation yielded similar results. There were numerous outliers on each variable's low and high ends. The data doesn't follow a specific distribution or pattern as $S_{582}$ of Case Study A with a normalised activity of 1. This subsystem had the highest normalised activity but the $4^{th}$ highest maintenance priority due to its lower unique active user base.\par Using different techniques to formulate a software maintenance priority factor may have placed $S_{582}$ higher. If the effect of the total unique users were not used as the primary prioritisation factor for this situation. 

\clearpage

\section{Conclusion}

\subsection{In summary}
\begin{itemize}
	\item There is a need to improve software maintenance activities in the industry, but software maintenance prioritisation is still a problem for most software developers.
	\item Event-based logging is a proven method to get valuable information about a software system.
	\item A log analysis of the utilisation of the software system can provide the needed evidence to prioritise software systems based on the extracted log data.
	\item Creating a user-based event logging mechanism to implement a system utilisation log analysis is a solution to the identified problem.
	\item Three different case studies with two different operational use cases verified the designed methodology for this study.
\end{itemize}

\subsection{In conclusion}
Analysing user-based event logs can enhance software maintenance resource management by prioritising maintenance tasks through a comprehensive log analysis.