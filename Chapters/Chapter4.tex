\chapter{Conclusion}
\label{chap:4}
\ChapterPageStuff{4}

\section{Discussion}

\subsection{Contributions made to literature}
A problem with software maintenance practices is identified where software developers struggle to
implement software maintenance. These software maintenance activities were primarily not correctly
prioritised. A method was needed to identify these software systems and create prioritisation
recommendations for software maintenance. Event logging is an existing software tool identified from
literature in \Cref{chap:1} that can identify systems needing software maintenance. \par The study
was split into two main study objectives that represent the logging mechanism and the log analysis
to make recommendations for software maintenance. To create recommendations for software maintenance
using a log analysis, the logging mechanism had to be designed to capture the desired user-based
event logs. \par A generic methodology was created for a logging mechanism that focuses on the
needed logging points, logging attributes and the overall log quality defined from obtained
literature for best log practices in the industry. The log analysis focuses on using user-based
event logs for software maintenance recommendations.\par The addition of the requirements for the
specific log analysis for software maintenance recommendations narrows down the scope of which
industry practices are needed for the logging mechanism. The combination of these two separate study
objectives creates a new generic methodology defined in \Cref{chap:2}.

\subsection{Value added to industry}
The method in \Cref{chap:2} is used on different case studies in \Cref{sec:ch3_caseStudies}. Each of
these case studies explored the different applications of the generic logging mechanism to obtain
user-based events. The results proved that the logging mechanism could get the desired user-based
activities. However, additional adaptions were needed for each case study to ensure the log quality
was acceptable for consistent, reliable log analysis.\par These adaptations were due to the software
environment (software languages and design methodologies used) and the purpose of the software
system. Older systems had to use different logging points to yield the same result as newer software
systems that can use less or one logging point.\par These systems used the same user-based event
type with similar operational software use cases. Systems with the same software architecture but
different operational use cases also had different adaptations to their logging mechanism. These
adaptions ensured that the desired log attributes were correctly obtained for the log analysis. \par
The maintenance prioritisation recommendations for each study case made use of the defined generic
methodology. The log analysis for each case study was similar as the log requirements for the
user-based utilisation for all the case studies were only different for each of their user-based
event types. These user-based event types represent the operational use cases for the case study to
further observe the type of utilisation of the subsystems. \par The results proved that the generic
methodology defined in \Cref{chap:2} could be implemented on different software systems with
different operational use cases for each case study. In the web development industry, similar
software systems as the case study or different software systems will benefit from implementing this
methodology to improve software maintenance decisions using these recommendations for
prioritisation.

\subsection{Validation strategy}
To validate whether the study objectives have been met, the following validation strategy is
implemented:

\begin{enumerate}[label=\textbf{\Roman*.}]	
	\item \textbf{Gap identification:} There is a gap in the knowledge of improving software
	maintenance by prioritising certain software systems that need the most software maintenance
	resources spent on it by implementing a suitable generic methodology to identify these systems
	fully. This is defined in \Cref{chap:1} from the literature where software maintenance is
	studied further for the Operation and Maintenance phase of the SDLC.
	\item \textbf{Problem statement:} In \Cref{sec:ch1_problemStatement}, the problem has been
	identified that software maintenance is an important part of software development. Implementing
	a suitable maintenance methodology can be difficult for developers as the software systems that
	need the most attention can be different from the user's perspective.  
	\item \textbf{Need for the study:} From the identified problem statement, there is a need to
	obtain user-based data to do a system utilisation analysis to provide evidence for software
	maintenance prioritising. This needs, for the study, forms the basis for what is needed to
	resolve the identified problem.
	\item \textbf{Study objectives:} From the need for the study, two primary objectives were
	identified: the logging mechanism objective and the log analysis objective.\par The logging
	mechanism objective is to use existing methodologies to create a logging mechanism to capture
	user-based event logs. This also includes improving the log quality to increase the accuracy and
	trustworthiness of the log analysis.\par The log analysis study objective is to use the obtained
	logs and implement a utilisation analysis on the software components to make software
	maintenance recommendations on which software components should be prioritised.
	\item \textbf{Methodology:} For this study, the methodology was designed for web-based
	applications to narrow down the broad scope of case studies. In \Cref{chap:2}, the methodology
	was implemented on a test system to verify if all the development of solution functional
	requirements in \Cref{tbl:ch2_developmenetRequirements} are met.\par These functional
	requirements are intended to decompose the study objectives into more specific and detailed
	requirements to facilitate the development of a solution. To validate the logging mechanism
	these main functional requirements:
	 \begin{itemize}
		\item Define the needed \textbf{log attributes} that is needed to define what events are
		user-based events using the user-based event types and what needs to be logged to complete a
		log analysis.
		\item Create and place \textbf{logging points} to capture the desired log attributes
		accurately and precisely during the software system's runtime without taxing the software
		system's normal operation too much.
	 \end{itemize}

	 To validate the logging analysis study objective, the following main functional requirements
	 need to be implemented:
	 \begin{itemize}
		\item Using or creating a \textbf{log analysis} tool to extract the log data and transform
		it into usable analysis information about the software systems the users interact with.
		\item Applying a \textbf{software maintenance prioritisation} algorithm to be able to make
		recommendations on what software systems need to receive the most software maintenance
		resources.
	 \end{itemize}

	 After analysing the results of the case studies when applying the methodology and verifying
	 which functional requirements are met for each case study has been addressed as shown in
	 \Cref{tbl:ch3_functionalRequirements}. There were some exceptions where certain functional
	 requirements were not met. This was discussed why this happened for each specific case study.
\end{enumerate}

\begin{enumerate}[label=\textbf{\Roman*.}]
	\item \textbf{Methodology:} In \Cref{chap:2} development of solution functional requirements are made in \Cref{sec:ch2_developementOfSolution} to provide a generic method to create a logging mechanism and do a log analysis of the obtained user-based logs. To validate that the development of solution functional requirements of \Cref{tbl:ch2_developmenetRequirements}, a test logging mechanism is created with a utilisation log analysis.\par This logging meachnism aims to implement each of the functional requirements of the development of solutions requirements on a test system by testing certain inputs to verify expected outputs. The following validation method was used for this test implementation in \Cref{sec:ch3_implementation}:
		\begin{itemize}
			\item Identifying and creating the log attributes nesseacary for the log analysis (\textbf{\ref{fr:logAttributes}}). For maintenance to be prioritise the key log attributes is specifically defined for the software system. The user-activity type forms the base requirement for a user-based log.
			
			\item Logging points are made to capture these logging attributes at specific locations in the software system (\textbf{\ref{fr:loggingPoints}}). The placement of the logging points to capture user-based logs are verified if it is able to consistently, accurately and discretely without impacting software system's performance. 
			
			\item The log analysis is verified through making use of thiird party log analysis tool or an implementation of a log analysis (\textbf{\ref{fr:logAnalysis}}). For the log analysis to be successful the log qaulity needs to be on accetptable standard. This is verified by the logging points performance and the useabilty of the logs without too much post logging corrections made.
			
			\item The method of creating software maintenance Priortising
		\end{itemize}
\end{enumerate}

\section{Recommendations}
In \Cref{sec:ch3_caseStudies}, the case studies highlighted the limitations of the methodology used.
The functional requirements can be expanded on the methodology to improve the method.

\subsection{Logging quality}
In \Cref{sec:ch1_loggingQuality} described the event log quality as an essential part of the logging
mechanism. The event logs must be accurate, manage complexity structure, and be consistent and
complete. For the methodology used, not all of the dimensions of \Cref{fig:ch1_EventQModel} are used
to design the logging mechanism.\par This presents Case Study B's logging points functional
requirement (\ref{fr:loggingPoints}). The multiple logging points introduced inconsistencies in the
event logs for the groups of subsystems used. There are studies on improving event log quality, but
creating an event log quality model specifically for user-based event logging can increase the log
quality. The event log quality model doesn't have to have all the needed properties of
\Cref{fig:ch1_EventQModel}, but be modified for user-based event logging as
\Cref{fig:ch1_EventQModel} is a more generic log quality model for all types of event logging.

\subsection{Maintenance prioritisation}
The maintenance prioritisation (\ref{fr:maintenancePrioritising}) can have improvements made in
multiple ways. The
\Cref{eq:ch2_maintenanceFactorSimplified,eq:ch2_eventNormalised,eq:ch2_priorityNormalised} makes use
of the total user-based event logs per subsystem $A_X$ and total users linked per system $P_X$ as
its base variables.\par From the results observed of the case study in \Cref{sec:ch3_caseStudies}
these base variables had an overwhelming impact on the prioritisation factor. Introducing other
variables that represent other factors from the log attributes can improve the accuracy of the
maintenance factor. One important variable that could have been used is the user activity type of
each case study.\par Throughout the study, the user activity type was essential to describe what
event can be classified as a user event. This primary log attribute can add another dimension to the
results, as some user activities may be more important than others. \par For software systems, as in
Case Study C, where a large majority of the user activity types were grouped, it can be beneficial
for the final results. There aren't many differences between some activity types. Adding weight to
how important each user type is another limitation of this study.\par Normalisation is used in
\Cref{eq:ch2_maintenanceFactorSimplified,eq:ch2_eventNormalised,eq:ch2_priorityNormalised} to make a
comparable scale for both of the main variables for the software maintenance prioritisation. While
Normalisation yielded comparable results. There were numerous outliers on each variable's low and
high ends. The data doesn't follow a specific distribution or pattern as $S_{582}$ of Case Study A
with $A_N=1$. This subsystem had the highest $A_N$ but the $4^{th}$ highest $M_{PF}$ due to its
lower unique active user base.\par Using different techniques to formulate a software maintenance
priority factor may have placed $S_{582}$ higher if the effect of the total unique users wasn't used
as the primary prioritisation factor. There is a need to be able to rank each subsystem with each
other accurately.

\section{Conclusion}

\subsection{In summary}
\begin{itemize}
	\item There is a need to improve software maintenance activities in the industry, but software
	maintenance prioritisation is still a problem for most software developers.
	\item Event-based logging is a proven method to get valuable information about a software
	system.
	\item A log analysis of the utilisation of the software system can provide the needed evidence
	to prioritise software systems based on the extracted log data.
	\item Creating a user-based event logging mechanism to implement a system utilisation log analysis
	is a solution to the identified problem.
	\item Three different case studies with two different operational use cases verified the
	designed methodology for this study.
\end{itemize}

\subsection{In conclusion}
Analysing user-based event logs can enhance software maintenance resource management by prioritising
maintenance tasks through a comprehensive log analysis.