\chapter{Conclusion}
\label{chap:4}
\ChapterPageStuff{4}

\section{Discussion}

\subsection{Contributions made to literature}
\Cref{chap:1} highlighted the importance of software maintenance in the entire life cycle of most software systems. It is an essential task in the Software Development Life-Cycle that will use up to $60\%-80\%$ of the total available software resources.\par In \Cref{sec:ch1_softwareMaintenanceIntro}, it has been identified that software maintenance should be efficiently implemented due to its high resource cost. Prioritisation of software maintenance can improve the software maintenance efforts of developers more efficiently. Event logging has been identified as a proven method for prioritisation of software maintenance.\par An extensive literature study revealed three State of the Art topics for this study: software maintenance, event logging and log analysis. In \Cref{tbl:ch1_stateOfTheArt2}, the State of the Art further highlighted the gap in the existing literature. Especially in event logging, where it is mainly used for system diagnostics.\par The method in \Cref{chap:2} it has been identified that the event logging should be made explicitly for the log analysis. Additional functional requirements were made for a user-based logging mechanism using existing logging practices defined in the literature. This method contributes to the literature by creating these functional requirements that the logging mechanism needs to be able to meet for software maintenance prioritisation. \par The software maintenance literature mainly focused on the optimisation of the actual software maintenance tasks or the implementation of the software maintenance models. The value added to the literature is the method used in \Cref{chap:3} to make maintenance prioritisation recommendations using a log analysis on the obtained logs. A maintenance priority factor was created in this log analysis to make these recommendations.\par The method was verified by implementing it on a test system in \Cref{sec:ch3_implementation,sec:ch3_Verification}. The verification proved that the method could make software maintenance prioritisation by comparing the input with the expected output.\par After verifying the method, it was implemented in three case studies. These case studies consisted of software systems with other operational use cases and software environments to evaluate the results of the method. The functional requirements defined by the technique were mainly satisfied in each case to study to create the software maintenance prioritisation successfully.\par These results are evaluated, and the method's shortcomings were identified for each case study. Each case was also compared to the other after their in-depth analysis.

\clearpage

\subsection{Value added to industry}
The method in \Cref{chap:2} is used on different case studies in \Cref{sec:ch3_caseStudies}. Each of these case studies explored the various applications of the generic logging mechanism to obtain user-based events. The results proved that the logging mechanism could get the desired user-based activities. However, additional adaptions were needed for each case study to ensure the log quality was acceptable for consistent, reliable log analysis.\par These adaptations were due to the software environment (software languages and design methodologies used) and the purpose of the software system. Older systems had to use different logging points to yield the same result as newer software systems that can use less or one logging point.\par These systems used the same user-based event type with similar operational software use cases. Systems with the same software architecture but different functional use cases also had various adaptations to their logging mechanism. These adaptions ensured that the desired log attributes were correctly obtained for the log analysis. \par The maintenance prioritisation recommendations for each study case used the defined generic methodology. The log analysis for each case study was similar as the log requirements for the user-based utilisation for all the case studies were only different for each of their user-based event types. These user-based event types represent the operational use cases for the case study to further observe the kind of utilisation of the subsystems. \par The results proved that the generic methodology defined in \Cref{chap:2} could be implemented on different software systems with other operational use cases for each case study. In the web development industry, similar software systems as the case study or other software systems will benefit from implementing this methodology to improve software maintenance decisions using these recommendations for prioritisation.

\subsection{Validation strategy}
\Cref{tbl:ch4_ValidationStart} is the validation strategy used for this study. Each of the literature and empirical objectives, respectively, in \Cref{sec:ch1_empiricalObjective,sec:ch1_literatureObjective} is validated in the section it was met.

\clearpage

\begin{table}[!htb]
	\centering
	\caption[Study validation]
	{\textit{Study validation}}
	\label{tbl:ch4_ValidationStart}
	\begin{tabularx}{\textwidth}{Xp{3cm}p{3cm}}
		\toprule
		\textbf{Objective}  & \textbf{Section} & \textbf{Objective met} \\ \midrule
		\textbf{Literature Objectives:} & & \\ \midrule
		\rowcolor{lightgray}
		\RaggedRight \objAi & \RaggedRight \Cref{sec:ch2_logAttributesRequirements,sec:ch2_webApplicationArchitecture} & \cmark \\
		\RaggedRight \objAii & \RaggedRight \Cref{sec:ch2_loggingPoints,sec:ch2_webApplicationArchitecture} & \cmark \\
		\rowcolor{lightgray}
		\RaggedRight \objAiii & \Cref{sec:ch2_logAnalysisTools} & \cmark \\ 
		\RaggedRight \objAiv & \Cref{sec:ch2_utilisationImprovements} & \cmark \\ \midrule
		\textbf{Empirical Objectives:} & & \\ \midrule
		\rowcolor{lightgray}
		\RaggedRight \objBi: 
			\begin{enumerate}
				\item \objBiSubA
				\item \objBiSubB
				\item \objBiSubD
				\item \objBiSubC
			\end{enumerate} & \Cref{sec:ch3_implementation} & \cmark \\
		\RaggedRight \objBii & \Cref{sec:ch3_Verification} & \cmark \\
		\rowcolor{lightgray}
		\RaggedRight \objBiii & \Cref{sec:ch3_caseStudies} & \cmark \\
		\bottomrule
	\end{tabularx}
\end{table}


\section{Recommendations}
In \Cref{sec:ch3_caseStudies}, the case studies highlighted the limitations of the methodology used. The functional requirements can be expanded on the process to improve the method.

\subsection{Logging quality}
\Cref{sec:ch1_loggingQuality} described the event log quality as an essential part of the logging mechanism. The event logs must be accurate, manage complex structures, and be consistent and complete. For the methodology used, not all of the dimensions of \Cref{fig:ch1_EventQModel} are used to design the logging mechanism.\par This presents Case Study B's logging points functional requirement (\ref{fr:loggingPoints}). The multiple logging points introduced inconsistencies in the event logs for the groups of subsystems used. There are studies on improving event log quality, but creating an event log quality model specifically for user-based event logging can increase the log quality. \par Some of the defined event log quality model requirements specified in \Cref{fig:ch1_EventQModel} still need to be fully integrated into the method. Some of these requirements can add value to the logging quality.

\subsection{Maintenance prioritisation}
The maintenance prioritisation (\ref{fr:maintenancePrioritising}) can have improvements made in multiple ways. The \Cref{eq:ch2_maintenanceFactorSimplified,eq:ch2_eventNormalised,eq:ch2_priorityNormalised} makes use of the full user-based event logs per subsystem and total users linked per system as its base variables.\par From the results observed of the case studies in \Cref{sec:ch3_caseStudies}, the base variables had an overwhelming impact on the prioritisation factor. Introducing other variables that represent other factors from the log attributes can improve the accuracy of the maintenance prioritisation. One important variable that could have been used is the user activity type of each case study.

\subsubsection{User activity types}
Throughout the study, the user activity type was essential to describe what can be classified as a user event. This primary log attribute can add another dimension to the results, as some user activities may be more important than others. \par For software systems, as in Case Study C, where most user activity types were grouped, can benefit the final results. There are a few differences between some activity types. 

\subsubsection{User and activity parameters}
\par Normalisation is used in \Cref{eq:ch2_maintenanceFactorSimplified,eq:ch2_eventNormalised,eq:ch2_priorityNormalised} to make a comparable scale for both of the main variables for the software maintenance prioritisation. While Normalisation yielded similar results, there were numerous outliers on each variable's low and high ends. The data does not follow a specific distribution or pattern as $S_{582}$ of Case Study A with a normalised activity of 1. This subsystem had the highest normalised activity but the $4^{th}$ highest maintenance priority due to its lower unique active user base.\par Using different techniques to formulate a software maintenance priority factor may have placed $S_{582}$ higher. If the effect of the total unique users were not used as the primary prioritisation factor for this situation. 

\clearpage

\section{Conclusion}

\subsection{In summary}
\begin{itemize}
	\item There is a need to improve software maintenance activities in the industry, but software maintenance prioritisation is still a problem for most software developers.
	\item Event-based logging is a proven method to get valuable information about a software system.
	\item A log analysis of the utilisation of the software system can provide the needed evidence to prioritise software systems based on the extracted log data.
	\item Creating a user-based event logging mechanism to implement a system utilisation log analysis is a solution to the identified problem.
	\item Three different case studies with two different operational use cases verified the designed methodology for this study.
\end{itemize}

\subsection{In conclusion}
Analysing user-based event logs can enhance software maintenance resource management by prioritising maintenance tasks through a comprehensive log analysis.