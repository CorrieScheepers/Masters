\chapter{Introduction}
\label{chap:1}
\ChapterPageStuff{1}

\section{Preamble}
Maintenance of software systems is continuous and is a reduced form of software
development.~\cite{Sneed2004}~Software maintenance needs to be done on regular
systematically procedure in its entire lifespan. Most companies will strife to increase their digital products and services over the life cycle of the software project.~\cite{Niu2018},~\cite{Galster2019}

\section{Background}
Software systems can be complex and inefficient as the need for new development increases on it. Human and software resources are limited or ineffective.~\cite{Pecchia2015}~It is expensive to allocate new resources to software maintenance without a cost model for software
maintenance.~\cite{Galster2019}\par A lot of time needs to be invested in
software maintenance to build an effective model to implement and it makes up to
$15\%$ of the development cost.~\cite{Lenarduzzi2017}~Maintaining software
systems will receive lower development priority as the need for new
functionality for the software project increases.~\cite{Sneed2004}~Unused or
software not meeting the user's requirements will increase over the life cycle
of the software project.~\cite{Thankachan2018}\par It is difficult to track
unused software systems that can be deprecated. Some parts of the software can
be removed or revealed to address the needs of the
user.~\cite{Dalpiaz2018},~\cite{Shahid2016}~Tracking the behavioural patterns of
the user. It may be that the user is uninformed of how to use the system or the
developers did not design the system to meet the user's requirement
specifications.~\cite{Slaninova2014},~\cite{Chen2019}\par Tracking any user
activities can be a strenuous task depending on the method that is used to get the
data. The use of questionnaires is will not be a viable solution as the software
systems may be too large and diverse to get any meaningful feedback from the
user.~\cite{Slaninova2014},~\cite{Waqar2017}~Using logs to track the user's
the activity gives more valid data of how the user is interacting with the software
systems.~\cite{Lei2018}

The event logging is used for maintenance to collect data on the behaviour of
software systems. Event logs are useful for:~\cite{Park2017}
\begin{itemize}
    \item Debugging of software systems
    \item Failure analysis and system recovery
    \item To get key behavioural information of individual software components
    \item System utilisation
    \item Users contribute to initiating events in software systems
\end{itemize}
This can be used to track user activities to make data-driven decisions of how
to maintain the software systems as such as Web-based
systems.~\cite{Rong2018}~\cite{Razavi2008},~\cite{Lei2018}

\newpage

\subsection{Need for a logging mechanism to track user-based activities}
No study was found where both the logging mechanism and analysis were combined.
There is a need to develop a method to implement a user-activity logging
the mechanism to do further analysis of the logs to improve software
maintenance.

\subsection{Objectives of the study}
The goal of the study is to develop a logging mechanism to track user-based
activities to preform analysis of these logs to improve system maintenance in
software environment. The study is divided in two components to achieve the
primary goal which is the design and implementation of the logging mechanism
and the analysis of the system utilisation to improve system maintenance.

\subsubsection{Logging mechanism:}
\begin{itemize}
    \item Random text.
\end{itemize}

\subsubsection{Analysis of the system utilisation to improve software maintenance}
\begin{itemize}
    \item Random text.
\end{itemize}

\subsection{Overview of the dissertation}
\subsubsection{Chapter 1: Introduction}
This chapter contains the background of software maintenance and system
utilisation analysis.
\subsubsection{Chapter 2: Mehtodolgy}

\newpage
\section{Literature Study}

\subsection{Preamble}
Introduction of section.

\subsection{Software Maintenance}


\subsection{Event-based logging}
In this section the event-based logging will be discussed and how it is part of software maintenance. 

\subsection{User-based activities in Web-based applications}
In this section, the user-activities in Web-based applications will be discussed.

\subsection{Logging mechanism development}

\section{Conclusion}

