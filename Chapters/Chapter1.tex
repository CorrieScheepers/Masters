\chapter{Introduction}
\label{chap:1}
\ChapterPageStuff{1}

\section{Preamble}
Maintenance of software is a fundamental \cite{Levin2019}

\section{Background}

Maintenance of software systems is continuous and is a reduced form of software
development.~\cite{Sneed2004}~ Most companies will strife to increase their
digital products and services over the life cycle of the software
project.~\cite{Niu2018},~\cite{Galster2019}\par Software systems are complex and
inefficient as the need for new development increases. Human and software
resources are limited or ineffective.~\cite{Pecchia2015}~It is expensive to
allocate new resources to software maintenance without a cost model for software
maintenance.~\cite{Galster2019}\par A lot of time needs to be invested in
software maintenance to build an effective model to implement and it makes up to
$15\%$ of the development cost.~\cite{Lenarduzzi2017}~Maintaining software
systems will receive lower development priority as the need for new
functionality for the software project increases.~\cite{Sneed2004}~Unused or
software not meeting the user's requirements will increase over the life cycle
of the software project.~\cite{Thankachan2018}\par It is difficult to track
unused software systems that can be deprecated. Some parts of the software can
be removed or revealed to address the needs of the
user.~\cite{Dalpiaz2018},~\cite{Shahid2016}~Tracking the behavioural patterns of
the user. It may be that the user is uninformed of how to use the system or the
developers did not design the system to meet the user's requirement
specifications.~\cite{Slaninova2014},~\cite{Chen2019}\par Tracking any user
activities can be a strenuous task depending on the method that is used to get the
data. The use of questionnaires is will not be a viable solution as the software
systems may be too large and diverse to get any meaningful feedback from the
user.~\cite{Slaninova2014},~\cite{Waqar2017}~Using logs to track the user's
activity gives more valid data of how the user is interacting with the software
systems.~\cite{Lei2018}

The event logging is used for maintenance to collect data on the behaviour of
software systems. Event logs are useful for:~\cite{Park2017}
\begin{itemize}
    \item Debugging of software systems
    \item Failure analysis and system recovery
    \item To get key behavioural information of individual software components
    \item System utilisation
    \item Users contribute to initiating events in software systems
\end{itemize}
This can be used to track user activities to make data-driven decisions of how
to maintain the software systems as such as Web-based
systems.~\cite{Rong2018}~\cite{Razavi2008},~\cite{Lei2018}

\newpage
\subsection{State of the Art}\label{sec:stateofart}

Several studies are obtained for user-based activity tracking where
five key components are identified that is relevant to the topic:

\begin{itemize}
    \item \textbf{Log analysis} is a data mining process focused on the analysis of computer-generated records.~\cite{Vaarandi2015},~\cite{Pecchia2015},~\cite{Song2008}
    \item \textbf{Logging implementation} by using different types of logging mechanisms for event logging.~\cite{Potey2013},~\cite{Rong2018},~\cite{Rong2018a},~\cite{Tian2017}
    \item \textbf{Logging points} generated like database transactions.~\cite{Potey2013},~\cite{Li2018}
    \item \textbf{Pattern discovery} of user behavioural activities in the event logs.~\cite{Dhanalakshmi2016},~\cite{Slaninova2014},~\cite{Lu2019},~\cite{Port2017}
    \item \textbf{System utilisation analysis} based on the user behavioural activities of these event logs.~\cite{Park2017},~\cite{Slaninova2014},~\cite{Lu2019},~\cite{Chen2019}
\end{itemize}

In Table~\ref{tbl:stateoftheart} the above components is used to identify which
studies are relevant to user-based activity tracking and analysis. Red indicates
if the study is less relevant and green indicates if the study is relevant to
the topic.

\begin{table}[!htb]
    \centering
    \small
    \caption{State of the Art}
    \label{tbl:stateoftheart}
    \begin{tabularx}{\textwidth}{|c|X|X|X|X|X|}
        \hline
        Source & Logging Points & Logging Implementation & Logging Analysis &
        \raggedright Pattern Discovery & System Utilisation Analysis \\ \hline
        \cite{Sneed2004} & \red & \green & \red & \green & \red \\ \hline
        \cite{Thankachan2018} & \green & \green & \green & \red & \red \\ \hline
        \cite{Park2017} & \red & \red & \red & \green & \green \\ \hline
        \cite{Rong2018} & \red & \green & \green & \red & \red \\ \hline
        \cite{Vaarandi2015} & \red & \red & \green & \green & \green \\ \hline
        \cite{Potey2013} & \red & \red & \green & \green & \green \\ \hline
        \cite{Rong2018a} & \red & \red & \green & \green & \green \\ \hline
        \cite{Li2018} & \green & \green & \green & \red & \red \\ \hline
        \cite{Lu2019} & \red & \green & \green & \green & \red \\ \hline
        \cite{Cinque2013} & \red & \red & \red & \green & \green \\ \hline
        \cite{Pathan2014} & \red & \red & \red & \green & \green \\ \hline
    \end{tabularx}
\end{table}

In Table~\ref{tbl:stateoftheart} the previously discussed topics can be split
into the main topic which is the logging user activities and system utilisation by
analysing the obtained logs. Some of the sources proposed data-driven decision-making models to improve software maintenance.

\subsection{Need for the study}

No study was found where both the logging mechanism and analysis were combined.
There is a need to develop a method to implement an user-activity logging
mechanism to do further analysis of the logs to improve software maintenance.

\newpage
\section{Literature Study}

\subsection{Preamble}
Introduction of section.

\subsection{Software Maintenance}
Describe what software maintenance and the importance of software maintenance.

\subsection{Event-based logging}
In this section the event-based logging will be discussed and how it is part of software maintenance. 

\subsection{User-based activities in Web-based applications}
In this section the user-activities in Web-based applications will be discussed.

\subsection{Logging mechanism development}

\section{Conclusion}

