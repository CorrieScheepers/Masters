\chapter{Introduction}
\label{chap:1}
\ChapterPageStuff{1}

\section{Background}\label{section:ch1_background}
Most businesses use digital products and services to maximise their profits \cite{Gralha2018}. This increases the need to create new, innovative software systems to meet the specific needs of users. Software projects can vary in size, type, and degree of difficulty of implementation. Therefore, the quality of software is crucial. There are defined attributes and characteristics that software systems need to adhere to, which are termed software quality \cite{Khan2013}. \par A \textit{System Development Life Cycle (SDLC)}\footnote{\textbf{System Development Life Cycle (SDLC)} highlights specific procedures, practise s and guidelines for software development. Refer to the source: DOJ, "SDLC", DOJ, Available: \url{https://www.justice.gov/archive/jmd/irm/lifecycle/ch1.htm} (visited on 2023-07-25)} methodology, such as the Agile Software Development methodology, is crucial to making the software development process efficient and predictable. This enforces various degrees of discipline in the software development process to ensure that the quality of the software is acceptable for the user's requirements \cite{Khan2013, Al-Saiyd2015}. Table \ref{tbl:ch1_SDLC} lists the Life-Cycle Phases (LCP) of the SDLCs.

\begin{xltabular}{\textwidth}{p{4cm}X}
    \caption[System Development Life Cycle Phases]
    {\textit{System Development Life Cycle Phases \cite{Khan2013}}}
    \label{tbl:ch1_SDLC} \\
    
    \toprule
    \thead{Life-Cycle phase} & \thead{Description} \\
    \midrule
    \endfirsthead
    
    \caption[]{\continueCaption} \\
    \toprule
    \thead{Life-Cycle phase} & \thead{Description} \\
    \midrule
    \endhead
    
    \midrule
    \multicolumn{2}{r}{\continueText} \\ 
    \endfoot
    \endlastfoot

    \rowcolor{lightgray}
    \RaggedRight Initiation & \RaggedRight The sponsor identifies a need or an opportunity, creating a concept proposal for the new opportunity. This new opportunity will require a software solution to make the project successful for all stakeholders. \\ 
 
    \RaggedRight System Concept Development Phase & \RaggedRight In this phase, the feasibility and suitability of the concept proposal are reviewed and need approval. The Systems Boundary Document identifies the scope and requires additional approval before implementing the planning phases. \\ 
 
    \rowcolor{lightgray}
    \RaggedRight Planning & \RaggedRight The project management plan and other documents are created in the planning phase. These documents define the available budget, project resources, activities, schedules, tools, and reviews. \\ 
 
    \RaggedRight Requirements Analysis & \RaggedRight In this phase, user requirements are defined and analysed. The functional requirements are created with the Functional Requirement Document. Non-functional requirements are also defined to ensure the software system operations will not deviate from working within the non-functional specifications. These non-functional requirements can be harmful and costly to the software system's operations and potentially reduce its usability or life cycle. \\ 

    \rowcolor{lightgray}
    \RaggedRight Design & \RaggedRight In this stage, the detailed requirements are completed in the System Design Document, which describes the detailed logic specifications of the software system. \\ 
 
    \RaggedRight Development & \RaggedRight In this phase, the design specifications are converted into an executable software system. This also includes acquiring other third-party or internal software to install in specific software environments, creating and testing databases, performing test readiness reviews, and other software development activities. \\

    \rowcolor{lightgray}
    \RaggedRight Integration and Test & \RaggedRight In this phase, the software systems are integrated and systematically tested to examine whether they meet all functional requirements and other accreditation activities for the approval of the test and the user's approval using user tests. \\ 

    \RaggedRight Implementation & \RaggedRight This phase is initiated after the software systems pass the testing phase and are accepted by the users. This phase contains the implementation of the software system in a production environment and the deployment of the production version of the software. This phase continues until the software systems are operational in a final production environment. \\

    \rowcolor{lightgray}
    \RaggedRight Operations and Maintenance & \RaggedRight In this phase, the software system's performance is continuously monitored according to the defined user requirements. Any additional modifications after the initial software development are done here, and other tasks are needed to maintain the software system. Post-implementation and In-Process Reviews for the software system form part of the documentation for this LCP. \\
    
    \RaggedRight Disposition & \RaggedRight This phase is for any disposition activities to terminate the software system. Essential data are also preserved to reactivate the software system in the future. Any other termination policies and activities must be defined and documented. This ensures that the termination of the software system is performed correctly and completed and that the correct data are permanently removed or preserved. \\
    \bottomrule
\end{xltabular}

There are various adaptations of the LCP listed in \Cref{tbl:ch1_SDLC} for different SDLC methodologies used in the software development industry \cite{Al-Saiyd2015}. The planning phases of the SDLC should be well-documented, and the software architecture should be well-structured and defined to make it easier to implement the LCP related to development and maintenance \cite{Ackermann2009}.\par Each LCP ensures that the software development is correctly implemented if it is part of the adopted SDLC methodology. The SDLC methodology is implemented for the entire software system life cycle. There will be alterations to the SDLC methodology to meet any new design and development requirements. \par In implementing the SDLC methodology, some design patterns may drift away from the initial software designs \cite{Reimanis2016}. This can be due to unforeseen issues during the development phase due to time or other resource constraints. The initial design of the software system may also not be flexible enough for any newer modifications that can occur in the operational and maintenance phases. \par Software development could provide workarounds or shortcuts that resolve the identified problems in both situations. These short-term benefits will not always translate to good long-term software quality due to prioritising functionality above good software design patterns. The decrease in software code quality is caused by technical debt if the software system is not implemented with suitable SDLC practise s \cite{DeLeon-Sigg2020, Reimanis2016}. \par Technical debt can be defined as technical compromises that software engineers and developers will introduce to a software system for short-term goals that can increase the complexity and sustainability of the system in the long term \cite{Snipes2018, Gralha2018}. With the need to consistently deliver software systems for more innovative, complex, and larger software systems, the risk of introducing technical debt also increases \cite{Reimanis2016, Khan2013}. \par The operations and maintenance phase of the software system is where technical debt is usually resolved or reduced to some extent. More technical debt will increase the maintenance efforts that need to be implemented to offset the negative impact that short-term goals can potentially create. \Cref{fig:ch1_tdRepayment} represents the technical debt repayment to the software system during its entire life cycle.

\begin{figure}[!htb]
	\centering % cent the figure
	\includegraphics[width=0.9\textwidth]{img/Chapter1/TD_repayment/TD_repayment.pdf}
	\caption[Project plan with included technical debt repayment phase]
	{\textit{Project plan with included technical debt repayment phase \cite{Wiese2021}}}\label{fig:ch1_tdRepayment}
\end{figure} 

Technical debt will always be present to some degree in software systems throughout their life cycle \cite{Wiese2021}. In \Cref{fig:ch1_tdRepayment}, technical debt repayment refers to software development activities that aim to resolve technical debt issues identified after the initial deployment of the software system.\par Tickets are assigned to identified problems caused by technical debt when functional requirements are reviewed in the initial use phase. The technical debt tickets are then prioritised based on their importance and utilisation. This ensures that the development efforts in the Operations and Maintenance phase are efficiently implemented to ensure user satisfaction and sustainability of the software system.

\clearpage

According to the United States Department of Commerce, the software maintenance efforts of the SDLC Operations and Maintenance LCP in \Cref{tbl:ch1_SDLC} will contribute approximately $60\%-80\%$ of the total development cost for the entire software system life cycle \cite{Ogheneovo2014, Ackermann2009, Tang2010}. Therefore, following good software maintenance practises is necessary to avoid \cite{DeLeon-Sigg2020}:

\begin{itemize}
	\item additional software and hardware resources needed that can be costly,
	\item software quality issues,
	\item making any new modifications impossible without negatively impacting existing software features or systems and
	\item is shortening the usability of the software system, which can lead to earlier termination of the software system.
\end{itemize}

Maintenance of the LCP operation and maintenance software is essential in software development. It can directly reduce the cost and effort to create new software systems or modify them in the future and minimise technical debt \cite{Thamburaj2017, DeLeon-Sigg2020}.

\section{State of the art}\label{sec:ch1_stateOfArtSection}

\subsection{Software maintenance}\label{sec:ch1_softwareMaintenanceIntro}{}
Maintenance of software systems is a continuous process and a reduced form of software development aimed at modifying software systems while preserving their integrity for current and future operations \cite{Sneed2004, Ackermann2009, Port2017}. Software maintenance aims to improve the software in terms of the following:

\begin{itemize}
	\item \textbf{correctness:}~software systems always have some defects or faults that must be corrected to improve their traceability, consistency, and completeness.
	\item \textbf{enhancements:}\RaggedRight~existing software components need to be improved to adapt to changes in user requirements and improve system performance and sustainability.
\end{itemize}

A defined maintenance process must be followed to improve the -accuracy and enhancements of the software. According to the \textit{IEEE Standard 1219}\footnote{\textbf{IEEE Standards} documents are developed within the Technical Committees of the IEEE Societies, and the Standards Coordinating Committees of the IEEE Standards Board \cite{Mamone1994}.}, software maintenance includes the following phases \cite{Mamone1994, Hasan2012, Stojanov2017}:

\begin{itemize}
	\item Identifying the problem or modification and classifying it.
	\item Analysing the identification of the maintenance issue.
	\item Designing the solution to implement maintenance.
	\item Implementing the solution.
	\item System testing of the modified software system.
	\item Acceptance testing on the fully integrated system.
	\item Meeting the delivery requirements of the modified software system.
\end{itemize}

These maintenance phases of software cannot be omitted when the software system is still active. Software maintenance must be implemented to ensure that the software system can keep up with the new user requirements defined in the future. This will increase the maintenance required on new and old systems \cite{Niu2018, Galster2019, Hasan2012}. \Cref{fig:ch1_costsOfFixingBugs} represents the total cost of implementing software maintenance.

\begin{figure}[!htb]
	\centering % cent the figure
	\includegraphics[width=0.55\textwidth]{Chapter1/Cost_of_fixing_bugs/Cost_of_fixing_bugs.pdf}
	\caption[Resource cost of software maintenance]
	{\textit{Resource cost of software maintenance \cite{Ogheneovo2014}}}\label{fig:ch1_costsOfFixingBugs}
\end{figure}

In \Cref{fig:ch1_costsOfFixingBugs}, the total cost of the resources will increase significantly as the need for software maintenance increases. As previously stated, software maintenance can use up to $60\%-80\%$ of the unlimited resources, and it can be expected that most of the resource costs will be for the Operations and Maintenance LCP. These software resources that cost both money and time can include the following:

\begin{itemize}
	\item software developers and other support staff involved in the software maintenance process and
	\item software development tools and services such as testing software environments, analysis tools, online surveys, software fault reporting systems, etc.
\end{itemize}

\subsubsection{Software maintenance types}

Maintenance problems or modifications are regularly identified and addressed based on an initial priority ranking. This priority classification is determined using classification models to determine what type of maintenance is needed, as described in \Cref{tbl:ch1_maintenanceTypes} for the Operations and Maintenance LCP of \Cref{tbl:ch1_SDLC} \cite{Tang2010, Ping2010}.

\begin{table}[!htb]
	\centering
	\caption[Software maintenance types]
	{\textit{Software maintenance types \cite{Ping2010,Hasan2012}}}
	\label{tbl:ch1_maintenanceTypes}
	\begin{tabularx}{\textwidth}{cXc}
		\toprule
		\thead{Maintenance type} & \thead{Description} & \thead{$\%$ of maintenance activities} \\ 
		\midrule

		\rowcolor{lightgray}
		Adaptive & \raggedright Adaptive maintenance in software systems is any modification or enhancement to keep it usable with a changing or changed software environment. & $\approx 37.5\%$ \\ 

		Perfective & \raggedright Perfective maintenance consists of modifications based on changing the end-user's new requirements. It can also improve the performance or maintainability of the software system in its life cycle. & $\approx 37.5\%$ \\ 
		
		\rowcolor{lightgray}
		Corrective & \raggedright Corrective maintenance is improvements made to fix certain defects or errors in a software system. & $\approx 20\%$ \\ 

		\rowcolor{lightgray}
		Preventive & \raggedright  Preventive maintenance refers to improvements made to software systems that prevent problems in the future. & $\approx 5\%$ \\ 
		\bottomrule
	\end{tabularx}
\end{table}

According to \Cref{tbl:ch1_maintenanceTypes}, adaptive and perfective maintenance types account for approximately $75\%$ of the total maintenance of software development for the Operations and Maintenance LCP. These maintenance types address technical debt issues that may arise after the initial software deployment. They are typically identified through technical debt tickets, as shown in \Cref{fig:ch1_tdRepayment}. Adaptive and perfective maintenance is critical to ensure that the software system continues to evolve and improve, meeting the system requirements to ensure that it is usable and feasible \cite{Kumar2013}.\par Software faults or defects that require repair and deployment are inevitable. These maintenance software changes are usually minor and aim to increase the accuracy of the software system. Preventive measures may also be taken to avoid technical debt issues in the future through preventive maintenance efforts. \par However, prioritising available resources for certain parts of the software system to maintain and prevent or fix software defects can be challenging \cite{Mamone1994, Hasan2012}. The defect density of a software system is determined by the number of possible defects divided by the size of the software system, as shown in \Cref{eq:ch1_defectDensity}~\cite{Rahmani2010}:

\begin{equation}
	\label{eq:ch1_defectDensity}
	Defect~Density = \frac{CNDD}{KLoC},
\end{equation}

where:

\begin{itemize}
	\item $CNDD$ is the cumulative number of defects in the post-release version of the software system
	\item $KLoC$ (thousands of lines of code) is the size of the observed executable code in a software system. 
\end{itemize}

A lower defect density indicates good software quality \cite{Shah2012, Alenezi2016}; however, it does not necessarily imply that the software meets all user requirements but that fewer possible faults exist.\par In open-source software systems (OSS), the defect density tends to increase due to the size of the system and the number of developers working on it \cite{Rahmani2010}. Adding more developers to improve a software system may not always lead to improvements in all maintenance types listed in \Cref{tbl:ch1_maintenanceTypes}.\par This increase in the size and complexity of the software system may also lead to a higher defect density, increasing the need for corrective maintenance efforts. Therefore, it can further exacerbate developer challenges as they try to resolve maintenance issues \cite{Alenezi2016}.

\subsubsection{Problems with implementing software maintenance}\label{sec:ch1_maintenanceProblems}
Under most circumstances, maintenance is implemented if a software system does not meet the required functions specified by the user or performance requirements \cite{Ogheneovo2014, Sneed2004}. Maintenance can be difficult to implement due to the following:

\begin{itemize}
	\item \textbf{Problem domain being complex:} The software may not be well-defined or structured during the planning phases of LCP of \Cref{tbl:ch1_SDLC}. This is due to the size of software systems throughout their entire life cycle or the duplicate software components. \par There is a poor understanding of the system architecture or insufficient documentation about the software system when analysing maintenance problems \cite{Galster2019}. Software engineers and developers tend not to create or update documentation as it is time-consuming when software needs to be delivered on schedule.
	\item \textbf{Difficulties of managing the development process:} Most companies will strive to increase their digital products and services throughout the life cycle of the software project to maximise possible profits with the invested resources \cite{Niu2018}. Expanding the development process's production will only strain software systems' maintenance efforts \cite{Sneed2004}. \par Software engineers and developers feel pressured to deliver software features on time \cite{Galster2019, Lenarduzzi2017}. They will quickly feel overwhelmed and suffer from development burnout if the development process is not correctly managed to include additional software development by implementing maintenance.
	\item \textbf{Flexibility of the software:} Trying to predict the future architecture and modify it while preserving the integrity of the software may be difficult in software maintenance \cite{Garlan1999}. Software is considered flexible when it can be modified by modifying the problem domain \cite{Ogheneovo2014}. \par Most development teams will follow a software development methodology to create a future architecture that is modular and structured to preserve the development integrity of new software \cite{Vijayasarathy2016}. This will also have an impact on the type of maintenance activity (such as in \Cref{tbl:ch1_maintenanceTypes}) the development team will need to implement \cite{Thamburaj2017, Snipes2018}.
	\item \textbf{Change in user's requirements:} In software development, users will often request additional requirements for the software systems that are delivered to them \cite{Ogheneovo2014}. Modifying software systems may include new features and other features that change the initial architecture of the system. Maintenance of these systems is crucial to ensure that existing components of the system will work as intended with the new members that are added.
	\item \textbf{Environmental changes:} Rapid changes in software development are always present. As such, the need for more innovative solutions to solve new complex problems exists. These changes are not always compatible with existing software systems, even if the initial software architecture is well-defined \cite{Ogheneovo2014}. \par There will always be a need to improve the software system through third-party software updates and services. The software system must be modified to accommodate these changes, as it benefits its sustainability and operation.
	\item \textbf{Bad design:} Maintenance can be complex if the system is poorly designed when SDLC is implemented. Maintaining the system without making significant changes unfeasible. The complexity of the software system may be too complicated for some developers; this will lead to complexity in software systems within the mentioned problem domain \cite{Lenarduzzi2017}.
\end{itemize}

\subsubsection{Software maintenance prioritisation}\label{sec:ch1_maintenanceModel}
Planning maintenance operations is challenging for most software engineers and developers due to time constraints and available resources \cite{DeLeon-Sigg2020}. The increased cost of software maintenance resources is usually due to a lack of planning or preventive measures to keep the software system from degrading \cite{Alenezi2016}. Continuous analysis of identified maintenance problems or modifications enables software engineers and developers to create a preliminary plan to address these problems efficiently \cite{Port2017}. Implementing a suitable maintenance framework to resolve the issues identified reduces technical debt.\par Various maintenance models can be used to solve these issues when implementing any of the maintenance software types. A software maintenance model is an abstract representation of the evolution of software systems to keep track of all the maintenance activities when implementing software maintenance \cite{Ren2011}. The \textit{IEEE Standard 1219} for software maintenance is the standard that should be followed when planning software maintenance, as seen in \Cref{fig:ch1_ieeeModel}. 

\begin{figure}[!htb]
	\centering % cent the figure
	\includegraphics[width=0.65\textwidth]{Chapter1/IEEE_Model/IEEE_Model.pdf}
	\caption[IEEE Standard 1219 model for software maintenance]
	{\textit{IEEE Standard 1219 model for software maintenance \cite{Ren2011}}} \label{fig:ch1_ieeeModel}
\end{figure}

The software maintenance model in \Cref{fig:ch1_ieeeModel} emphasises that software defects or faults must be identified and classified for maintenance. If the maintenance effort is worth implementing, a feasibility analysis of any required changes should be done. Several resources will be allocated to execute maintenance, affecting the design and implementation phase.\par For feasibility analysis, using a system characterisation report may help identify possible maintenance focus points in a software system \cite{Araujo2021}. This will increase the effectiveness of designing solutions to implement maintenance because a system assessment focus can be made. The metrics can be defined as:

\begin{itemize}
	\item positively or negatively impacts the performance of the system,
	\item fulfil the defined user's requirements,
	\item increase user engagement with the defect and fault fixes or expand existing software components with additional features and
	\item is worth a large portion of the user base to implement.
\end{itemize}

User engagement with the software system will determine whether the software system is sustainable in generating revenue for the organisation. It may be the most crucial focus metric for feasibility analysis to determine whether a maintenance effort is worth implementing \cite{Araujo2021}. Compared to the model in \Cref{fig:ch1_maintenanceFlow}, a maintenance flow model can resolve this problem.  

\Cref{fig:ch1_maintenanceFlow} is an example of a practical maintenance flow model that an organisation would likely use to implement software maintenance. Initially, a developer will complete a request form that indicates a new problem or feature request that needs to be implemented \cite{Tang2010}. 

\par After all maintenance tasks are defined, the development team will prioritise the higher-rated issues that need to be solved. This process will repeat until all the work for that specific software system is completed. 

\par To fully follow the \textit{IEEE Standard 1219} of implementing maintenance on a software system, the defects or areas of improvement should be identified. The analysis of event logs can detect hidden flaws or performance problems in a software system to implement software maintenance \cite{Cinque2013, Rong2018a, Levin2019}.

\par System and acceptance tests are essential to ensure the system is fully functional and meets user requirements. After the system is thoroughly tested and approved, it will be available to the user, and the maintenance process will start again when there are new improvements to the software system.\par In \Cref{sec:ch1_softwareMaintenanceIntro}, it was identified that software maintenance is essential to fulfilling the user's requirements. For any maintenance model to be effective, as described in \Cref{sec:ch1_maintenanceModel}, the software maintenance model must prioritise maintenance issues efficiently. \par Most organisations' maintenance models will be based on \Cref{fig:ch1_ieeeModel} to manage their Operation and Maintenance LCP maintenance efforts. Due to the problems with the implementation of maintenance discussed in \Cref{sec:ch1_maintenanceProblems} up to $50\%$ of a software engineer or developer's total time invested in implementing maintenance is to understand what the software system is supposed to do \cite{Tang2010}. \par If the issue is a problem in the software system, the severity of the problem must be assessed to decide on the priority level to resolve it. This type of maintenance is mainly corrective and can also be preventive if it is a possible solution to prevent software failures in the future \cite{Tang2010}. Other maintenance requests are adaptive or perfective and are usually placed in the development team's task queue. \par Prioritising maintenance for the user's requirement to extend usability and increase user satisfaction is preferable to maximise profits. When no suitable software maintenance model is used, software maintenance is often reactive \cite{Araujo2021}. However, this is not a sustainable maintenance policy for larger, more complex software systems. \par A systematic characterisation of the software system needs to be made to prioritise maintenance more efficiently. In \Cref{sec:ch1_maintenanceModel}, user engagement with the software system has been identified as an essential metric when implementing maintenance.

\clearpage

Knowing what the user uses or interacts with in the software system provides valuable data to the development team. But to access those data, some form of tracking is needed to obtain the data automatically.

\begin{figure}[!htb]
	\centering % cent the figure
	\includegraphics[width=0.98\textwidth]{Chapter1/Maintenance_flow/Maintenance_Flow.pdf}
	\caption[Maintenance flow model]
	{\textit{Maintenance flow model \cite{Tang2010}}} \label{fig:ch1_maintenanceFlow}
\end{figure}

\clearpage

\subsection{Event logging}\label{sec:ch1_eventLogging}
As described in \Cref{sec:ch1_maintenanceModel}, a tracking method is needed to capture data on user engagement with the software system for maintenance purposes. It is a common practise in the software industry to record complex system run-time information in event logs. Developers or engineers can analyse these event logs later to solve software-related problems \cite{Zhu2019}. \par Event logging is a proven implementation to obtain information on the behaviour of software systems \cite{Baccanico2014}. Event logs are textual files generated by the software system that collect data on reported events of interest during various software system operations \cite{Cinque2013, Baccanico2014}.

\par The technique to collect numeric or textual data that describes the behaviour of a computer system is called the event log \cite{Pecchia2015, Baccanico2014}. Event logs collect textual data containing records of events in a software system and are used for system management tasks, as seen in \Cref{tbl:ch1_eventLogsUsage} \cite{Rong2018a, Rong2018, Baccanico2014}. Event logging has three main purposes \cite{Pecchia2015, Baccanico2014}:

\begin{itemize}
	\item \textbf{state dump} reports the values of certain variables or data structures inside the software system.
	\item \textbf{execution tracing} is the reporting of certain states of the software system or what is currently happening in the software system and
	\item \textbf{event reporting} focuses on any desired events in the software system that have textual information of that event.
\end{itemize}

Event logs are used mainly for event reporting to support debugging and system integration activities to reduce the amount of code that needs to be inspected \cite{Baccanico2014}. \Cref{tbl:ch1_eventLogsUsage} is the industry's most common use of event logging.

\begin{xltabular}{\textwidth}{p{4cm}X}
	\caption[Event logs usage]
	{\textit{Event logs usage}}
	\label{tbl:ch1_eventLogsUsage} \\
	\toprule
	\thead{Usage} & \thead{Description} \\
	\midrule
	\endfirsthead

	\caption[]{\continueCaption} \\
	\toprule
	\thead{Usage} & \thead{Description} \\
	\midrule
	\endhead

	\midrule
	\multicolumn{2}{r}{\continueText} \\ 
	\endfoot
	\endlastfoot

	\rowcolor{lightgray}
	\RaggedRight Debugging of software systems and services & \RaggedRight Event logging is used mainly to record events or behaviours of software systems or services during its runtime \cite{Rong2018a}. \\

	\RaggedRight Anomaly detection & \RaggedRight Event logs can detect any abnormal system behaviour using an anomalous detection algorithm using log data \cite{Gurumdimma2016}. This can also be used to find vulnerabilities or predict software environment defects \cite{Dwyer2013}. \\

	\rowcolor{lightgray}
	Performance diagnosis & \RaggedRight Software performance is important to produce quality software for the end user \cite{EvangelinGeetha2007, Baccanico2014}. Making informed decisions about improving the software system or service to improve performance and other financial and resource implications is also important. \par This type of performance event log of software systems or services is used to monitor the software system, which is useful for tuning resources, balancing load and checking system scalability throughout the life cycle of the software system or service \cite{Song2017}. \\
	
	Auditing & \RaggedRight In a software environment, significant changes in database data may need to be recorded for audit purposes \cite{Rong2018a}. All establishments and enterprises must ensure compliance with industry regulations is met with their software systems by adding audit logs. They are also legally bound to have audit logs to provide evidence for any legal investigation or administrative tasks to maintain accountability. \\

	\rowcolor{lightgray}
	Error and failure analysis & \RaggedRight Event logs are used to analyse the failure behaviours of software systems, which allow software engineers or developers to understand system failure, find the root cause of these failures, prevent them, and improve the reliability of future versions of the system \cite{Cinque2013}.\\


	\RaggedRight Analysis of security alerts & \RaggedRight Security is a significant concern in any software environment or information technology infrastructure \cite{Pathan2014, Dwyer2013}. It is essential to know the overall security status of the software system. \\
	\bottomrule
\end{xltabular}

\paragraph{Logging practise in software development}\label{sec:ch1_loggingPractice}\leavevmode\\
Logging practise in software development is not always well documented, and multiple logging mechanisms can be implemented in the same software system \cite{Pecchia2015, Kitchenham2007}. In modern software systems, logging practise is crucial to software development and maintenance throughout its life cycle \cite{Rong2018}.\par In \Cref{apx:loggingPractice}, new studies have focused on providing practical logging practise guidelines to software engineers and developers. Logging in the industry uses many third-party log libraries and frameworks, such as Apache's log4net and Microsoft's ULS frameworks \cite{Zhu2015, Rong2018}.\par Software engineers and developers can use these tools to implement logging in a compatible software system. However, it will still need to know how to place the logging points to obtain the desired logs strategically. The logging practise can be implemented using the guides provided by the tools and other online guides. When using a third-party logging mechanism, the software engineers and developers will, in most cases, need to:

\begin{itemize}
	\item add the logging points in the software system at locations where it can capture the desired logs and
	\item enables the log parsing stage to write a log entry into a database.
\end{itemize}

Logging guides can give examples and suggestions on where to place the log points. However, it can still be difficult for software engineers and developers to identify these desired locations because logging guides are either mostly application-specific or the logging mechanism can only capture certain event types to create event logs. \par For more custom logging, the software engineers and developers will need to develop new logging mechanisms. This introduces new requirements for the logging mechanism to be functional. For a log-keeping practise to be successful, two critical problems need to be resolved \cite{Zhu2015, Zhu2019, Rong2018}:

\begin{itemize}
	\item \textbf{What needs to be logged?}\par In \Cref{tbl:ch1_eventLogsUsage}, recording is diverse and will impact how the logging mechanism will be designed.
	\item \textbf{Where to log?} \par Different types of logs will only be present in the software system at certain locations during runtime. Knowing what to log narrows down which locations can be used to obtain certain events while they are present.
\end{itemize}

To answer these two questions, the event log must meet specific log quality requirements. This ensures the created logs are correct and consistent when extracted and viewed. 

\subsubsection{Logging quality}\label{sec:ch1_loggingQuality}

Software engineers and developers must make informed logging decisions for event logging. These decisions may negatively affect the software system's run-time operations and event logging efficiency \cite{Zhu2015, Zhu2019, Kherbouche2017}. An event log quality model is defined in \Cref{fig:ch1_EventQModel} to ensure that the event logs have consistent integrity when capturing the event log data.

\clearpage

\begin{figure}[!htb]
	\centering % cent the figure
	\includegraphics[width=0.99\textwidth]{Chapter1/Event_Log_Model/Event_Log_Model.pdf}
	\caption[Quality model for event logs]
	{\textit{Quality model for event logs \cite{Kherbouche2017}}} \label{fig:ch1_EventQModel}
\end{figure}

The event log quality model in \Cref{fig:ch1_EventQModel} comprises four different levels to define each property of the event log quality model:

\begin{itemize}
	\item Root level, these are the main requirements for the event log quality model,
	\item Dimension level, which compromises four main dimensions of the event log quality model according to \cite{Kherbouche2017}: complexity, accuracy, consistency, and completeness of the event log,
	\item Attributes level, for which a set of quality attributes for the dimension level are defined,
	\item Measurement level, which should be defined in the design process of the logging mechanism to achieve the attributes and dimension levels.
\end{itemize}

\clearpage

Each of the four dimensions is discussed in detail in the \Cref{fig:ch1_EventQModel}:

\begin{itemize}
	\item \textbf{Accuracy of event logging}\par It can become difficult to record every event in a software system as these systems will become larger and more complex during their development life cycle \cite{Stojanov2017}. Precisely capturing certain events that need to be logged must be consistent to ensure that the data are trustworthy and reliable. The log attributes of the event log also need to be precisely captured for each event log and should be correct.\par Capturing more event logs does not guarantee that the accuracy of the event log is acceptable, as the log's attributes can be incorrect or cause duplicated event logs if the data are the same in a sequence of event logs. \Cref{tbl:ch1_loggingTooMuch} indicates common problems associated with the accuracy.

	\begin{table}[!htb]
		\centering
		\caption[Problems with too much logging]
		{\textit{Problems with too much logging \cite{Zhu2015}}}
		\label{tbl:ch1_loggingTooMuch}
		\begin{tabularx}{\linewidth}{p{3cm}X}
			\toprule 
			\thead{Problem}  & \thead{Description} \\
			\midrule

			\rowcolor{lightgray}
			Excess code & Adding multiple logging points may increase code added to the software system to capture the event logs. The code may take some time to write and maintain. This can increase the structural and behavioural complexity of the code throughout its life cycle. \\

			System resource impact & With the additional code needed to capture the logs, the usage of system resources, such as CPU and I/O channels, will increase. This may negatively impact the performance of existing system operations or increase the cost of keeping the system at the same operating speed by increasing the system resources.\\

			\rowcolor{lightgray}
			Unusable logs & Adding numerous logging points or logging too many at points can produce numerous trivial or useless logs that will not improve system utilisation analysis. When implementing the system utilisation analysis stage, the logs might need to be filtered or modified to be more meaningful. As many as $70\%$ of the logs may be irrelevant \cite{Fedaghi2010}. \par The software engineers and developers write the logs, which can sometimes be irrelevant to other managers or system administrators when implementing a log analysis report. More event logs can have missing or incomplete logging attributes due to the additional logging points. The increase in the behavioural complexity of the log can impact the decisions made to improve software maintenance. \\
			\bottomrule
		\end{tabularx}
	\end{table}

	The accuracy and trustworthiness of the event log are more important than capturing the many available event logs in a software system \cite{Zhu2015, Jans2012}. The additional unnecessary logs will also take up more storage space, increasing costs and possibly influencing the software system's performance.
	
	\clearpage

	\item \textbf{Event log complexities}\par Software always involves complexity that increases as the software system becomes more considerable. For the event log quality model, the complexity of the event can be split into two different complexities, namely \cite{Kherbouche2017}:

	\begin{itemize}
		\item Structural complexity is the application of different algorithms in the software that allows the event log to be evaluated when it occurs, which can alter the behaviour of the event log.
		\item Behavioural complexity is the complexity of the behaviour of the event logs that refers to the number of smaller events in each captured trace and the different variations of these traces within an event log.
	\end{itemize}

	These two complexity attributes can be costly when the event logging mechanism needs to be constantly maintained in large software systems, which can affect the rest of the system’s performance or the integrity of the captured event logs \cite{Ogheneovo2014}. The constant modification of the event log software can be due to technical debt, as the complexities of the event log system lead to technical issues when attempting to log an event or when they are not compatible with other systems \cite{DeLeon-Sigg2020}.  

 	\item \textbf{Consistency of event logging}\par The accuracy and consistency of event logs are critical to making reliable decisions based on the identified behaviour of the software system and the historical data in the event logs, which are discussed in the previous dimension of event log quality \cite{Stojanov2017, Kherbouche2017}. With the accuracy and trustworthiness of the event logs properly applied, their consistency should be accepted as correct and verifiable when comparing them to the software system. \par An event log quality model is essential to ensure that the logs are high quality for the log analysis data mining process; therefore, the event log data should be consistent. To ensure the consistency of the event logs, the structure of the event logging points and log parsers should be compatible with capturing all the critical log attributes. The event log data should also be consistently analysed with different methods used as part of the consistency of the event logging process.

	\item \textbf{Completeness of event logging}\par The event logs will be analysed later; the logs should be fully complete when used, as some of the other logging attributes might not be available at that stage. The event logs' available attributes should be accurately captured before storing them in a database to ensure that no missing event data or missing events are discarded due to incomplete information. Two types of completeness attributes exclude the availability attribute in \Cref{fig:ch1_EventQModel}:

	\begin{itemize}
		\item Local completeness refers to all event data that can be captured for an instance of the event that takes place that can be added as a log attribute to the event log \cite{Kherbouche2017, VanDerAalst2004}.
		\item Global completeness refers to the occurrence of all possible outcomes or behaviours of the event logs that can be captured, which is required for the system utilisation analysis \cite{Kherbouche2017, VanDerAalst2004}.
	\end{itemize}

	Ensuring that both completeness levels are achieved and that the event log data is complete can impact performance if specific data are not directly available during the instance when the event has occurred. The logging mechanism must capture this as efficiently as possible without causing performance issues to the rest of the software system's operations \cite{Zhu2015, Zhu2019}. 
\end{itemize}

\paragraph{Logging attributes}\leavevmode\\
Before the logs can be parsed to a structured data set, the key attributes must be defined for the event log \cite{Bekeneva2020}. The attributes describing the event log in \Cref{tbl:ch1_logBasicAttributes} are the essential attributes a log event should have. 

\begin{table}[!htb]
	\centering
	\caption[Basic log event attributes]
	{\textit{Basic log event attributes \cite{Bekeneva2020}}}
	\label{tbl:ch1_logBasicAttributes}
	\begin{tabularx}{\textwidth}{lX}
		\toprule 
		\thead{Attribute} & \thead{Description} \\
		\midrule

		\rowcolor{lightgray}
		Case number & Unique identifiers for each log event. This is usually the primary identifier of the log event. \\

		Timestamp & The time and date that the log event occurred. This is part of identifying the order of events or traces along with the case number of the event log \cite{Kherbouche2017}. \\

		\rowcolor{lightgray}
		Event type & Each log event can be grouped with other log events with similar actions. These event-type attributes should be classified based on a state change, failure to execute an instruction, or an activity occurrence, such as service availability \cite{Fedaghi2010}. \par The event type is usually also the log level. The log level in event logging reflects the severity of the event log \cite{Rong2020}. An event of interest may have different log levels, making it easier to capture the kinds of events software engineers and developers are interested in. \\

		Originator & The origin of the event in the software system. This can be parts of the software that perform the event action or was the cause for the event to be initiated by another part of the software. \\

		\rowcolor{lightgray}
		Other metadata & This is any other relevant information that can be used as the event log's attribute that further expands the information of the log event. This can be an additional field or many other individual attribute fields.\\
		\bottomrule
	\end{tabularx}
\end{table}

\clearpage

\subsubsection{Logging parsing and log points}\label{sec:ch1_loggignPoints}
Knowing what to log can significantly reduce any overhead the logging mechanism may produce in the software system \cite{Jia2018, Pecchia2015}. Preserving quality (as described in \Cref{sec:ch1_loggingQuality}) is necessary to ensure that the obtained logs fulfil their purpose when analysed.

These attributes make it possible to mine and analyse the logs and increase the precision and reliability of the event log \cite{Kherbouche2017}. The case number and timestamp attributes in \Cref{tbl:ch1_logBasicAttributes} can be defined anytime during the logging process. This is not the case for the rest of the attributes. \par Every log should have a defined action that will sort it into a group of logs that can be defined as the type of event. These event types can be predefined by what is expected from the event action or will need to be observed later in the analysis of the logs in case there is no explicit grouping of the logs \cite{Bekeneva2020, Fedaghi2010}.

\paragraph{Log points}\leavevmode\\
The sources of the log event assist in determining where the event took place. For event logs, it is essential to recreate scenarios or actions based on the relevant parts of the software system that participated in the event action.\par Other metadata can increase the quality of the log by providing additional information about the executed software instructions. These attributes add more information that can be used to recreate the scenario or action that may be unique parameters or other events that participated in the event log.\par Obtaining the attributes in \Cref{tbl:ch1_logBasicAttributes}, an instruction generates the log and parses it onto a data set. These log instructions are called logging points in the software environment \cite{Pecchia2015, Zhu2015}. They can be any instruction, such as a print function that displays the information for the user to more complex processes or libraries that third-party developers can create. \Cref{fig:ch1_logParsing} is an example of a logging point parsing a log message in a structured log. 

\clearpage

\begin{figure}[!htb]
	\centering % cent the figure
	\includegraphics[width=0.95\textwidth]{Chapter1/Log_Parsing_Example/Log_Parsing_Example.pdf}
	\caption[An illustrative example of log parsing]
	{\textit{An illustrative example of log parsing \cite{Zhu2019}}} \label{fig:ch1_logParsing}
\end{figure}

The defined attributes are captured by the logging point when the event takes place or occurs. The created log message is then parsed into a structured log to be safe in a database or displayed. The logging point should be strategically placed to capture the attributes required to complete the log event \cite{Fedaghi2010}.\par Determine where to place the logging point in software directly affected by the attributes and whether the captured log will be high quality as described in \Cref{sec:ch1_loggingQuality}. The availability of consistent, high-quality logs will directly impact the process of mining the logs when analysing them \cite{Kherbouche2017}. \par To strategically place a logging point, developers need to consider what the activation of the logging point will be during the run time of the software system \cite{Pecchia2015, Cinque2013}. The activation can be straightforward if a statement meets specific criteria or instructions that execute after an event or action occurs (e.g. run-time errors). \par The log level or event type outlined in \Cref{tbl:ch1_logBasicAttributes} can be used to determine where the log points should be placed. The aim of establishing the logging point should be to capture all the log attributes when the event of interest has happened.

\subsubsection{Log analysis in web-based applications}
With the log parsing and log points defined in \Cref{sec:ch1_loggignPoints}, a log analysis can be made from the stored event logs. The log analysis is the data mining process focused on the software system's generated event logs \cite{Slaninova2014,Hasiloglu2018}. \par The defined log attributes outlined in \Cref{tbl:ch1_logBasicAttributes} will be used to complete the log analysis. Each software system will have some variation of the log attributes in \Cref{tbl:ch1_logBasicAttributes}. For a web-based application, the log attributes will also contain data about the requests between a Web server and a Web client and its responses \cite{Slaninova2014, Dhanalakshmi2016}. This will also reflect on the log level at which weblogs are obtained for log analysis.


\subsection{Log analysis}\label{sec:ch1_systemUtilisation}
Gathering usage statistics and user behaviour data in web-based applications is called Web analytics \cite{Kocsis2012}. Web analytics can be used for user modelling efforts and is a form of log analysis. User modelling in software engineering is the customisation and adaptation of software systems to the users' required needs \cite{Waqar2017, Paliouras1999}. \par User modelling can also include the implementation of software maintenance, as maintenance is an adaptation of the software system to the user's needs, which is the utilisation analysis using event logs. The web analytics for this study focuses on different analytics, as seen in \Cref{tbl:ch1_webAnalytics}. 

\begin{table}[!htb]
	\centering
	\caption[Web analytic for user-based data]
	{\textit{Web analytic for user-based data}}
	\label{tbl:ch1_webAnalytics}
	\begin{tabularx}{\textwidth}{lX}
		\toprule 
		\thead{Analytic}  & \thead{Description} \\
		\midrule

		\rowcolor{lightgray}
		Identity of the user & This is any information about the user's identity in the software system. Users can have different roles when using the software system, such as a system admin or general user. These roles mostly dictate what the user can access and do on a website. \\

		Site interaction & The different Websites the user is accessing during their active Web session. This would also contain all the information about the following: 
		\begin{itemize}
			\item how often the users visit a website,
			\item how much time they spent on a specific website, and
			\item navigation between different web pages of the website.
		\end{itemize}
		\\
		\bottomrule
	\end{tabularx}
\end{table}

These same analytics can be used for none-web-based applications because the: 
\begin{itemize}
	\item identity of the user can be captured if the software system uses a software license and
	\item different parts of the system can be tracked, as well as services the user uses.
\end{itemize} 

\subsubsection{Analytic tools for event logs}
Numerous third-party analytic tools for event log data monitoring and management to visualise log data graphically. Choosing the correct device to use can depend on what the software engineers and developers want to analyse and the availability of the tools due to external factors such as cost and usability. \par Event logging monitoring and management tools are sometimes underused and are often not used to their full potential when analysing event logs \cite{Fedaghi2010}. This can be due to the log quality of event logs not meeting the standards of \Cref{fig:ch1_EventQModel}. 

\newcommand{\studyGap}{Implementing software maintenance efficiently with the limited available resources organisations or individuals have is crucial to improving the efficiency of software maintenance efforts. Assisting developers in making better-informed decisions about which systems to prioritise maintenance can improve software maintenance.}

\subsection{Gap identification}\label{sec:ch1_gapIdentification}
\Cref{section:ch1_background} emphasises that software maintenance is an important stage in the SDLC of software systems. \studyGap

\subsubsection{State of the art topics}
From the literature, there were a few critical focus points identified to create a log-in mechanism for user-based activities. These focus points exist for the research done in \Cref{sec:ch1_softwareMaintenanceIntro,sec:ch1_eventLogging,sec:ch1_systemUtilisation}.\par The research for this study focused on accredited peer-reviewed published journals or articles for the last two decades (2000-2022). Some exceptions include literature older than 20 years since software maintenance has been essential to modern software development over the past few decades. These studies were obtained from journals in the IEEE Digital Library by focussing on three key topics: software maintenance, event logging, and log analysis. Each of the three primary states of the art topics is divided into subtopics that explore different studies relevant to the main issue.

\paragraph{Software maintenance} \leavevmode\\
Software maintenance implementation in industry and best practises when implementing subtopics of software maintenance are described in \Cref{tbl:ch1_soaSoftwareMaintennace}. \par The three different state of the art subtopics for software maintenance in \Cref{tbl:ch1_soaSoftwareMaintennace} evaluate the studies that discuss software maintenance and software maintenance implementation. These subtopics focus on what is needed to implement software maintenance for any given software environment using industry standards.

\clearpage

\begin{table}[!htb]
	\centering
	\caption[Software maintenance state of the art sub-topics]
	{\textit{Software maintenance state of the art sub-topics}}
	\label{tbl:ch1_soaSoftwareMaintennace}
	\begin{tabularx}{\linewidth}{lXX}
		\toprule
		\thead{Topic}  & \thead{Description} & \thead{Evaluation criteria}\\
		\midrule
		\rowcolor{lightgray}
		Models & \RaggedRight Software maintenance models and implementation in industry. & \RaggedRight \begin{enumerate}
			\item Did the study focus on software maintenance?
			\item Did the study discuss different software maintenance implementations?
		\end{enumerate} \\
		Problems & Challenges with implementing software maintenance. & \RaggedRight \begin{enumerate}
			\item Did the study identify frequent problems with implementing software maintenance?
		\end{enumerate}\\
		\rowcolor{lightgray}
		Prioritisation & Prioritisation methods. & \RaggedRight \begin{enumerate}
			\item Did the study discuss the importance of prioritisation techniques of certain software maintenance activities, models, or software components?
		\end{enumerate} \\
		\bottomrule
	\end{tabularx}
\end{table}

\paragraph{Event logging} \leavevmode\\
Event logging in a software environment and activities implementing state of the art event logging subtopics are described in \Cref{tbl:ch1_soaEventLogging}. \par The event logging subtopics aim to create or use a suitable logging mechanism for the log analysis requirements through the captured logging points. \Cref{tbl:ch1_soaEventLogging} identifies studies that will adhere to the best industry practises to create the logging mechanism and use or create the design methodology needed for the logging mechanism.

\clearpage

\begin{table}[!htb]
	\centering
	\caption[Event logging state of the art topics]
	{\textit{Event logging state of the art topics}}
	\label{tbl:ch1_soaEventLogging}
	\begin{tabularx}{\linewidth}{lXX}
		\toprule
		\thead{Topic}  & \thead{Description} & \thead{Evaluation criteria}\\
		\midrule
		\rowcolor{lightgray}
		Points & \RaggedRight Software logging points and log attribute identification & \RaggedRight \begin{enumerate}
			\item Did the study discuss what needs to be logged or how to identify key logging attributes from certain software events?
			\item Did the study aim to create or identify key logging points for specific log analysis purposes?
		\end{enumerate} \\
		Parsing & \RaggedRight Challenges with implementing software maintenance. & \RaggedRight \begin{enumerate}
			\item Did the study aim to create or identify key logging points for specific log analysis purposes in a software environment?
			\item Did the study provide a method or standards for log parsing to adhere to a log quality model?
		\end{enumerate}\\	
		\bottomrule
	\end{tabularx}
\end{table}

\paragraph{Log analysis} \leavevmode\\
A log analysis of software systems using state of the art event logging topics is described in \Cref{tbl:ch1_soaLogAnalysis}.

\begin{table}[!htb]
	\centering
	\caption[A log analysis using state of the art topics]
	{\textit{A log analysis using state of the art topics}}
	\label{tbl:ch1_soaLogAnalysis}
	\begin{tabularx}{\linewidth}{lXX}
		\toprule
		\thead{Topic}  & \thead{Description} & \thead{Evaluation criteria}\\
		\midrule
		\rowcolor{lightgray}
		Utilisation & \RaggedRight Utilisation log analysis for user-based event logging. & \RaggedRight \begin{enumerate}
			\item Is a utilisation analysis done?
			\item Is prioritisation done based on the utilisation log analogy?
		\end{enumerate} \\
		\bottomrule
	\end{tabularx}
\end{table}

In \Cref{tbl:ch1_soaLogAnalysis}, the topics focus more on studies that implement a log analysis for utilisation purposes. Tracking certain events is therefore essential for log analysis to ensure that the precision criteria of log quality can be achieved by \Cref{sec:ch1_loggingQuality} to efficiently use or create a log analysis report for software maintenance.

\clearpage

\subsubsection{State of the art summary}
Using the defined criteria created for the state of the art topics in \Cref{tbl:ch1_soaEventLogging,tbl:ch1_soaLogAnalysis,tbl:ch1_soaSoftwareMaintennace}, state of the art research for the obtained studies is done in \Cref{tbl:ch1_stateOfTheArt2}.

\newcommand{\sotaCore}{\rowcolor{lightgray}
\cite{Ogheneovo2014} & Partial & \cmark & \cmark & \xmark & \xmark & \xmark \\
\cite{Tang2010} & Partial & Partial & \cmark & \xmark & \xmark & \xmark \\
\rowcolor{lightgray}
\cite{Sneed2004} & Partial & \cmark & \cmark & \xmark & \xmark & \xmark \\
\cite{Stojanov2017} & \xmark & \cmark & \cmark & \xmark & \xmark & \xmark \\
\rowcolor{lightgray}
\cite{Galster2019} & \cmark & Partial & \xmark & \xmark & \xmark & \xmark \\
\cite{Lenarduzzi2017} & Partial & \cmark & \cmark & \xmark & \xmark & \xmark \\
\rowcolor{lightgray}
\cite{Araujo2021} & Partial & \xmark & \cmark & \xmark & \xmark & \xmark \\	
\cite{Zhu2019} & \xmark & \xmark & \xmark & \cmark & Partial & \xmark \\
\rowcolor{lightgray}
\cite{Rong2018} & \xmark & \xmark & \xmark & Partial & Partial & \xmark \\
\cite{Zhu2015} & \xmark & \xmark & \xmark & \cmark & \cmark & \xmark \\
\rowcolor{lightgray}
\cite{Kherbouche2017} & \xmark & \xmark & \xmark & Partial & Partial & \xmark \\
\cite{Hasiloglu2018} & \xmark & \xmark & \xmark & \cmark & \cmark & Partial \\
\rowcolor{lightgray}
\cite{Slaninova2014} & \xmark & \xmark & \xmark & \cmark & \cmark & Partial \\
\cite{Waqar2017} & \xmark & \xmark & \xmark & \xmark & \xmark & \cmark \\
\rowcolor{lightgray}
\cite{Kumar2017} & \xmark & \xmark & \xmark & \xmark & \xmark & \cmark \\
%\cite{Bogatinovski2023} & \xmark & \xmark & \xmark & \xmark & \xmark & \cmark \\
}

\begin{table}[!htb]
	\centering
	\caption[State of the art]
	{\textit{State of the art}}
	\label{tbl:ch1_stateOfTheArt2}
	\begin{tabularx}{\textwidth}{cYYYYYY}
		\toprule
		\thead{Ref.} & \multicolumn{3}{c}{\thead{Software maintenance}} & \multicolumn{2}{c}{\thead{Event logging}} & \multicolumn{1}{c}{\thead{Log analysis}} \\ 
		 \cmidrule(lr){2-4} \cmidrule(lr){5-6} \cmidrule(lr){7-7}
		 & \RaggedRight \thead{Models} & \RaggedRight \thead{Problems} & \RaggedRight \thead{Prioritisation} & \thead{Parsing} & \thead{Points} & \RaggedRight \thead{Utilisation} \\ 		
		\midrule
			\sotaCore
		\bottomrule
	\end{tabularx}	
\end{table}

In \Cref{tbl:ch1_stateOfTheArt2}, studies marked with partial status satisfied the subtopic evaluation criteria to some extent. These studies still contained much-needed literature for the subtopic, even if it was marked as practically meeting the evaluation criteria. \par In \Cref{tbl:ch1_stateOfTheArt2}, the three main topics split the obtained literature into two main parts. The first part comprises software maintenance and how to implement and prioritise it using a suitable software maintenance model. The second part includes event logging and log analysis. 

\clearpage

\subsubsection{Review of the state of art studies}
This section describes the state of the art studies identified in \Cref{tbl:ch1_stateOfTheArt2} and provides commentary on the relevance of each to this study.




\begin{table}[htbp]
	\centering
	\rowcolors{2}{gray!5}{gray!5}
	\begin{tabularx}{\textwidth}{|l|X|}
		\hline
		\rowcolor{pastelgreen!40!black}
		\multicolumn{2}{|c|}{On the Relationship between Software Complexity and Maintenance Costs\cite{Ogheneovo2014}} \\ \hline
		\textbf{Study Summary:} & Defines the different problems with the implementation of software maintenance and its complexities as described in Section 1 (Maintenance Problems). \\ \hline
		\textbf{Method:} & They used three different software operating systems to compare their estimated development costs, maintenance costs, code lines, and code complexity. This is used in the analysis to determine the relationship between the complexity of the software and the maintenance costs. \\ \hline
		\textbf{Results:} & The results showed that the estimated maintenance costs are high enough that the complexity of the software needs management to reduce these costs. \\ \hline
		\textbf{Commentary:} & This study is used in this review of the literature when discussing the importance of software maintenance difficulties in the implementation of a software maintenance model. The study does not provide solutions on how to implement software maintenance effectively. It also explores the strategies and best practices in its review of the literature to help with software maintenance resource costs. \\ \hline
	\end{tabularx}
\end{table}



\begin{tcolorbox}[colback=gray!5!white, colframe=pastelgreen!40!black, title=On the Relationship between Software Complexity and Maintenance Costs\cite{Ogheneovo2014}]
	\begin{minipage}[t]{0.25\textwidth}
		\textbf{Study summary:}
	\end{minipage}
	\hfill
	\begin{minipage}[t]{0.65\textwidth}
		Defines the different problems with the implementation of software maintenance and its complexities
as described in \Cref{sec:ch1_maintenanceProblems}. 
	\end{minipage}

	% Add space here
	\vspace{0.75em} 

	\begin{minipage}[t]{0.25\textwidth}
		\textbf{Method:}
	\end{minipage}
	\hfill
	\begin{minipage}[t]{0.65\textwidth}
		They used three different software operating systems to compare their estimated development costs, maintenance costs, code lines, and code complexity. This is used in the analysis to determine the relationship between the complexity of the software and the maintenance costs.
	\end{minipage}

	% Add space here
	\vspace{0.75em} 

	\begin{minipage}[t]{0.25\textwidth}
		\textbf{Results:}
	\end{minipage}
	\hfill
	\begin{minipage}[t]{0.65\textwidth}
		The results showed that the estimated maintenance costs are high enough that the complexity of the software needs management to reduce these costs.
	\end{minipage}

	% Add space here
	\vspace{0.75em} 

	\begin{minipage}[t]{0.25\textwidth}
		\textbf{Commentary:}
	\end{minipage}
	\hfill
	\begin{minipage}[t]{0.65\textwidth}
		This study is used in this review of the literature when discussing the importance of software
maintenance difficulties in the implementation of a software maintenance model. The study does not
provide solutions on how to implement software maintenance effectively. It also explores the strategies and best
practises in its review of the literature to help with software maintenance resource costs.
	\end{minipage}
	\tcblower
	\textbf{Study keywords:} Software; Software Maintenance; Software Evolution; maintenance costs;
	software; software evolution; software maintenance
\end{tcolorbox}

\begin{tcolorbox}[colback=gray!5!white, colframe=pastelgreen!40!black, title=Metric-based tracking management in software maintenance\cite{Tang2010}]
	\begin{minipage}[t]{0.25\textwidth}
		\textbf{Study summary:}
	\end{minipage}
	\hfill
	\begin{minipage}[t]{0.65\textwidth}
		This study explores what happens when software maintenance needs to be implemented with certain roleplayers involved.
	\end{minipage}

	% Add space here
	\vspace{0.75em} 

	\begin{minipage}[t]{0.25\textwidth}
		\textbf{Method:}
	\end{minipage}
	\hfill
	\begin{minipage}[t]{0.65\textwidth}
		Assigned roles of each individual in the software maintenance process (user, coordinator, decision maker and maintenance operator) and evaluated their impact on the software maintenance process. 
	\end{minipage}

	% Add space here
	\vspace{0.75em} 

	\begin{minipage}[t]{0.25\textwidth}
		\textbf{Results:}
	\end{minipage}
	\hfill
	\begin{minipage}[t]{0.65\textwidth}
		The results showed that there is explicitly a management issue when software maintenance is implemented. 
	\end{minipage}

	% Add space here
	\vspace{0.75em} 

	\begin{minipage}[t]{0.25\textwidth}
		\textbf{Commentary:}
	\end{minipage}
	\hfill
	\begin{minipage}[t]{0.65\textwidth}
		The study recommended implementing a practical software maintenance model to solve management problems and make maintenance processes more efficient.
	\end{minipage}
	\tcblower
	\textbf{Study keywords:} Metric; Software maintenance; Tracking management
\end{tcolorbox}

\begin{tcolorbox}[colback=gray!5!white, colframe=pastelgreen!40!black, title=A cost model for software maintenance \& evolution\cite{Sneed2004}]
	\begin{minipage}[t]{0.25\textwidth}
		\textbf{Study summary:}
	\end{minipage}
	\hfill
	\begin{minipage}[t]{0.65\textwidth}
		This study examines the cost of software maintenance implementation. It also investigates difficulties with the implementation of software maintenance and the evolution of software maintenance models. 
	\end{minipage}

	% Add space here
	\vspace{0.75em} 

	\begin{minipage}[t]{0.25\textwidth}
		\textbf{Method:}
	\end{minipage}
	\hfill
	\begin{minipage}[t]{0.65\textwidth}
		The study attempted to assess the cost involved for a software system by implementing a prediction cost analysis of multiple types of maintenance activities.
	\end{minipage}

	% Add space here
	\vspace{0.75em} 

	\begin{minipage}[t]{0.25\textwidth}
		\textbf{Results:}
	\end{minipage}
	\hfill
	\begin{minipage}[t]{0.65\textwidth}
		The results showed that the estimated maintenance costs depended on multiple factors, such as the size of the development team, the complexity of the code and the quality of the code.
	\end{minipage}

	% Add space here
	\vspace{0.75em} 

	\begin{minipage}[t]{0.25\textwidth}
		\textbf{Commentary:}
	\end{minipage}
	\hfill
	\begin{minipage}[t]{0.65\textwidth}
		This study highlights the need to implement software maintenance efficiently, but does not provide any solutions to improve it. It provides information on all the possible resource costs involved when implementing a software maintenance model.
	\end{minipage}
	\tcblower
	\textbf{Study keywords:} Maintenance Cost Estimation; Software Life Cycle Costing Models;
Software Maintenance and Evolution; Software Product Management
\end{tcolorbox}


\begin{tcolorbox}[colback=gray!5!white, colframe=pastelgreen!40!black, title=Trends in software maintenance tasks distribution among programmers: A study in a micro software company \cite{Stojanov2017}]
	\begin{minipage}[t]{0.25\textwidth}
		\textbf{Study summary:}
	\end{minipage}
	\hfill
	\begin{minipage}[t]{0.65\textwidth}
		This study examines the software maintenance efforts when implementing a software
maintenance model. It also looks at how software developers prioritise and divide these
maintenance tasks.
	\end{minipage}

	% Add space here
	\vspace{0.75em} 

	\begin{minipage}[t]{0.25\textwidth}
		\textbf{Method:}
	\end{minipage}
	\hfill
	\begin{minipage}[t]{0.65\textwidth}
		Maintenance tasks were classified into different types and distributed among developers with multiple test distribution techniques.
	\end{minipage}

	% Add space here
	\vspace{0.75em} 

	\begin{minipage}[t]{0.25\textwidth}
		\textbf{Results:}
	\end{minipage}
	\hfill
	\begin{minipage}[t]{0.65\textwidth}
		The distribution techniques effectively divide the maintenance task between the developers and improve the study scheduling.
	\end{minipage}

	% Add space here
	\vspace{0.75em} 

	\begin{minipage}[t]{0.25\textwidth}
		\textbf{Commentary:}
	\end{minipage}
	\hfill
	\begin{minipage}[t]{0.65\textwidth}
		This study's decision-making results are created from the maintenance tasks obtained by the developers. It specifically only looks at the contribution of these maintenance tasks from
each developer and how efficiently they distribute the work among them.
	\end{minipage}
	\tcblower
	\textbf{Study keywords:} Maintenance engineering, Market research, Companies, Software
maintenance
\end{tcolorbox}

\begin{tcolorbox}[colback=gray!5!white, colframe=pastelgreen!40!black, title=Supporting Software Architecture Maintenance by Providing Task-specific Recommendations \cite{Galster2019}]
	\begin{minipage}[t]{0.25\textwidth}
		\textbf{Study summary:}
	\end{minipage}
	\hfill
	\begin{minipage}[t]{0.65\textwidth}
		This study discusses the different types of maintenance implemented in the industry and the
		need for software maintenance. The study defines the types of maintenance and typical activities associated when implementing it for the Operations and Maintenance of \Cref{tbl:ch1_SDLC}.
	\end{minipage}

	% Add space here
	\vspace{0.75em} 

	\begin{minipage}[t]{0.25\textwidth}
		\textbf{Method:}
	\end{minipage}
	\hfill
	\begin{minipage}[t]{0.65\textwidth}
		Attempt to address developers' lack of information when trying to implement maintenance.
	\end{minipage}

	% Add space here
	\vspace{0.75em} 

	\begin{minipage}[t]{0.25\textwidth}
		\textbf{Results:}
	\end{minipage}
	\hfill
	\begin{minipage}[t]{0.65\textwidth}
		List multiple solutions for software maintenance recommendations.
	\end{minipage}

	% Add space here
	\vspace{0.75em} 

	\begin{minipage}[t]{0.25\textwidth}
		\textbf{Commentary:}
	\end{minipage}
	\hfill
	\begin{minipage}[t]{0.65\textwidth}
		These studies examine the software maintenance process in the industry, not its improvement. It mainly focusses on maintenance tasks and the correct approach to efficiently
implementing maintenance.
	\end{minipage}
	\tcblower
	\textbf{Study keywords:} Software maintenance; natural language processing; software
	architecture; text classification
\end{tcolorbox}

\begin{tcolorbox}[colback=gray!5!white, colframe=pastelgreen!40!black, title=Analyzing Forty Years of Software Maintenance Models\cite{Lenarduzzi2017}]
	\begin{minipage}[t]{0.25\textwidth}
		\textbf{Study summary:}
	\end{minipage}
	\hfill
	\begin{minipage}[t]{0.65\textwidth}
		This study will analyse software maintenance models and their characteristics for the last
four decades.
	\end{minipage}

	% Add space here
	\vspace{0.75em} 

	\begin{minipage}[t]{0.25\textwidth}
		\textbf{Method:}
	\end{minipage}
	\hfill
	\begin{minipage}[t]{0.65\textwidth}
		Studies were obtained and a systematic analysis of 1,044 articles between 1970 and 2015 on software maintenance models was implemented on the data.
	\end{minipage}

	% Add space here
	\vspace{0.75em} 

	\begin{minipage}[t]{0.25\textwidth}
		\textbf{Results:}
	\end{minipage}
	\hfill
	\begin{minipage}[t]{0.65\textwidth}
		The results showed the common aspects of software maintenance in the industry and common problems. These issues were that the research had limited third-party validation, did not always attempt to improve or build on existing software maintenance models, had limited comparison with the literature, and was difficult to replicate due to private data sets and custom tools
	\end{minipage}

	% Add space here
	\vspace{0.75em} 

	\begin{minipage}[t]{0.25\textwidth}
		\textbf{Commentary:}
	\end{minipage}
	\hfill
	\begin{minipage}[t]{0.65\textwidth}
		This study didn't provide any ways to improve software maintenance, but the limitations of
software maintenance in the industry. This study highlighted these limitations by the slow adaptability of software maintenance models, the lack of comparison in the literature due to close
source models and the replicability of the researched studies.
	\end{minipage}
	\tcblower
	\textbf{Study keywords:} Software Maintenance; Systematic review of the literature.
\end{tcolorbox}

\begin{tcolorbox}[colback=gray!5!white, colframe=pastelgreen!40!black, title=A Software Maintenance Methodology: An Approach Applied to Software Aging\cite{Araujo2021}]
	\begin{minipage}[t]{0.25\textwidth}
		\textbf{Study summary:}
	\end{minipage}
	\hfill
	\begin{minipage}[t]{0.65\textwidth}
		The study primarily discusses implementing suitable software maintenance models for software
maintenance activities through the characterisation of the software system.
	\end{minipage}

	% Add space here
	\vspace{0.75em} 

	\begin{minipage}[t]{0.25\textwidth}
		\textbf{Method:}
	\end{minipage}
	\hfill
	\begin{minipage}[t]{0.65\textwidth}
		The study implemented a software rejuvenation method by implementing maintenance on specific systems and evaluating the performance of the system afterward of these older systems.
	\end{minipage}

	% Add space here
	\vspace{0.75em} 

	\begin{minipage}[t]{0.25\textwidth}
		\textbf{Results:}
	\end{minipage}
	\hfill
	\begin{minipage}[t]{0.65\textwidth}
		The results showed that applying the software rejuvenation method on specific systems improved system performance.
	\end{minipage}

	% Add space here
	\vspace{0.75em} 

	\begin{minipage}[t]{0.25\textwidth}
		\textbf{Commentary:}
	\end{minipage}
	\hfill
	\begin{minipage}[t]{0.65\textwidth}
		The study doesn't provide a guideline to efficiently implement maintenance, but on which
software maintenance strategies can be used for older systems without doing full software maintenance.
	\end{minipage}
	\tcblower
	\textbf{Study keywords:} Software ageing and rejuvenation; methodology; software maintenance
\end{tcolorbox}

\begin{tcolorbox}[colback=gray!5!white, colframe=pastelgreen!40!black, title=Tools and Benchmarks for Automated Log Parsing\cite{Zhu2019}]
	\begin{minipage}[t]{0.25\textwidth}
		\textbf{Study summary:}
	\end{minipage}
	\hfill
	\begin{minipage}[t]{0.65\textwidth}
		This study focusses on log parsing and benchmarking different log parsing methods. 
	\end{minipage}

	% Add space here
	\vspace{0.75em} 

	\begin{minipage}[t]{0.25\textwidth}
		\textbf{Method:}
	\end{minipage}
	\hfill
	\begin{minipage}[t]{0.65\textwidth}
		Compare multiple third-party log parsers with each other with different benchmarks.
	\end{minipage}

	% Add space here
	\vspace{0.75em} 

	\begin{minipage}[t]{0.25\textwidth}
		\textbf{Results:}
	\end{minipage}
	\hfill
	\begin{minipage}[t]{0.65\textwidth}
		The results showed these log parses' robustness, efficiency and accuracy with the benchmarking data.
	\end{minipage}

	% Add space here
	\vspace{0.75em} 

	\begin{minipage}[t]{0.25\textwidth}
		\textbf{Commentary:}
	\end{minipage}
	\hfill
	\begin{minipage}[t]{0.65\textwidth}
		Highlight the efficiency that event logging needs to adhere to. The study uses mainly log
parsers for system diagnostics and not specifically for user-based event logging.
	\end{minipage}
	\tcblower
	\textbf{Study keywords:} AIOps; anomaly detection; log analysis; log management; log parsing
\end{tcolorbox}

\begin{tcolorbox}[colback=gray!5!white, colframe=pastelgreen!40!black, title=A Systematic Review of Logging Practice in Software Engineering\cite{Rong2018}]
	\begin{minipage}[t]{0.25\textwidth}
		\textbf{Study summary:}
	\end{minipage}
	\hfill
	\begin{minipage}[t]{0.65\textwidth}
		This study examines how logging practises are implemented in the industry. The study also
examines the number of publications on logging practises in software engineering, further
discussed in \Cref{apx:loggingPractice}. This emphasises the lack of research on logging practises.
	\end{minipage}

	% Add space here
	\vspace{0.75em} 

	\begin{minipage}[t]{0.25\textwidth}
		\textbf{Method:}
	\end{minipage}
	\hfill
	\begin{minipage}[t]{0.65\textwidth}
		The study created multiple research equations to obtain and analyse the research done for logging practise in software engineering.
	\end{minipage}

	% Add space here
	\vspace{0.75em} 

	\begin{minipage}[t]{0.25\textwidth}
		\textbf{Results:}
	\end{minipage}
	\hfill
	\begin{minipage}[t]{0.65\textwidth}
		The results showed that the relevant estimated research was carried out each year and the topic of development of each article was categorised.
	\end{minipage}

	% Add space here
	\vspace{0.75em} 

	\begin{minipage}[t]{0.25\textwidth}
		\textbf{Commentary:}
	\end{minipage}
	\hfill
	\begin{minipage}[t]{0.65\textwidth}
		This study further highlights the need to create comprehensive methods for developers to
implement a logging practise. There is not enough research done on logging practises and
specific use cases are beneficial to the software industry. 
	\end{minipage}
	\tcblower
	\textbf{Study keywords:} Logging Practise; Software Engineering; Systematic Literature Review
\end{tcolorbox}

\begin{tcolorbox}[colback=gray!5!white, colframe=pastelgreen!40!black, title=Learning to Log: Helping Developers Make Informed Logging Decisions \cite{Zhu2015}]
	\begin{minipage}[t]{0.25\textwidth}
		\textbf{Study summary:}
	\end{minipage}
	\hfill
	\begin{minipage}[t]{0.65\textwidth}
		This study summarises the importance of how and what to log in to help software.
		developers in creating efficient logging mechanisms. 
	\end{minipage}

	% Add space here
	\vspace{0.75em} 

	\begin{minipage}[t]{0.25\textwidth}
		\textbf{Method:}
	\end{minipage}
	\hfill
	\begin{minipage}[t]{0.65\textwidth}
		An automated \textit{LogAdvisor} tool for developers trained in different learning models.
	\end{minipage}

	% Add space here
	\vspace{0.75em} 

	\begin{minipage}[t]{0.25\textwidth}
		\textbf{Results:}
	\end{minipage}
	\hfill
	\begin{minipage}[t]{0.65\textwidth}
		The results showed the benchmark results of the tool performance with different learning models.
	\end{minipage}

	% Add space here
	\vspace{0.75em} 

	\begin{minipage}[t]{0.25\textwidth}
		\textbf{Commentary:}
	\end{minipage}
	\hfill
	\begin{minipage}[t]{0.65\textwidth}
		This study provides a generic method for the log analysis of events based on the intended log analysis.
		The study provides important guidelines discussed in the literature review for event logging
and log analysis, which should be part of the method of a new logging mechanism.
	\end{minipage}
	\tcblower
	\textbf{Study keywords:} keywords
\end{tcolorbox}

\begin{tcolorbox}[colback=gray!5!white, colframe=pastelgreen!40!black, title=Towards a better assessment of event logs quality\cite{Kherbouche2017}]
	\begin{minipage}[t]{0.25\textwidth}
		\textbf{Study summary:}
	\end{minipage}
	\hfill
	\begin{minipage}[t]{0.65\textwidth}
		This study examines what development practises can be implemented to improve or maintain
event log quality.
	\end{minipage}

	% Add space here
	\vspace{0.75em} 

	\begin{minipage}[t]{0.25\textwidth}
		\textbf{Method:}
	\end{minipage}
	\hfill
	\begin{minipage}[t]{0.65\textwidth}
		The study created metrics for the quality assessment of the event log in real-life and artificial case studies.
	\end{minipage}

	% Add space here
	\vspace{0.75em} 

	\begin{minipage}[t]{0.25\textwidth}
		\textbf{Results:}
	\end{minipage}
	\hfill
	\begin{minipage}[t]{0.65\textwidth}
		The results have been validated with natural and synthetic data that validated the quality metrics of the event log.	
	\end{minipage}

	% Add space here
	\vspace{0.75em} 

	\begin{minipage}[t]{0.25\textwidth}
		\textbf{Commentary:}
	\end{minipage}
	\hfill
	\begin{minipage}[t]{0.65\textwidth}
		The event logging quality assessment highlights the importance of maintaining an acceptable
log quality to make the logs consistent and complete. It also ensures that the logging mechanism can accurately capture the logs with the least structural and behavioural complexity. 
	\end{minipage}
	\tcblower
	\textbf{Study keywords:} event logs; process mining; process mining algorithms; qualitative
model
\end{tcolorbox}

\begin{tcolorbox}[colback=gray!5!white, colframe=pastelgreen!40!black, title=Central Audit Logging Mechanism in Personal Data Web Services\cite{Hasiloglu2018}]
	\begin{minipage}[t]{0.25\textwidth}
		\textbf{Study summary:}
	\end{minipage}
	\hfill
	\begin{minipage}[t]{0.65\textwidth}
		This study aims to create a logging mechanism to get audit logs from a Web application.
	\end{minipage}

	% Add space here
	\vspace{0.75em} 

	\begin{minipage}[t]{0.25\textwidth}
		\textbf{Method:}
	\end{minipage}
	\hfill
	\begin{minipage}[t]{0.65\textwidth}
		This study created a central logging mechanism for audit logs in web services.
	\end{minipage}

	% Add space here
	\vspace{0.75em} 

	\begin{minipage}[t]{0.25\textwidth}
		\textbf{Results:}
	\end{minipage}
	\hfill
	\begin{minipage}[t]{0.65\textwidth}
		The results of this study showed the use of an applied logging model to create the central audit logging mechanism for a Web service.	
	\end{minipage}

	% Add space here
	\vspace{0.75em} 

	\begin{minipage}[t]{0.25\textwidth}
		\textbf{Commentary:}
	\end{minipage}
	\hfill
	\begin{minipage}[t]{0.65\textwidth}
		This study did create a method for audit logs for a web service, but no log analysis was done afterward on the logs. The advantages and disadvantages of the created logging mechanism also explored the differences between client- and server-side logging mechanisms. 
	\end{minipage}
	\tcblower
	\textbf{Study keywords:} API; API Policy; Audit Logging; Personal Data; Web Service
\end{tcolorbox}

\begin{tcolorbox}[colback=gray!5!white, colframe=pastelgreen!40!black, title=User behavioural patterns and reduced user profiles extracted from log files\cite{Slaninova2014}]
	\begin{minipage}[t]{0.25\textwidth}
		\textbf{Study summary:}
	\end{minipage}
	\hfill
	\begin{minipage}[t]{0.65\textwidth}
		This study focusses on the behavioural patterns extracted from log files through log
analysis. Create user profiles to compare them with other individual users or groups of users.
	\end{minipage}

	% Add space here
	\vspace{0.75em} 

	\begin{minipage}[t]{0.25\textwidth}
		\textbf{Method:}
	\end{minipage}
	\hfill
	\begin{minipage}[t]{0.65\textwidth}
		The study analyses the behavioural patterns in a given set of log files generated by users interacting with the system. It attempts to get the similarities between the logs through specific log attributes.
	\end{minipage}

	% Add space here
	\vspace{0.75em} 

	\begin{minipage}[t]{0.25\textwidth}
		\textbf{Results:}
	\end{minipage}
	\hfill
	\begin{minipage}[t]{0.65\textwidth}
		The results show the similarities between users or groups of users when they interact with the system.	
	\end{minipage}

	\vspace{0.75em} 

	\begin{minipage}[t]{0.25\textwidth}
		\textbf{Commentary:}
	\end{minipage}
	\hfill
	\begin{minipage}[t]{0.65\textwidth}
		In this study, no logging mechanism method is provided that can be implemented on any
		system. It only implements a log analysis to create user profiles. It also explores what
events in a system can be seen as user-based events for the log analysis.
	\end{minipage}
	\tcblower
	\textbf{Study keywords:} Analysis of users' behaviour; Behavioural patterns; Complex networks;
User profiles
\end{tcolorbox}

\begin{tcolorbox}[colback=gray!5!white, colframe=pastelgreen!40!black, title=Tracking User Activities and Marketplace Dynamics in Classified Ads\cite{Waqar2017}]
	\begin{minipage}[t]{0.25\textwidth}
		\textbf{Study summary:}
	\end{minipage}
	\hfill
	\begin{minipage}[t]{0.65\textwidth}
		This study tracks users' activities on a Web application when users interact with ads. A log
		analysis was implemented to create a probabilistic model of user behaviour data based on the
ads the user has interacted with.
	\end{minipage}

	% Add space here
	\vspace{0.75em} 

	\begin{minipage}[t]{0.25\textwidth}
		\textbf{Method:}
	\end{minipage}
	\hfill
	\begin{minipage}[t]{0.65\textwidth}
		They created models to track user activities when interacting with Web page ads.
	\end{minipage}

	% Add space here
	\vspace{0.75em} 

	\begin{minipage}[t]{0.25\textwidth}
		\textbf{Results:}
	\end{minipage}
	\hfill
	\begin{minipage}[t]{0.65\textwidth}
		The results showed the performance and evaluation of each model used.
	\end{minipage}

	% Add space here
	\vspace{0.75em} 

	\begin{minipage}[t]{0.25\textwidth}
		\textbf{Commentary:}
	\end{minipage}
	\hfill
	\begin{minipage}[t]{0.65\textwidth}
		In this study, only user interactions with the ads are recorded for the log analysis. This
study shows a clear goal for the logging mechanism to capture certain user-based data for
user marketing profiling. In industry, this would be the main goal of organisations when
they attempt to track user-based data for commercial reasons.
	\end{minipage}
	\tcblower
	\textbf{Study keywords:} Classified ads; Temporal analysis; User modelling; User tracking
\end{tcolorbox}

\begin{tcolorbox}[colback=gray!5!white, colframe=pastelgreen!40!black, title=Analysis of visitor's behavior from Web Log using WebLog Expert Tool\cite{Kumar2017}]
	\begin{minipage}[t]{0.25\textwidth}
		\textbf{Study summary:}
	\end{minipage}
	\hfill
	\begin{minipage}[t]{0.65\textwidth}
		This study explains the log analysis for websites using a Web log tool that performs data processing, pattern discovery, and research. This log analysis aims to obtain the user's
activity data of the total time spent per webpage, which webpage they visited, and other visitor browser analysis data.
	\end{minipage}

	% Add space here
	\vspace{0.75em} 

	\begin{minipage}[t]{0.25\textwidth}
		\textbf{Method:}
	\end{minipage}
	\hfill
	\begin{minipage}[t]{0.65\textwidth}
		Use a Web Log Tool to capture certain event logs for a Web Log analysis of the user's behaviour.
	\end{minipage}

	% Add space here
	\vspace{0.75em} 

	\begin{minipage}[t]{0.25\textwidth}
		\textbf{Results:}
	\end{minipage}
	\hfill
	\begin{minipage}[t]{0.65\textwidth}
		The log data used in the log analysis could be used to improve the usability of the website.
	\end{minipage}

	% Add space here
	\vspace{0.75em} 

	\begin{minipage}[t]{0.25\textwidth}
		\textbf{Commentary:}
	\end{minipage}
	\hfill
	\begin{minipage}[t]{0.65\textwidth}
		This study doesn't provide a method for the logging mechanism, but rather the log analysis of
the user's data.  
	\end{minipage}
	\tcblower
	\textbf{Study keywords:} web server log; Web usage mining
\end{tcolorbox}

\newcommand{\problemStatement}{Software maintenance is a problem in the industry due to how inefficiently developers prioritise maintenance activities. A proven method to monitor software behaviours is event logging. The logging mechanism and log analysis must be explicitly designed for user-based events to improve software maintenance.}

\clearpage

\section{Problem statement}\label{sec:ch1_problemStatement}
Software maintenance is a vital part of the entire life cycle of any software system. \Cref{sec:ch1_maintenanceProblems} discussed why limited available resources make it more challenging to improve maintenance efforts. It is beneficial to prioritise maintenance to ensure that limited resources are used efficiently. \par A possible solution to this problem is using event logs to determine system usage due to event logging being a proven method in the industry to track system usage. Software developers can access third-party software logging tools to get the event logs. The majority of tools focus on system runtime utilisation rather than user activities. Logging tools that track user-based movements do exist, depending on the framework the software system is developed. \par Although there are live tools that track user-generated events, it is not guaranteed that log quality will be acceptable. Important logging attributes that will aid log analysis may not be logged. \par Designing and implementing a logging mechanism could bridge the gap between software maintenance and log analysis to create a system usage report. \par Software developers must still design the overall logging mechanism and decide where to place the logging points to capture the event logs in a software system. There are proven methods to create a suitable logging mechanism, but many do not include analysing the records for user-based utilisation to improve software maintenance. The problem statement can be summarised as:

\begin{center}
	\begin{tcolorbox}[colback=lightgray, colframe=black, sharp corners=all, arc=4pt]
		\begin{minipage}{\textwidth}
			\RaggedRight\textit{\problemStatement}
		\end{minipage}
	\end{tcolorbox}
\end{center}

\clearpage

\section{Objectives of the study}\label{sec:ch1_objectives}
In \Cref{sec:ch1_problemStatement}, the problem of efficiently implementing software maintenance has been identified. There is a need to create a system utilisation analysis report from event logs captured using a logging mechanism. \par The study is divided into two components to achieve the primary objective: design and implement the logarithmic mechanism. 

% Define a variable to store constant text
\newcommand{\objAi}{Investigate what will define a user-based event log and how to obtain the needed log attributes for a comprehensive user utilisation log analysis.}
\newcommand{\objAii}{Investigate where to place logging points in a software system to capture user-based event logs.}
\newcommand{\objAiii}{Investigate how to evaluate the log quality of the stored user-based event logs.}
\newcommand{\objAiv}{Investigate how to implement maintenance prioritisation.}

\newcommand{\objBi}{Implement the method to create a user-based logging mechanism on a test system}
\newcommand{\objBiSubA}{Define the user activity types and create the needed log attributes for a user-based event log.}
\newcommand{\objBiSubB}{Implement the logging points at strategic locations to capture the relevant log attributes to create event logs.}
\newcommand{\objBiSubC}{Use or create a log analysis tool to evaluate the log quality of the captured logs.}
\newcommand{\objBiSubD}{Implement a software maintenance prioritisation method in the log analysis.}
\newcommand{\objBii}{Verify the results for the test system's software maintenance prioritisation.}
\newcommand{\objBiii}{Validate the software maintenance prioritisation method by critically analysing the case studies' results.}


\subsection{Literature Objectives:}\label{sec:ch1_literatureObjective}
\begin{enumerate}
	\item \objAi
	\item \objAii
	\item \objAiii
	\item \objAiv 
\end{enumerate}

\subsection{Empirical objectives:}\label{sec:ch1_empiricalObjective}
\begin{enumerate}
	\item \objBi: 
		\begin{enumerate}
			\item \objBiSubA
			\item \objBiSubB
			\item \objBiSubD
			\item \objBiSubC
		\end{enumerate}
	\item \objBii
	\item \objBiii
\end{enumerate}

\clearpage

\section{Overview of the dissertation}
\subsubsection{Chapter 1: Introduction}
This chapter discusses the background of software maintenance, event logging, and system utilisation analysis. It defines the complexities and general issues of software maintenance for software developers. Efficient use of limited resources requires software developers to implement better software maintenance decisions. Event logging is a proven method to gather system information that can assist with prioritising software maintenance to implement it efficiently.

\subsubsection{Chapter 2: Methodology}
This chapter discusses the design of the generic method used to create a logging mechanism from a set of defined logging points and attributes. The software system for creating the logging mechanism is a web-based application. The second part of this chapter is the system utilisation analysis, using the captured logs to generate an analysis report.

\subsubsection{Chapter 3: Results}
This chapter contains the results of implementing a logging mechanism in a case study web-based application, as designed in Chapter 2. The results of the implemented logging mechanism are analysed as part of a system utilisation analysis, and recommendations are made on improving the maintenance of the case study web-based application. The results are discussed and validated to show how they address the problem statement and fulfil the study objectives.

\subsubsection{Chapter 4: Conclusion}
This chapter concludes by creating a logging mechanism for system utilisation analysis to improve software maintenance for a web-based application. Limitations and recommendations are also identified based on the methodology and results.
